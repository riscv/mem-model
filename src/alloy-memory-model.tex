\section{Formal Axiomatic Specification in Alloy}
\label{sec:alloy}

\lstdefinelanguage{alloy}{
  morekeywords={abstract, sig, extends, pred, fun, fact, no, set, one, lone, let, not, all, iden, some, run, for},
  morecomment=[l]{//},
  morecomment=[s]{/*}{*/},
  commentstyle=\color{green!40!black},
  keywordstyle=\color{blue!40!black},
  moredelim=**[is][\color{red}]{@}{@},
  escapeinside={!}{!},
}
\lstset{language=alloy}
\lstset{aboveskip=0pt}
\lstset{belowskip=0pt}

We present a formal specification of the RVWMO memory model in Alloy (\url{http://alloy.mit.edu}).
This model is available online at \url{https://github.com/daniellustig/riscv-memory-model}.

The online material also contains some litmus tests and some examples of how Alloy can be used to model check some of the mappings in Section~\ref{sec:porting}.

\begin{figure}[h!]
  {
  \tt\bfseries\centering\footnotesize
  \begin{lstlisting}
////////////////////////////////////////////////////////////////////////////////
// =RVWMO PPO=

// Preserved Program Order
fun ppo : Event->Event {
  // same-address ordering
  po_loc :> Store
  + rdw
  + (AMO + StoreConditional) <: rfi

  // explicit synchronization
  + ppo_fence
  + Acquire <: ^po :> MemoryEvent
  + MemoryEvent <: ^po :> Release
  + RCsc <: ^po :> RCsc

  // syntactic dependencies
  + addrdep
  + datadep
  + ctrldep :> Store

  // pipeline dependencies
  + (addrdep+datadep).rfi
  + addrdep.^po :> Store
}

// the global memory order respects preserved program order
fact { ppo in ^gmo }
\end{lstlisting}}
  \caption{The RVWMO memory model formalized in Alloy (1/5: PPO)}
  \label{fig:alloy1}
\end{figure}
\begin{figure}[h!]
  {
  \tt\bfseries\centering\footnotesize
  \begin{lstlisting}
////////////////////////////////////////////////////////////////////////////////
// =RVWMO axioms=

// Load Value Axiom
fun candidates[r: MemoryEvent] : set MemoryEvent {
  (r.~^gmo & Store & same_addr[r]) // writes preceding r in gmo
  + (r.^~po & Store & same_addr[r]) // writes preceding r in po
}

fun latest_among[s: set Event] : Event { s - s.~^gmo }

pred LoadValue {
  all w: Store | all r: Load |
    w->r in rf <=> w = latest_among[candidates[r]]
}

// Atomicity Axiom
pred Atomicity {
  all r: Store.~pair |            // starting from the lr,
    no x: Store & same_addr[r] |  // there is no store x to the same addr
      x not in same_hart[r]       // such that x is from a different hart,
      and x in r.~rf.^gmo         // x follows (the store r reads from) in gmo,
      and r.pair in x.^gmo        // and r follows x in gmo
}

// Progress Axiom - implicit

pred RISCV_mm { LoadValue and Atomicity /* and Progress */ }

\end{lstlisting}}
  \caption{The RVWMO memory model formalized in Alloy (2/5: Axioms)}
  \label{fig:alloy2}
\end{figure}
\begin{figure}[h!]
  {
  \tt\bfseries\centering\footnotesize
  \begin{lstlisting}
////////////////////////////////////////////////////////////////////////////////
// Basic model of memory

sig Hart {  // hardware thread
  start : one Event
}
sig Address {}
abstract sig Event {
  po: lone Event // program order
}

// .aq and .rl represent the opcode bits.  The ordering annotations assigned
// to those bits are described further down.
abstract sig MemoryEvent extends Event {
  address: one Address,
  acquireRCpc: lone MemoryEvent,
  acquireRCsc: lone MemoryEvent,
  releaseRCpc: lone MemoryEvent,
  releaseRCsc: lone MemoryEvent,
  addrdep: set MemoryEvent,
  ctrldep: set Event,
  datadep: set MemoryEvent,
  gmo: set MemoryEvent,  // global memory order
  rf: set MemoryEvent
}
sig LoadNormal extends MemoryEvent {} // l{b|h|w|d}
sig LoadReserve extends MemoryEvent { // lr
  pair: lone StoreConditional
}
sig StoreNormal extends MemoryEvent {}       // s{b|h|w|d}
// all StoreConditionals in the model are assumed to be successful
sig StoreConditional extends MemoryEvent {}  // sc
sig AMO extends MemoryEvent {}               // amo
sig NOP extends Event {}

fun Load : Event { LoadNormal + LoadReserve + AMO }
fun Store : Event { StoreNormal + StoreConditional + AMO }

sig Fence extends Event {
  pr: lone Fence, // opcode bit
  pw: lone Fence, // opcode bit
  sr: lone Fence, // opcode bit
  sw: lone Fence  // opcode bit
}
sig FenceTSO extends Fence {}

/* Alloy encoding detail: opcode bits are either set (encoded, e.g.,
 * as f.pr in iden) or unset (f.pr not in iden).  The bits cannot be used for
 * anything else */
fact { pr + pw + sr + sw in iden }
// likewise for ordering annotations
fact { acquireRCpc + acquireRCsc + releaseRCpc + releaseRCsc in iden }
// don't try to encode FenceTSO via pr/pw/sr/sw; just use it as-is
fact { no FenceTSO.(pr + pw + sr + sw) }
\end{lstlisting}}
  \caption{The RVWMO memory model formalized in Alloy (3/5: model of memory)}
  \label{fig:alloy3}
\end{figure}

\begin{figure}[h!]
  {
  \tt\bfseries\centering\footnotesize
  \begin{lstlisting}
////////////////////////////////////////////////////////////////////////////////
// =Basic model rules=

// Ordering annotation groups
fun Acquire : MemoryEvent { MemoryEvent.acquireRCpc + MemoryEvent.acquireRCsc }
fun Release : MemoryEvent { MemoryEvent.releaseRCpc + MemoryEvent.releaseRCsc }
fun RCpc : MemoryEvent { MemoryEvent.acquireRCpc + MemoryEvent.releaseRCpc }
fun RCsc : MemoryEvent { MemoryEvent.acquireRCsc + MemoryEvent.releaseRCsc }

// There is no such thing as store-acquire or load-release, unless it's both
fact { Load & Release in Acquire }
fact { Store & Acquire in Release }

// FENCE PPO
fun FencePRSR : Fence { Fence.(pr & sr) }
fun FencePRSW : Fence { Fence.(pr & sw) }
fun FencePWSR : Fence { Fence.(pw & sr) }
fun FencePWSW : Fence { Fence.(pw & sw) }

fun ppo_fence : MemoryEvent->MemoryEvent {
    (Load  <: ^po :> FencePRSR).(^po :> Load)
  + (Load  <: ^po :> FencePRSW).(^po :> Store)
  + (Store <: ^po :> FencePWSR).(^po :> Load)
  + (Store <: ^po :> FencePWSW).(^po :> Store)
  + (Load  <: ^po :> FenceTSO) .(^po :> MemoryEvent)
  + (Store <: ^po :> FenceTSO) .(^po :> Store)
}

// auxiliary definitions
fun po_loc : Event->Event { ^po & address.~address }
fun same_hart[e: Event] : set Event { e + e.^~po + e.^po }
fun same_addr[e: Event] : set Event { e.address.~address }

// initial stores
fun NonInit : set Event { Hart.start.*po }
fun Init : set Event { Event - NonInit }
fact { Init in StoreNormal }
fact { Init->(MemoryEvent & NonInit) in ^gmo }
fact { all e: NonInit | one e.*~po.~start }  // each event is in exactly one hart
fact { all a: Address | one Init & a.~address } // one init store per address
fact { no Init <: po and no po :> Init }
\end{lstlisting}}
  \caption{The RVWMO memory model formalized in Alloy (4/5: Basic model rules)}
  \label{fig:alloy4}
\end{figure}

\begin{figure}[h!]
  {
  \tt\bfseries\centering\footnotesize
  \begin{lstlisting}
// po
fact { acyclic[po] }

// gmo
fact { total[^gmo, MemoryEvent] } // gmo is a total order over all MemoryEvents

//rf
fact { rf.~rf in iden } // each read returns the value of only one write
fact { rf in Store <: address.~address :> Load }
fun rfi : MemoryEvent->MemoryEvent { rf & (*po + *~po) }

//dep
fact { no StoreNormal <: (addrdep + ctrldep + datadep) }
fact { addrdep + ctrldep + datadep + pair in ^po }
fact { pair in datadep }
fact { datadep in datadep :> Store }
fact { ctrldep.*po in ctrldep }
fact { no pair & (^po :> (LoadReserve + StoreConditional)).^po }
fact { StoreConditional in LoadReserve.pair } // assume all SCs succeed

// rdw
fun rdw : Event->Event {
  (Load <: po_loc :> Load)  // start with all same_address load-load pairs,
  - (~rf.rf)                // subtract pairs that read from the same store,
  - (po_loc.rfi)            // and subtract out "fri-rfi" patterns
}

// filter out redundant instances and/or visualizations
fact { no gmo & gmo.gmo } // keep the visualization uncluttered
fact { all a: Address | some a.~address }

////////////////////////////////////////////////////////////////////////////////
// =Optional: opcode encoding restrictions=

// the list of blessed fences
fact { Fence in
  Fence.pr.sr
  + Fence.pw.sw
  + Fence.pr.pw.sw
  + Fence.pr.sr.sw
  + FenceTSO
  + Fence.pr.pw.sr.sw
}

pred restrict_to_current_encodings {
  no (LoadNormal + StoreNormal) & (Acquire + Release)
}

////////////////////////////////////////////////////////////////////////////////
// =Alloy shortcuts=
pred acyclic[rel: Event->Event] { no iden & ^rel }
pred total[rel: Event->Event, bag: Event] {
  all disj e, e': bag | e->e' in rel + ~rel
  acyclic[rel]
}
\end{lstlisting}}
  \caption{The RVWMO memory model formalized in Alloy (5/5: Auxiliaries)}
  \label{fig:alloy5}
\end{figure}

