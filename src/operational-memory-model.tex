\section{An Operational Memory Model}
\label{sec:operational}
This is an alternative presentation of the RVWMO memory model in
operational style.
%
It aims to admit exactly the same extensional behaviour as the
axiomatic presentation: for any given program, admitting an execution
if and only if the axiomatic presentation allows it.

The axiomatic presentation is defined as a predicate on complete
candidate executions.  In contrast, this operational presentation has
an abstract microarchitectural flavour: it is expressed as a state
machine, with states that are an abstract representation of hardware
machine states, and with explicit out-of-order and speculative
execution
(but abstracting from more implementation-specific microarchitectural
details such as register renaming, store buffers, cache hierarchies, cache protocols, etc.).
As such, it can provide useful intuition.
It can also
construct executions incrementally, making it possible to
interactively and randomly explore the behaviour of larger examples,
while the axiomatic model requires complete candidate executions
over which the axioms can be checked.

The operational presentation covers mixed-size execution, with
potentially overlapping memory accesses of different power-of-two byte
sizes.  Misaligned accesses are broken up into single-byte accesses.

An interactive version of the model, together with a library of litmus tests,
is provided online: \url{http://www.cl.cam.ac.uk/~pes20/rmem}.
This is integrated with a fragment of the RISC-V ISA semantics
(RV64I and A) expressed explicitly in Sail
(\url{https://github.com/rems-project/sail})).
% TODO: compare with the herd and alloy versions


Below is an informal introduction of the model states and transitions.
The description of the formal model starts in the next subsection.


Terminology: In contrast to the axiomatic presentation, here every memory operation is either a load or a store.
Hence, AMOs give rise to two distinct memory operations, a load and a store.
When used in conjunction with ``instruction'', the terms ``load'' and ``store'' refer to instructions that give rise to such memory operations.
As such, both include AMO instructions.
The term ``acquire'' refers to instruction (or its memory operation) with the acquire-RCpc or acquire-RCsc annotation.
The term ``release'' refers to instruction (or its memory operation) with the release-RCpc or release-RCsc annotation.

\paragraph{Model states}
A model state consists of a shared memory and a tuple of hart states.
\begin{center}
\sffamily
\begin{tabular}{ccc}
\cline{1-1}\cline{3-3}
\multicolumn{1}{|c|}{Hart 0} & \bf \dots & \multicolumn{1}{|c|}{Hart $n$} \\
\cline{1-1}\cline{3-3}
$\big\uparrow$ $\big\downarrow$ & & $\big\uparrow$ $\big\downarrow$ \\
\hline
\multicolumn{3}{|c|}{Shared Memory} \\
\hline
\end{tabular}
\end{center}
The shared memory state records all the memory store operations that have propagated so far, in the order they propagated (this can be made more efficient, but for simplicity of the presentation we keep it this way).

Each hart state consists principally of a tree of instruction instances, some of which have been \emph{finished}, and some of which have not.
Non-finished instruction instances can be subject to \emph{restart}, e.g.~if they depend on an out-of-order or speculative load that turns out to be unsound.

Conditional branch and indirect jump instructions may have multiple successors in the instruction tree.
When such instruction is finished, any un-taken alternative paths are discarded.

Each instruction instance in the instruction tree has a state that
includes an execution state of the intra-instruction semantics (the
ISA pseudocode for this instruction).
The model uses a formalisation of the intra-instruction semantics in Sail.
One can think of the execution state of an instruction as a representation of the pseudocode control state, pseudocode call stack, and local variable values.
An instruction instance state also includes information about the instance's memory and register footprints, its register reads and writes, its memory operations, whether it is finished, etc.

\paragraph{Model transitions}
The model defines, for any model state, the set of allowed transitions, each of which is a single atomic step to a new abstract machine state.
Execution of a single instruction will typically involve many
transitions, and they may be interleaved in operational-model
execution with transitions arising from other instructions. 
Each transition arises from a single instruction instance; it will
change the state of that instance, and it may depend on or change the
rest of its hart state and the shared memory state, but it does not depend on other hart states, and it will not change them.
% Instructions cannot be treated
% as atomic units: complete execution of a single instruction instance may
% involve many transitions, which can be interleaved with those of other
% instances in the same or other harts, and some of this is programmer-visible.
The transitions are introduced below and defined in Section~\ref{sec:omm:transitions}, with a precondition and a construction of the post-transition model state for each.

\noindent Transitions for all instructions:
\begin{itemize}
\item \nameref{omm:fetch}: This transition represents a fetch and
  decode of a new instruction instance, as a program order successor
  of a previously fetched instruction instance (or the initial fetch
  address).

The model assumes the instruction memory is fixed; it does not
describe the behaviour of self-modifying code. 
In particular, the \nameref{omm:fetch} transition does not generate memory load operations, and the shared memory is not involved in the transition.
Instead, the model depends on an external oracle that provides an opcode when given a memory location.
%


\item[$\circ$] \nameref{omm:reg_write}: This is a write of a register value.
\item[$\circ$] \nameref{omm:reg_read}: This is a read of a register
  value from the most recent program-order-predecessor instruction instance that writes to that register.
\item[$\circ$] \nameref{omm:sail_interp}: This covers pseudocode
  internal computation: arithmetic, function calls, etc.
\item[$\circ$] \nameref{omm:finish}: At this point the instruction pseudocode is done, the instruction cannot be restarted, memory accesses cannot be discarded, and all memory effects have taken place.
For conditional branch and indirect jump instructions, any program order successors that were fetched from an address that is not the one that was written to the {\em pc} register are discarded, together with the sub-tree of instruction instances below them.
\end{itemize}

\noindent Transitions specific to load instructions:
\begin{itemize}
\item[$\circ$] \nameref{omm:initiate_load}: At this point the memory
  footprint of the load instruction is provisionally known (it could change if
  earlier instructions are restarted) and its individual memory load operations can start being satisfied.
\item \nameref{omm:sat_by_forwarding}: This partially or entirely
  satisfies a single memory load operation by forwarding, from
  program-order-previous memory store operations.
\item \nameref{omm:sat_from_mem}: This entirely satisfies the outstanding slices of a single memory load operation, from memory.
\item[$\circ$] \nameref{omm:complete_loads}: At this point all the memory load operations of the instruction have been entirely satisfied and the instruction pseudocode can continue executing.
A load instruction can be subject to being restarted until the \nameref{omm:finish} transition.
But, under some conditions, the model might treat a load instruction as non-restartable even before it is finished (e.g.~see \nameref{omm:prop_store}).
\end{itemize}

\noindent Transitions specific to store instructions:
\begin{itemize}
\item[$\circ$] \nameref{omm:initiate_store_footprint}: At this point the memory footprint of the store is provisionally known.
\item[$\circ$] \nameref{omm:instantiate_store_value}: At this point the memory store operations have their values and program-order-successor memory load operations can be satisfied by forwarding from them.
\item[$\circ$] \nameref{omm:commit_stores}: At this point the store operations are guaranteed to happen (the instruction can no longer be restarted or discarded), and they can start being propagated to memory.
\item \nameref{omm:prop_store}: This propagates a single memory store operation to memory.
\item[$\circ$] \nameref{omm:complete_stores}: At this point all the memory store operations of the instruction have been propagated to memory, and the instruction pseudocode can continue executing.
\end{itemize}

\noindent Transitions specific to {\tt  sc} instructions:
\begin{itemize}
\item \nameref{omm:early_sc_fail}: This causes the {\tt sc} to fail, either a spontaneous fail or because it is not paired with a program-order-previous {\tt lr}.
\item \nameref{omm:paired_sc}: This transition indicates the {\tt sc} is paired with an {\tt lr} and might succeed.
\item \nameref{omm:commit_sc}: This is an atomic execution of the transitions \nameref{omm:commit_stores} and \nameref{omm:prop_store}, it is enabled only if the stores from which the {\tt lr} read from have not been overwritten.
\item \nameref{omm:late_sc_fail}: This causes the {\tt sc} to fail, either a spontaneous fail or because the stores from which the {\tt lr} read from have been overwritten.
\end{itemize}

\noindent Transitions specific to AMO instructions:
\begin{itemize}
\item \nameref{omm:do_amo}: This is an atomic execution of all the transitions needed to satisfy the load operation, do the required arithmetic, and propagate the store operation.
\end{itemize}

\noindent Transitions specific to fence instructions:
\begin{itemize}
\item[$\circ$] \nameref{omm:commit_fence}
\end{itemize}

The transitions labelled~$\circ$ can always be taken eagerly, as soon as their precondition is satisfied, without excluding other behaviour; the $\bullet$ cannot.
Although \nameref{omm:fetch} is marked with a $\bullet$, it can be taken eagerly as long as it is not taken infinitely many times.

An instance of a non-AMO load instruction, after being fetched, will typically experience the following transitions in this order:
\begin{enumerate}
\item \nameref{omm:reg_read}
\item \nameref{omm:initiate_load}
\item \nameref{omm:sat_by_forwarding} and/or \nameref{omm:sat_from_mem} (as many as needed to satisfy all the load operations of the instance)
\item \nameref{omm:complete_loads}
\item \nameref{omm:reg_write}
\item \nameref{omm:finish}
\end{enumerate}
Before, between and after the transitions above, any number of \nameref{omm:sail_interp} transitions may appear.
In addition, a \nameref{omm:fetch} transition for fetching the instruction in the next program location will be available until it is taken.


This concludes the informal description of the operational model.
The following sections describe the formal operational model.

\subsection{Intra-instruction Pseudocode Execution}\label{sec:omm:pseudocode_exec}
The intra-instruction semantics for each instruction instance is expressed as a state machine, essentially running the instruction pseudocode.
Given a pseudocode execution state, it computes the next state.  Most
states identify a pending memory or register operation, requested by
the pseudocode, which the memory model has to do.  The
states are (this is a tagged union; tags in small-caps):

\begin{center}
\begin{tabular}{l@{ \quad-\quad }l}
{\sc Load\_mem}({\it kind}, {\it address}, {\it size}, {\it load\_continuation})
    & memory load operation\\
{\sc Early\_sc\_fail}({\it res\_continuation})
    & allow {\tt sc} to fail early\\
{\sc Store\_ea}({\it kind}, {\it address}, {\it size}, {\it next\_state})
    & memory store effective address\\
{\sc Store\_memv}({\it mem\_value}, {\it store\_continuation})
    & memory store value\\
{\sc Fence}({\it kind}, {\it next\_state})
    & fence\\
{\sc Read\_reg}({\it reg\_name}, {\it read\_continuation})
    & register read\\
{\sc Write\_reg}({\it reg\_name}, {\it reg\_value}, {\it next\_state})
    & register write\\
{\sc Internal}({\it next\_state})
    & pseudocode internal step\\
{\sc Done}
    & end of pseudocode\\
\end{tabular}
\end{center}
Here:
\begin{tightlist}
\item {\it mem\_value} and {\it reg\_value} are lists of bytes;
\item {\it address} is an integer of XLEN bits;
\item for load/store, {\it kind} identifies whether it is {\tt lr/sc}, acquire-RCpc/release-RCpc, acquire-RCsc/release-RCsc, acquire-release-RCsc;
\item for fence, {\it kind} identifies whether it is a normal or TSO, and (for normal fences) the predecessor and successor ordering bits;
\item {\it reg\_name} identifies a register and a slice thereof (start and
  end bit indices); and
\item the continuations describe how the instruction instance will continue for each value that might be provided by the surrounding memory model (the {\it load\_continuation} and {\it read\_continuation} take the value loaded from memory and read from the previous register write, the {\it store\_continuation} takes {\it false} for an {\tt sc} that failed and {\it true} in all other cases, and {\it res\_continuation} takes {\it false} if the {\tt sc} fails and {\it true} otherwise).
\end{tightlist}

\begin{commentary}
For example, given the load instruction \verb!lw x1,0(x2)!,
an execution will typically go as follows.
The initial execution state will be computed from the pseudocode for the given opcode.
This can be expected to be {\sc Read\_reg}({\tt x2}, {\it read\_continuation}).
Feeding the most recently written value of register {\tt x2} (the instruction semantics will be blocked if necessary until the register value is available), say {\tt 0x4000}, to {\it read\_continuation} returns {\sc Load\_mem}({\tt plain\_load}, {\tt 0x4000}, {\tt 4}, {\it load\_continuation}).
Feeding the 4-byte value loaded from memory location {\tt 0x4000}, say {\tt 0x42}, to {\it load\_continuation} returns
{\sc Write\_reg}({\tt x1}, {\tt 0x42}, {\sc Done}).
Many {\sc Internal}({\it next\_state}) states may appear before and between the states above.
\end{commentary}

Notice that writing to memory is split into two steps, {\sc Store\_ea} and {\sc Store\_memv}: the first one makes the memory footprint of the store provisionally known, and the second one adds the value to be stored.
We ensure these are paired in the pseudocode ({\sc Store\_ea} followed by {\sc Store\_memv}), but there may be other steps between them.
\begin{commentary}
It is observable that the {\sc Store\_ea} can occur before the value to be stored is determined.
For example, for the litmus test LB+fence.r.rw+data-po to be allowed by the operational model (as it is by RVWMO), the first store in Hart 1 has to take the {\sc Store\_ea} step before its value is determined, so that the second store can see it is to a non-overlapping memory footprint, allowing the second store to be committed out of order without violating coherence.
\end{commentary}

The pseudocode of each instruction performs at most one store or one load, except for AMOs that perform exactly one load and one store.
Those memory accesses are then split apart into the architecturally atomic units by the hart semantics (see \nameref{omm:initiate_load} and \nameref{omm:initiate_store_footprint} below).

Informally, each bit of a register read should be satisfied from a register write by the most recent (in program order) instruction instance that can write that bit (or from the hart's initial register state if there is no such write).
Hence, it is essential to know the register write footprint of each instruction instance, which we calculate when the instruction instance is created (see the action of \nameref{omm:fetch} below).
We ensure in the pseudocode that each instruction does at most one register write to each register bit, and also that it does not try to read a register value it just wrote.

Data-flow dependencies (address and data) in the model emerge from the
fact that each register read has to wait for the appropriate register write to be executed (as described above).

\subsection{Instruction Instance State}\label{sec:omm:inst_state}
Each instruction instance $i$ has a state comprising:
\begin{itemize}
\item {\it program\_loc}, the memory address from which the instruction was fetched;
\item {\it instruction\_kind}, identifying whether this is a load, store, AMO, fence, branch/jump or a `simple' instruction (this also includes a {\it kind} similar to the one described for the pseudocode execution states);
\item {\it src\_regs}, the set of source {\it reg\_name}s (including system registers), as statically determined from the pseudocode of the instruction;
\item {\it dst\_regs}, the destination {\it reg\_name}s (including system registers), as statically determined from the pseudocode of the instruction;
\item {\it pseudocode\_state} (or sometimes just `state' for short), one of (this is a tagged union; tags in small-caps):
  \begin{center}
  \begin{tabular}{l@{ \quad-\quad }l}
  {\sc Plain}({\it isa\_state})
    & ready to make a pseudocode transition \\
  {\sc Pending\_mem\_loads}({\it load\_continuation})
    & requesting memory load operation(s) \\
  {\sc Pending\_mem\_stores}({\it store\_continuation})
    & requesting memory store operation(s) \\
%   {\sc Pending\_exception}({\it exception}) & performing an exception;
  \end{tabular}
  \end{center}

\item {\it reg\_reads}, the register reads the instance has performed, including, for each one, the register write slices it read from;
\item {\it reg\_writes}, the register writes the instance has performed;
\item {\it mem\_loads}, a set of memory load operations, and for each one
  the as-yet-unsatisfied slices (the byte indices that have not been
  satisfied yet), and, for the satisfied slices, the store slices
  (each consisting of a memory store operation and subset of its byte indices) that satisfied it.
\item {\it mem\_stores}, a set of memory store operations, and for each one a flag that indicates whether it has been propagated (passed to the shared memory) or not.
\item information recording whether the instance is committed, finished, etc.
\end{itemize}

Each memory load operation includes a memory footprint (address and size).
Each memory store operations includes a memory footprint, and, when available, a value.

A load instruction instance with a non-empty {\it mem\_loads}, for which all the load operations are satisfied (i.e.~there are no unsatisfied load slices) is said to be {\it entirely satisfied}.

Informally, an instruction instance is said to have {\it fully determined data} if the load (and {\tt sc}) instructions feeding its source registers are finished.
Similarly, it is said to have a {\it fully determined memory footprint} if the load (and {\tt sc}) instructions feeding its memory operation address register are finished.
%
Formally, we first define the notion of {\it fully determined register write}: a register write $w$ from {\it reg\_writes} of instruction instance $i$ is said to be {\it fully determined} if one of the following conditions hold:
\begin{enumerate}
\item $i$ is finished; or
\item the value written by $w$ is not affected by a memory operation that $i$ has made (i.e. a value loaded from memory or the result of {\tt sc}), and, for every register read that $i$ has made, that affects $w$, the register write from which $i$ read is fully determined (or $i$ read from the initial register state).
\end{enumerate}
Now, an instruction instance $i$ is said to have  {\it fully determined data} if for every register read $r$ from {\it reg\_reads}, the register writes that $r$ reads from are fully determined.
An instruction instance $i$ is said to have a {\it fully determined memory footprint} if for every register read $r$ from {\it reg\_reads} that feeds into $i$'s memory operation address, the register writes that $r$ reads from are fully determined.
\begin{commentary}
The {\tt rmem} tool records, for every register write, the set of register writes from other instructions that have been read by this instruction at the point of performing the write.
By carefully arranging the pseudocode of the instructions covered by the tool we were able to make it so that this is exactly the set of register writes on which the write depends on.
\end{commentary}


\subsection{Hart State}
The model state of a single hart comprises:
\begin{itemize}
\item {\it hart\_id}, a unique identifier of the hart;
%\item {\it register\_data}, the name, bit width, and start bit index for each register;
\item {\it initial\_register\_state}, the initial register value for each register;
\item {\it initial\_fetch\_address}, the initial instruction fetch address;
\item {\it instruction\_tree}, a tree of the instruction instances that have been fetched (and not discarded), in program order.
\end{itemize}


\subsection{Shared Memory State}
The model state of the shared memory comprises a list of memory store operations, in the order they propagated to the shared memory.

When a store operation is propagated to the shared memory it is simply added to the end of the list.
When a load operation is satisfied from memory, for each byte of the load operation, the most recent corresponding store slice is returned.

\begin{commentary}
For most purposes, it is simpler to think of the shared memory as an array, i.e., a map from memory locations to memory store operation slices, where each memory location is mapped to a one-byte slice of the most recent memory store operation to that location.
However, this abstraction is not detailed enough to properly handle the {\tt sc} instruction.
The RVWMO \nameref{rvwmo:ax:atom} allows store operations from the same hart as the {\tt sc} to intervene between the store operation of the {\tt sc} and the store operations the paired {\tt lr} read from.
To allow such store operations to intervene, and forbid others, the array abstraction must be extended to record more information.
Here, we use a list as it is very simple, but a more efficient and scalable implementations should probably use something better.
\end{commentary}


\subsection{Transitions}\label{sec:omm:transitions}

Each of the paragraphs below describes a single kind of system transition.
The description starts with a condition over the current system state.
The transition can be taken in the current state only if the condition is satisfied.
The condition is followed by an action that is applied to that state when the transition is taken, in order to generate the new system state.

\paragraph{Fetch instruction}\label{omm:fetch}
A possible program-order-successor of instruction instance $i$ can be fetched from address {\it loc} if:
\begin{enumerate}
\item it has not already been fetched, i.e., none of the immediate successors of $i$ in the hart's {\it instruction\_tree} are from {\it loc}; and
\item if $i$'s pseudocode has already wrote an address to {\em pc}, then {\it loc} must be that address, otherwise {\it loc} is:
  \begin{itemize}
  \item for a conditional branch, the successor address and the branch target address;
  \item for a (direct) jump and link instruction ({\tt jal}), the target address;
  \item for an indirect jump instruction ({\tt jalr}), any address; and
  \item for any other instruction, $i.\textit{program\_loc}+4$.
  \end{itemize}
\end{enumerate}
% \fixme{Does an instruction at the end of memory need special-case treatment?}

Action: construct a freshly initialized instruction instance $i'$ for the instruction in the program memory at {\it loc}, with state {\sc Plain}({\it isa\_state}), computed from the instruction pseudocode, including the static information available from the pseudocode such as its {\it instruction\_kind}, {\it src\_regs}, and {\it dst\_regs}, and add $i'$ to the hart's {\it instruction\_tree} as a successor of $i$.
% If the instruction fails to decode, set the state of $i'$ to ``{\sc Pending\_exception} {\it exception}'' with {\it exception} that describes the decoding error.

\begin{commentary}
The possible next fetch addresses ({\it loc}) are available immediately after fetching $i$ and the model does not need to wait for the pseudocode to write to {\em pc}; this allows out-of-order execution, and speculation past conditional branches and jumps.
For most instructions these addresses are easily obtained from the instruction pseudocode.
The only exception to that is the indirect jump instruction ({\tt jalr}), where the address depends on the value held in a register.
%
In principle the mathematical model should allow speculation to
arbitrary addresses here. 
%
The exhaustive search in the {\tt rmem} tool handles this by running the exhaustive search multiple times with a growing set of possible next fetch addresses for each indirect jump.
The initial search uses empty sets, hence there is no fetch after indirect jump instruction until the pseudocode of the instruction writes to {\em pc}, and then we use that value for fetching the next instruction.
Before starting the next iteration of exhaustive search, we collect for each indirect jump (grouped by code location) the set of values it wrote to {\em pc} in all the executions in the previous search iteration, and use that as possible next fetch addresses of the instruction.
This process terminates when no new fetch addresses are detected.
\end{commentary}

\paragraph{Initiate memory load operations}\label{omm:initiate_load}
An instruction instance $i$ in state {\sc Plain}({\sc Load\_mem}({\it kind}, {\it address}, {\it size}, {\it load\_continuation})) can always initiate the corresponding memory load operations.
Action:
\begin{enumerate}
\item Construct the appropriate memory load operations $mlos$:
  \begin{itemize}
  \item if {\it address} is aligned to {\it size} then $mlos$ is a single memory load operation of {\it size} bytes from {\it address};
  \item otherwise, $mlos$ is a set of {\it size} memory load operations, each of one byte, from the addresses $\textit{address}\ldots\textit{address}+\textit{size}-1$.
  \end{itemize}
\item set {\it mem\_loads} of $i$ to $mlos$; and
\item update the state of $i$ to {\sc Pending\_mem\_loads}({\it load\_continuation}).
\end{enumerate}

\begin{commentary}
In Section~\ref{sec:rvwmo:primitives} it is said that misaligned memory accesses may be decomposed at any granularity.
Here we decompose them to one-byte accesses as this granularity subsumes all others.
\end{commentary}

\paragraph{Satisfy memory load operation by forwarding from unpropagated stores}\label{omm:sat_by_forwarding}
For a non-AMO load instruction instance $i$ in state {\sc Pending\_mem\_loads}({\it load\_continuation}), and a memory load operation $mlo$ in $i.\textit{mem\_loads}$ that has unsatisfied slices, the memory load operation can be partially or entirely satisfied by forwarding from unpropagated memory store operations by store instruction instances that are program-order-before $i$ if:
\begin{enumerate}
\item all program-order-previous {\tt fence} instructions with {\tt .sr} set are finished;
\item for every program-order-previous {\tt fence.tso} instruction, $f$, that is not finished, all load instructions that are program-order-before $f$ are finished;
% \item all program-order-previous {\tt fence.i} instructions are finished;
\item if $i$ is a load-acquire-RCsc, all program-order-previous store-releases-RCsc are finished;
\item if $i$ is a load-acquire-release, all program-order-previous instructions are finished;
\item all non-finished program-order-previous load-acquire instructions are entirely satisfied; and
\item all program-order-previous store-acquire-release instructions are finished;
\end{enumerate}

Let $msoss$ be the set of all unpropagated memory store operation slices from non-{\tt sc} store instruction instances that are program-order-before $i$ and have already calculated the value to be stored, that overlap with the unsatisfied slices of $mlo$, and which are not superseded by intervening stores operations or store operations that are read from by an intervening loads.
The last condition requires, for each memory store operation slice $msos$ in $msoss$ from instruction $i'$:
\begin{itemize}
\item that there is no store instruction program-order-between $i$ and $i'$ with a memory store operation overlapping $msos$; and
\item that there is no load instruction program-order-between $i$ and $i'$ that was satisfied from an overlapping memory store operation slice from a different hart.
\end{itemize}
Action:
\begin{enumerate}
\item update $i.\textit{mem\_loads}$ to indicate that $mlo$ was satisfied by $msoss$; and
\item restart any speculative instructions which have violated coherence as a result of this, i.e., for every non-finished instruction $i'$ that is a program-order-successor of $i$, and every memory load operation $mlo'$ of $i'$ that was satisfied from $msoss'$, if there exists a memory store operation slice $msos'$ in $msoss'$, and an overlapping memory store operation slice from a different memory store operation in $msoss$, and $msos'$ is not from an instruction that is a program-order-successor of $i$, restart $i'$ and its {\em restart-dependents}.
\end{enumerate}
Where, the {\em restart-dependents} of instruction $j$ are:
\begin{itemize}
\item if $j$ is a load-acquire, all the program-order-successors of $j$;
\item program-order-successors of $j$ that have data-flow dependency on a register write of $j$;
\item program-order-successors of $j$ that have a memory load operation that reads from a memory store operation of $j$ (by forwarding); and
\item (recursively) all the restart-dependents of all the instruction instances above.
\end{itemize}

\begin{commentary}
Forwarding memory store operations to a memory load might satisfy only some slices of the load, leaving other slices unsatisfied.

A program-order-previous store operation that was not available when taking the transition above might make $msoss$ unsound (violating coherence) when it becomes available.
That store will prevent the load from being finished (see \nameref{omm:finish}), and will cause it to restart when that store operation is propagated (see \nameref{omm:prop_store}).

A consequence of the transition condition above is that store-release-RCsc memory store operations cannot be forwarded to load-acquire-RCsc instructions:
$msoss$ does not include memory store operations from finished stores (as those must be propagated memory store operations), and the condition above requires all program-order-previous store-releases-RCsc to be finished when the load is acquire-RCsc.
\end{commentary}


\paragraph{Satisfy memory load operation from memory}\label{omm:sat_from_mem}
For an instruction instance $i$ of a non-AMO load instruction or an AMO instruction in the context of the ``\nameref{omm:do_amo}'' transition, any memory load operation $mlo$ in $i.\textit{mem\_loads}$ that has unsatisfied slices, can be satisfied from memory if:
\begin{enumerate}
\item all program-order-previous {\tt fence} instructions with {\tt .sr} set are finished;
\item for every program-order-previous {\tt fence.tso} instruction, $f$, that is not finished, all load instructions that are program-order-before $f$ are finished;
% \item all program-order-previous {\tt fence.i} instructions are finished;
\item if $i$ is a load-acquire-RCsc, all program-order-previous store-releases-RCsc are finished;
\item if $i$ is a load-acquire-release, all program-order-previous instructions are finished;
\item all non-finished program-order-previous load-acquire instructions are entirely satisfied; and
\item all program-order-previous store-acquire-release instructions are finished.
\end{enumerate}
Action: let $msoss$ be the memory store operation slices from memory covering the unsatisfied slices of $mlo$, and apply the action of \nameref{omm:sat_by_forwarding}.

\begin{commentary}
Note that \nameref{omm:sat_by_forwarding} might leave some slices of the memory load operation unsatisfied, those will have to be satisfied by taking the transition again, or taking \nameref{omm:sat_from_mem}.
\nameref{omm:sat_from_mem}, on the other hand, will always satisfy all the unsatisfied slices of the memory load operation.
\end{commentary}


\paragraph{Complete load operations}\label{omm:complete_loads}
A load instruction instance $i$ in state {\sc Pending\_mem\_loads}({\it load\_continuation}) can be completed (not to be confused with finished) if all the memory load operations $i.\textit{mem\_loads}$ are entirely satisfied (i.e.~there are no unsatisfied slices).
Action: update the state of $i$ to {\sc Plain}({\it load\_continuation(mem\_value)}), where {\it mem\_value} is assembled from all the memory store operation slices that satisfied $i.\textit{mem\_loads}$.


\paragraph{Early {\tt sc} fail}\label{omm:early_sc_fail}
An {\tt sc} instruction instance $i$ in state {\sc Plain}({\sc Early\_sc\_fail}({\it res\_continuation})) can always be made to fail.
Action: update the state of $i$ to {\sc Plain}({\it res\_continuation(false)}).


\paragraph{Paired {\tt sc}}\label{omm:paired_sc}
An {\tt sc} instruction instance $i$ in state {\sc Plain}({\sc Early\_sc\_fail}({\it res\_continuation})) can continue its (potentially successful) execution if $i$ is paired with an {\tt lr}.
Action: update the state of $i$ to {\sc Plain}({\it res\_continuation(true)}).


\paragraph{Initiate memory store operation footprints}\label{omm:initiate_store_footprint}
An instruction instance $i$ in state {\sc Plain}({\sc Store\_ea}({\it kind}, {\it address}, {\it size}, {\it next\_state})) can always announce its pending memory store operation footprint.
Action:
\begin{enumerate}
\item construct the appropriate memory store operations $msos$ (without the store value):
  \begin{itemize}
  \item if {\it address} is aligned to {\it size} then $msos$ is a single memory store operation of {\it size} bytes to {\it address};
  \item otherwise, $msos$ is a set of {\it size} memory store operations, each of one-byte size, to the addresses $\textit{address}\ldots\textit{address}+\textit{size}-1$.
  \end{itemize}
\item set $i.\textit{mem\_stores}$ to $msos$; and
\item update the state of $i$ to {\sc Plain}({\it next\_state}).
\end{enumerate}

\begin{commentary}
Note that after taking the transition above the memory store operations do not yet have their values.
The importance of splitting this transition from the transition below is that it allows other program-order-successor store instructions to observe the memory footprint of this instruction, and if they don't overlap, propagate out of order as early as possible (i.e.~before the data register value becomes available).
\end{commentary}


\paragraph{Instantiate memory store operation values}\label{omm:instantiate_store_value}
An instruction instance $i$ in state {\sc Plain}({\sc Store\_memv}({\it mem\_value}, {\it store\_continuation})) can always instantiate the values of the memory store operations $i.\textit{mem\_stores}$.
Action:
\begin{enumerate}
\item split {\it mem\_value} between the memory store operations $i.\textit{mem\_stores}$; and
\item update the state of $i$ to {\sc Pending\_mem\_stores}({\it store\_continuation}).
\end{enumerate}


\paragraph{Commit store instruction}\label{omm:commit_stores}
An uncommitted instruction instance $i$ of a non-{\tt sc} store instruction or an {\tt sc} instruction in the context of the ``\nameref{omm:commit_sc}'' transition, in state {\sc Pending\_mem\_stores}({\it store\_continuation}), can be committed (not to be confused with propagated) if:
\begin{enumerate}
\item $i$ has fully determined data;
\item all program-order-previous conditional branch and indirect jump instructions are finished;
\item all program-order-previous {\tt fence} instructions with {\tt .sw} set are finished;
\item all program-order-previous {\tt fence.tso} instructions are finished;
% \item all program-order-previous {\tt fence.i} instructions are finished;
\item all program-order-previous load-acquire instructions are finished;
\item all program-order-previous store-acquire-release instructions are finished;
\item  if $i$ is a store-release, all program-order-previous instructions are finished;
\item\label{omm:commit_store:prev_addrs} all program-order-previous memory access instructions have a fully determined memory footprint;
\item\label{omm:commit_store:prev_stores} all program-order-previous store instructions, except for {\tt sc} that failed, have initiated and so have non-empty {\it mem\_stores}; and
\item\label{omm:commit_store:prev_loads} all program-order-previous load instructions have initiated and so have non-empty {\it mem\_loads}.
\end{enumerate}
Action: record that $i$ is committed.

\begin{commentary}
Notice that if condition \ref{omm:commit_store:prev_addrs} is satisfied the conditions \ref{omm:commit_store:prev_stores} and \ref{omm:commit_store:prev_loads} are also satisfied, or will be satisfied after taking some eager transitions.
Hence, requiring them does not strengthen the model.
By requiring them, we guarantee that previous memory access instructions have taken enough transitions to make their memory operations visible for the condition check of \nameref{omm:prop_store}, which is the next transition the instruction will take, making that condition simpler.
\end{commentary}


\paragraph{Propagate store operation}\label{omm:prop_store}
For a committed instruction instance $i$ in state {\sc Pending\_mem\_stores}({\it store\_continuation}), and an unpropagated memory store operation $mso$ in $i.\textit{mem\_stores}$, $mso$ can be propagated if:
\begin{enumerate}
\item all memory store operations of program-order-previous store instructions that overlap with $mso$ have already propagated;
\item all memory load operations of program-order-previous load instructions that overlap with $mso$ have already been satisfied, and (the load instructions) are {\em non-restartable} (see definition below); and
\item all memory load operations that were satisfied by forwarding $mso$ are entirely satisfied.
\end{enumerate}
Where a non-finished instruction instance $j$ is {\em non-restartable} if:
\begin{enumerate}
\item there does not exist a store instruction $s$ and an unpropagated memory store operation $mso$ of $s$ such that applying the action of the ``\nameref{omm:prop_store}'' transition to $mso$ will result in the restart of $j$; and
\item there does not exist a non-finished load instruction $l$ and a memory load operation $mlo$ of $l$ such that applying the action of the ``\nameref{omm:sat_by_forwarding}''/``\nameref{omm:sat_from_mem}'' transition (even if $mlo$ is already satisfied) to $mlo$ will result in the restart of $j$.
\end{enumerate}
Action:
\begin{enumerate}
\item update the shared memory state with $mso$;
\item update $i.\textit{mem\_stores}$ to indicate that $mso$ was propagated; and
\item restart any speculative instructions which have violated coherence as a result of this, i.e., for every non-finished instruction $i'$ program-order-after $i$ and every memory load operation $mlo'$ of $i'$ that was satisfied from $msoss'$, if there exists a memory store operation slice $msos'$ in $msoss'$ that overlaps with $mso$ and is not from $mso$, and $msos'$ is not from a program-order-successor of $i$, restart $i'$ and its {\em restart-dependents} (see \nameref{omm:sat_by_forwarding}).
\end{enumerate}


\paragraph{Commit and propagate store operation of an {\tt sc}}\label{omm:commit_sc}
An uncommitted {\tt sc} instruction instance $i$, from hart $h$, in state {\sc Pending\_mem\_stores}({\it store\_continuation}), with a paired {\tt lr} $i'$ that has been satisfied by some store slices $msoss$, can be committed and propagated at the same time if:
\begin{enumerate}
\item $i'$ is finished;
\item every memory store operation that has been forwarded to $i'$ is propagated;
\item the conditions of \nameref{omm:commit_stores} is satisfied;
\item the conditions of \nameref{omm:prop_store} is satisfied (notice that an {\tt sc} instruction can only have one memory store operation); and
\item for every store slice $msos$ from $msoss$, $msos$ has not been overwritten, in the shared memory, by a store that is from a hart that is not $h$, at any point since $msos$ was propagated to memory.
\end{enumerate}
Action:
\begin{enumerate}
\item apply the actions of \nameref{omm:commit_stores}; and
\item apply the action of \nameref{omm:prop_store}.
\end{enumerate}


\paragraph{Late {\tt sc} fail}\label{omm:late_sc_fail}
An {\tt sc} instruction instance $i$ in state {\sc Pending\_mem\_stores}({\it store\_continuation}), that has not propagated its memory store operation, can always be made to fail.
Action:
\begin{enumerate}
\item clear $i.\textit{mem\_stores}$; and
\item update the state of $i$ to {\sc Plain}({\it store\_continuation(false)}).
\end{enumerate}

\begin{commentary}
For efficiency, the {\tt rmem} tool allows this transition only when it is not possible to take the \nameref{omm:commit_sc} transition.
This does not affect the set of allowed final states, but when explored interactively, if the {\tt sc} should fail one should use the \nameref{omm:early_sc_fail} transition instead of waiting for this transition.
\end{commentary}

\paragraph{Complete store operations}\label{omm:complete_stores}
A store instruction instance $i$ in state {\sc Pending\_mem\_stores}({\it store\_continuation}), for which all the memory store operations in $i.\textit{mem\_stores}$ have been propagated, can always be completed (not to be confused with finished).
Action: update the state of $i$ to {\sc Plain}({\it store\_continuation(true)}).


\paragraph{Satisfy, commit and propagate operations of an AMO}\label{omm:do_amo}
An AMO instruction instance $i$ in state {\sc Pending\_mem\_loads}({\it load\_continuation}) can perform its memory access if it is possible to perform the following sequence of transitions with no intervening transitions:
\begin{enumerate}
\item \nameref{omm:sat_from_mem}
\item \nameref{omm:complete_loads}
\item \nameref{omm:sail_interp} (zero or more times)
\item \nameref{omm:instantiate_store_value}
\item \nameref{omm:commit_stores}
\item \nameref{omm:prop_store}
\item \nameref{omm:complete_stores}
\end{enumerate}
Action: perform the above sequence of transitions, one after the other, with no intervening transitions.

\begin{commentary}
Notice that program-order-previous stores cannot be forwarded to the load of an AMO.
This is simply because the sequence of transitions above does not include the forwarding transition.
But even if it did include it, the sequence will fail when trying to do the \nameref{omm:prop_store} transition, as this transition requires all program-order-previous store operations to overlapping memory footprints to be propagated, and forwarding requires the store operation to be unpropagated.

In addition, the store of an AMO cannot be forwarded to a program-order-successor load.
Before taking the transition above, the store operation of the AMO does not have its value and therefore cannot be forwarded; after taking the transition above the store operation is propagated and therefore cannot be forwarded.
\end{commentary}


\paragraph{Commit fence}\label{omm:commit_fence}
A fence instruction instance $i$ in state {\sc Plain}({\sc Fence}({\it kind}, {\it next\_state})) can be committed if:
\begin{enumerate}
\item if $i$ is a normal fence and it has {\tt .pr} set, all program-order-previous load instructions are finished;
\item if $i$ is a normal fence and it has {\tt .pw} set, all program-order-previous store instructions are finished; and
\item if $i$ is a {\tt fence.tso}, all program-order-previous load and store instructions are finished.
% \item if $i$ is a {\tt fence.i} instruction, all program-order-previous memory access instructions have fully determined memory footprints.
\end{enumerate}
Action:
\begin{enumerate}
\item record that $i$ is committed; and
\item update the state of $i$ to {\sc Plain}({\it next\_state}).
\end{enumerate}


\paragraph{Register read}\label{omm:reg_read}
An instruction instance $i$ in state {\sc Plain}({\sc Read\_reg}({\it reg\_name}, {\it read\_cont})) can do a register read of {\it reg\_name} if every instruction instance that it needs to read from has already performed the expected {\it reg\_name} register write.

Let {\it read\_sources} include, for each bit of {\it reg\_name}, the write to
that bit by the most recent (in program order) instruction instance that can write to that bit, if any. If there is no such instruction, the source is the initial register value from {\it initial\_register\_state}.
Let  {\it reg\_value} be the value assembled from {\it read\_sources}.
Action:
\begin{enumerate}
\item add {\it reg\_name} to $i.\textit{reg\_reads}$ with {\it read\_sources} and {\it reg\_value}; and
\item update the state of $i$ to {\sc Plain}({\it read\_cont(reg\_value)}).
\end{enumerate}


\paragraph{Register write}\label{omm:reg_write}
An instruction instance $i$ in state {\sc Plain}({\sc Write\_reg}({\it reg\_name}, {\it reg\_value}, {\it next\_state})) can always do a {\it reg\_name} register write.
Action:
\begin{enumerate}
\item add {\it reg\_name} to $i.\textit{reg\_writes}$ with $deps$ and {\it reg\_value}; and
\item update the state of $i$ to {\sc Plain}({\it next\_state}).
\end{enumerate}
where $deps$ is a pair of the set of all {\it read\_sources} from $i.\textit{reg\_reads}$, and a flag that is true iff $i$ is a load instruction instance that has already been entirely satisfied.


\paragraph{Pseudocode internal step}\label{omm:sail_interp}
An instruction instance $i$ in state {\sc Plain}({\sc Internal}({\it next\_state})) can always do that pseudocode-internal step.
Action: update the state of $i$ to {\sc Plain}({\it next\_state}).


\paragraph{Finish instruction}\label{omm:finish}
A non-finished instruction instance $i$ in state {\sc Plain}({\sc Done}) can be finished if:
\begin{enumerate}
\item if $i$ is a load instruction:
  \begin{enumerate}
  \item all program-order-previous load-acquire instructions are finished; and
  \item it is guaranteed that the values read by the memory load operations of $i$ will not cause coherence violations, i.e., for any program-order-previous instruction instance $i'$, let $\textit{cfp}$ be the combined footprint of propagated memory store operations from store instructions program-order-between $i$ and $i'$, and {\em fixed memory store operations} that were forwarded to $i$ from store instructions program-order-between $i$ and $i'$ including $i'$, and let $\overline{\textit{cfp}}$ be the complement of $\textit{cfp}$ in the memory footprint of $i$.
  If $\overline{\textit{cfp}}$ is not empty:
    \begin{enumerate}
    \item $i'$ has a fully determined memory footprint;
    \item $i'$ has no unpropagated memory store operations that overlap with $\overline{\textit{cfp}}$; and
    \item if $i'$ is a load with a memory footprint that overlaps with $\overline{\textit{cfp}}$, then all the memory load operations of $i'$ that overlap with $\overline{\textit{cfp}}$ are satisfied and $i'$ is {\em non-restartable} (see the \nameref{omm:prop_store} transition for how to determined if an instruction is non-restartable).
    \end{enumerate}
  Here, a memory store operation is called fixed if the store instruction has fully determined data.
  \end{enumerate}
\item $i$ has a fully determined data; and
\item if $i$ is not a fence, all program-order-previous conditional branch and indirect jump instructions are finished.
\end{enumerate}
Action:
\begin{enumerate}
\item if $i$ is a conditional branch or indirect jump instruction, discard any untaken paths of execution, i.e., remove all instruction instances that are not reachable by the branch/jump taken in {\it instruction\_tree}; and
\item record the instruction as finished, i.e., set {\it finished} to {\it true}.
\end{enumerate}


\subsection{Limitations}\label{sec:omm:limitations}
\begin{itemize}
\item The model covers user-level RV64I and RV64A.
In particular, it does not support the misaligned atomics extension ``Zam'' or the total store ordering extension ``Ztso''.
It should be trivial to adapt the model to RV32I/A and to the G, Q and C extensions, but we have never tried it. This will involve, mostly, writing Sail code for the instructions, with minimal, if any, changes to the concurrency model.
% unsupported (system) instructions: ecall, ebreak, csrrw, csrrs, csrrc, csrrwi, csrrsi, csrrci
\item The model covers only normal memory accesses (it does not handle I/O accesses).
\item The model does not cover TLB-related effects.
\item The model assumes the instruction memory is fixed.
In particular, the \nameref{omm:fetch} transition does not generate memory load operations, and the shared memory is not involved in the transition.
Instead, the model depends on an external oracle that provides an opcode when given a memory location.
\item The model does not cover exceptions, traps and interrupts.
\end{itemize}

