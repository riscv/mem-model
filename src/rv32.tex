\chapter{RV32I Base Integer Instruction Set, Version 2.0}
\label{rv32}

This chapter describes version 2.0 of the RV32I base integer
instruction set.  Much of the commentary also applies to the RV64I
variant.

\begin{commentary}
RV32I was designed to be sufficient to form a compiler target and to
support modern operating system environments.  The ISA was also
designed to reduce the hardware required in a minimal implementation.
RV32I contains 47 unique instructions, though a simple implementation
might cover the eight ECALL/EBREAK/CSRR* instructions with a single
SYSTEM hardware instruction that always traps and might be able to
implement the FENCE and FENCE.I instructions as NOPs, reducing
hardware instruction count to 38 total.  RV32I can emulate almost any
other ISA extension (except the A extension, which requires additional
hardware support for atomicity).
\end{commentary}

\section{Programmers' Model for Base Integer Subset}

Figure~\ref{gprs} shows the user-visible state for the base integer
subset.  There are 31 general-purpose registers {\tt x1}--{\tt x31},
which hold integer values.  Register {\tt x0} is hardwired to the
constant 0.  There is no hardwired subroutine return address link
register, but the standard software calling convention uses register
{\tt x1} to hold the return address on a call.  For RV32, the {\tt x}
registers are 32 bits wide, and for RV64, they are 64 bits wide.  This
document uses the term XLEN to refer to the current width of an {\tt
  x} register in bits (either 32 or 64).

There is one additional user-visible register: the program counter {\tt pc}
holds the address of the current instruction.

\begin{commentary}
The number of available architectural registers can have large impacts
on code size, performance, and energy consumption.  Although 16
registers would arguably be sufficient for an integer ISA running
compiled code, it is impossible to encode a complete ISA with 16
registers in 16-bit instructions using a 3-address format.  Although a
2-address format would be possible, it would increase instruction
count and lower efficiency.  We wanted to avoid intermediate
instruction sizes (such as Xtensa's 24-bit instructions) to simplify
base hardware implementations, and once a 32-bit instruction size was
adopted, it was straightforward to support 32 integer registers.  A
larger number of integer registers also helps performance on
high-performance code, where there can be extensive use of loop
unrolling, software pipelining, and cache tiling.

For these reasons, we chose a conventional size of 32 integer
registers for the base ISA.  Dynamic register usage tends to be
dominated by a few frequently accessed registers, and regfile
implementations can be optimized to reduce access energy for the
frequently accessed registers~\cite{jtseng:sbbci}.  The optional
compressed 16-bit instruction format mostly only accesses 8 registers
and hence can provide a dense instruction encoding, while additional
instruction-set extensions could support a much larger register space
(either flat or hierarchical) if desired.

For resource-constrained embedded applications, we have defined the
RV32E subset, which only has 16 registers (Chapter~\ref{rv32e}).
\end{commentary}

\begin{figure}[H]
{\footnotesize
\begin{center}
\begin{tabular}{p{2in}}
\instbitrange{XLEN-1}{0}                                  \\ \cline{1-1}
\multicolumn{1}{|c|}{\reglabel{\ \ \ \ \ \ x0 / zero}}      \\ \cline{1-1}
\multicolumn{1}{|c|}{\reglabel{\ \ \ \ x1\ \ \ \ \ }}            \\ \cline{1-1}
\multicolumn{1}{|c|}{\reglabel{\ \ \ \ x2\ \ \ \ \ }}       \\ \cline{1-1}
\multicolumn{1}{|c|}{\reglabel{\ \ \ \ x3\ \ \ \ \ }}       \\ \cline{1-1}
\multicolumn{1}{|c|}{\reglabel{\ \ \ \ x4\ \ \ \ \ }}       \\ \cline{1-1}
\multicolumn{1}{|c|}{\reglabel{\ \ \ \ x5\ \ \ \ \ }}       \\ \cline{1-1}
\multicolumn{1}{|c|}{\reglabel{\ \ \ \ x6\ \ \ \ \ }}       \\ \cline{1-1}
\multicolumn{1}{|c|}{\reglabel{\ \ \ \ x7\ \ \ \ \ }}       \\ \cline{1-1}
\multicolumn{1}{|c|}{\reglabel{\ \ \ \ x8\ \ \ \ \ }}       \\ \cline{1-1}
\multicolumn{1}{|c|}{\reglabel{\ \ \ \ x9\ \ \ \ \ }}       \\ \cline{1-1}
\multicolumn{1}{|c|}{\reglabel{\ \ \ x10\ \ \ \ \ }}        \\ \cline{1-1}
\multicolumn{1}{|c|}{\reglabel{\ \ \ x11\ \ \ \ \ }}        \\ \cline{1-1}
\multicolumn{1}{|c|}{\reglabel{\ \ \ x12\ \ \ \ \ }}        \\ \cline{1-1}
\multicolumn{1}{|c|}{\reglabel{\ \ \ x13\ \ \ \ \ }}        \\ \cline{1-1}
\multicolumn{1}{|c|}{\reglabel{\ \ \ x14\ \ \ \ \ }}        \\ \cline{1-1}
\multicolumn{1}{|c|}{\reglabel{\ \ \ x15\ \ \ \ \ }}        \\ \cline{1-1}
\multicolumn{1}{|c|}{\reglabel{\ \ \ x16\ \ \ \ \ }}        \\ \cline{1-1}
\multicolumn{1}{|c|}{\reglabel{\ \ \ x17\ \ \ \ \ }}        \\ \cline{1-1}
\multicolumn{1}{|c|}{\reglabel{\ \ \ x18\ \ \ \ \ }}        \\ \cline{1-1}
\multicolumn{1}{|c|}{\reglabel{\ \ \ x19\ \ \ \ \ }}        \\ \cline{1-1}
\multicolumn{1}{|c|}{\reglabel{\ \ \ x20\ \ \ \ \ }}        \\ \cline{1-1}
\multicolumn{1}{|c|}{\reglabel{\ \ \ x21\ \ \ \ \ }}        \\ \cline{1-1}
\multicolumn{1}{|c|}{\reglabel{\ \ \ x22\ \ \ \ \ }}        \\ \cline{1-1}
\multicolumn{1}{|c|}{\reglabel{\ \ \ x23\ \ \ \ \ }}        \\ \cline{1-1}
\multicolumn{1}{|c|}{\reglabel{\ \ \ x24\ \ \ \ \ }}        \\ \cline{1-1}
\multicolumn{1}{|c|}{\reglabel{\ \ \ x25\ \ \ \ \ }}        \\ \cline{1-1}
\multicolumn{1}{|c|}{\reglabel{\ \ \ x26\ \ \ \ \ }}        \\ \cline{1-1}
\multicolumn{1}{|c|}{\reglabel{\ \ \ x27\ \ \ \ \ }}        \\ \cline{1-1}
\multicolumn{1}{|c|}{\reglabel{\ \ \ x28\ \ \ \ \ }}        \\ \cline{1-1}
\multicolumn{1}{|c|}{\reglabel{\ \ \ x29\ \ \ \ \ }}        \\ \cline{1-1}
\multicolumn{1}{|c|}{\reglabel{\ \ \ x30\ \ \ \ \ }}        \\ \cline{1-1}
\multicolumn{1}{|c|}{\reglabel{\ \ \ x31\ \ \ \ \ }}        \\ \cline{1-1}
\multicolumn{1}{c}{XLEN}                                  \\

\instbitrange{XLEN-1}{0}                                  \\ \cline{1-1}
\multicolumn{1}{|c|}{\reglabel{pc}}                         \\ \cline{1-1}
\multicolumn{1}{c}{XLEN}                                  \\
\end{tabular}
\end{center}
}
\caption{RISC-V user-level base integer register state.}
\label{gprs}
\end{figure}

\newpage

\section{Base Instruction Formats}

In the base ISA, there are four core instruction formats (R/I/S/U), as
shown in Figure~\ref{fig:baseinstformats}.  All are a fixed 32 bits in
length and must be aligned on a four-byte boundary in memory.  An instruction address misaligned exception is generated on a
taken branch or unconditional jump if the target address is not
four-byte aligned.  No instruction fetch misaligned exception is
generated for a conditional branch that is not taken.

\begin{commentary}
The alignment constraint for base ISA instructions is relaxed to a
two-byte boundary when instruction extensions with 16-bit lengths or
other odd multiples of 16-bit lengths are added.
\end{commentary}

\vspace{-0.2in}
\begin{figure}[h]
\begin{center}
\setlength{\tabcolsep}{4pt}
\begin{tabular}{p{1.2in}@{}p{0.8in}@{}p{0.8in}@{}p{0.6in}@{}p{0.8in}@{}p{1in}l}
\\
\instbitrange{31}{25} &
\instbitrange{24}{20} &
\instbitrange{19}{15} &
\instbitrange{14}{12} &
\instbitrange{11}{7} &
\instbitrange{6}{0} \\
\cline{1-6}
\multicolumn{1}{|c|}{funct7} &
\multicolumn{1}{c|}{rs2} &
\multicolumn{1}{c|}{rs1} &
\multicolumn{1}{c|}{funct3} &
\multicolumn{1}{c|}{rd} &
\multicolumn{1}{c|}{opcode} &
R-type \\
\cline{1-6}
\\
\cline{1-6}
\multicolumn{2}{|c|}{imm[11:0]} &
\multicolumn{1}{c|}{rs1} &
\multicolumn{1}{c|}{funct3} &
\multicolumn{1}{c|}{rd} &
\multicolumn{1}{c|}{opcode} &
I-type \\
\cline{1-6}
\\
\cline{1-6}
\multicolumn{1}{|c|}{imm[11:5]} &
\multicolumn{1}{c|}{rs2} &
\multicolumn{1}{c|}{rs1} &
\multicolumn{1}{c|}{funct3} &
\multicolumn{1}{c|}{imm[4:0]} &
\multicolumn{1}{c|}{opcode} &
S-type \\
\cline{1-6}
\\
\cline{1-6}
\multicolumn{4}{|c|}{imm[31:12]} &
\multicolumn{1}{c|}{rd} &
\multicolumn{1}{c|}{opcode} &
U-type \\
\cline{1-6}
\end{tabular}
\end{center}
\caption{RISC-V base instruction formats.  Each immediate subfield is
  labeled with the bit position (imm[{\em x}\,]) in the immediate
  value being produced, rather than the bit position within the
  instruction's immediate field as is usually done.  }
\label{fig:baseinstformats}
\end{figure}

The RISC-V ISA keeps the source ({\em rs1} and {\em rs2}) and
destination ({\em rd}) registers at the same position in all formats
to simplify decoding.  Except for the 5-bit immediates used in CSR
instructions (Section~\ref{sec:csrinsts}), immediates are always
sign-extended, and are generally packed towards the leftmost available
bits in the instruction and have been allocated to reduce hardware
complexity.  In particular, the sign bit for all immediates is always
in bit 31 of the instruction to speed sign-extension circuitry.

\begin{commentary}
Decoding register specifiers is usually on the critical paths in
implementations, and so the instruction format was chosen to keep all
register specifiers at the same position in all formats at the expense
of having to move immediate bits across formats (a property shared
with RISC-IV aka. SPUR~\cite{spur-jsscc1989}).

In practice, most immediates are either small or require all XLEN
bits.  We chose an asymmetric immediate split (12 bits in regular
instructions plus a special load upper immediate instruction with 20
bits) to increase the opcode space available for regular instructions.

Immediates are sign-extended because we did not observe a benefit to
using zero-extension for some immediates as in the MIPS ISA and wanted
to keep the ISA as simple as possible.
\end{commentary}

\section{Immediate Encoding Variants}

There are a further two variants of the instruction formats (B/J)
based on the handling of immediates, as shown in
Figure~\ref{fig:baseinstformatsimm}.

The only difference between the S and B formats is that the 12-bit
immediate field is used to encode branch offsets in multiples of 2 in
the B format.  Instead of shifting all bits in the
instruction-encoded immediate left by one in hardware as is
conventionally done, the middle bits (imm[10:1]) and sign bit stay in
fixed positions, while the lowest bit in S format (inst[7]) encodes a
high-order bit in B format.

Similarly, the only difference between the U and J formats is
that the 20-bit immediate is shifted left by 12 bits to form U
immediates and by 1 bit to form J immediates.  The location of
instruction bits in the U and J format immediates is chosen to
maximize overlap with the other formats and with each other.

\begin{figure}[h]
\begin{small}
\begin{center}
\setlength{\tabcolsep}{4pt}
\begin{tabular}{p{0.3in}@{}p{0.8in}@{}p{0.6in}@{}p{0.18in}@{}p{0.7in}@{}p{0.6in}@{}p{0.6in}@{}p{0.3in}@{}p{0.5in}l}
\\
\multicolumn{1}{c}{\instbit{31}} &
\instbitrange{30}{25} &
\instbitrange{24}{21} &
\multicolumn{1}{c}{\instbit{20}} &
\instbitrange{19}{15} &
\instbitrange{14}{12} &
\instbitrange{11}{8} &
\multicolumn{1}{c}{\instbit{7}} &
\instbitrange{6}{0} \\
\cline{1-9}
\multicolumn{2}{|c|}{funct7} &
\multicolumn{2}{c|}{rs2} &
\multicolumn{1}{c|}{rs1} &
\multicolumn{1}{c|}{funct3} &
\multicolumn{2}{c|}{rd} &
\multicolumn{1}{c|}{opcode} &
R-type \\
\cline{1-9}
\\
\cline{1-9}
\multicolumn{4}{|c|}{imm[11:0]} &
\multicolumn{1}{c|}{rs1} &
\multicolumn{1}{c|}{funct3} &
\multicolumn{2}{c|}{rd} &
\multicolumn{1}{c|}{opcode} &
I-type \\
\cline{1-9}
\\
\cline{1-9}
\multicolumn{2}{|c|}{imm[11:5]} &
\multicolumn{2}{c|}{rs2} &
\multicolumn{1}{c|}{rs1} &
\multicolumn{1}{c|}{funct3} &
\multicolumn{2}{c|}{imm[4:0]} &
\multicolumn{1}{c|}{opcode} &
S-type \\
\cline{1-9}
\\
\cline{1-9}
\multicolumn{1}{|c|}{imm[12]} &
\multicolumn{1}{c|}{imm[10:5]} &
\multicolumn{2}{c|}{rs2} &
\multicolumn{1}{c|}{rs1} &
\multicolumn{1}{c|}{funct3} &
\multicolumn{1}{c|}{imm[4:1]} &
\multicolumn{1}{c|}{imm[11]} &
\multicolumn{1}{c|}{opcode} &
B-type \\
\cline{1-9}
\\
\cline{1-9}
\multicolumn{6}{|c|}{imm[31:12]} &
\multicolumn{2}{c|}{rd} &
\multicolumn{1}{c|}{opcode} &
U-type \\
\cline{1-9}
\\
\cline{1-9}
\multicolumn{1}{|c|}{imm[20]} &
\multicolumn{2}{c|}{imm[10:1]} &
\multicolumn{1}{c|}{imm[11]} &
\multicolumn{2}{c|}{imm[19:12]} &
\multicolumn{2}{c|}{rd} &
\multicolumn{1}{c|}{opcode} &
J-type \\
\cline{1-9}
\end{tabular}
\end{center}
\end{small}
\caption{RISC-V base instruction formats showing immediate variants.}
\label{fig:baseinstformatsimm}
\end{figure}

Figure~\ref{fig:immtypes} shows the immediates produced by each of the
base instruction formats, and is labeled to show which instruction
bit (inst[{\em y}\,]) produces each bit of the immediate value.

\begin{figure}[h]
\begin{center}
\setlength{\tabcolsep}{4pt}
\begin{tabular}{p{0.2in}@{}p{1.2in}@{}p{1.0in}@{}p{0.2in}@{}p{0.7in}@{}p{0.7in}@{}p{0.2in}l}
\\
\multicolumn{1}{c}{\instbit{31}} &
\instbitrange{30}{20} &
\instbitrange{19}{12} &
\multicolumn{1}{c}{\instbit{11}} &
\instbitrange{10}{5} &
\instbitrange{4}{1} &
\multicolumn{1}{c}{\instbit{0}} &
\\
\cline{1-7}
\multicolumn{4}{|c|}{--- inst[31] ---} &
\multicolumn{1}{c|}{inst[30:25]} &
\multicolumn{1}{c|}{inst[24:21]} &
\multicolumn{1}{c|}{inst[20]} &
I-immediate \\
\cline{1-7}
\\
\cline{1-7}
\multicolumn{4}{|c|}{--- inst[31] ---} &
\multicolumn{1}{c|}{inst[30:25]} &
\multicolumn{1}{c|}{inst[11:8]} &
\multicolumn{1}{c|}{inst[7]} &
S-immediate \\
\cline{1-7}
\\
\cline{1-7}
\multicolumn{3}{|c|}{--- inst[31] ---} &
\multicolumn{1}{c|}{inst[7]} &
\multicolumn{1}{c|}{inst[30:25]} &
\multicolumn{1}{c|}{inst[11:8]} &
\multicolumn{1}{c|}{0} &
B-immediate \\
\cline{1-7}
\\
\cline{1-7}
\multicolumn{1}{|c|}{inst[31]} &
\multicolumn{1}{c|}{inst[30:20]} &
\multicolumn{1}{c|}{inst[19:12]} &
\multicolumn{4}{c|}{--- 0 ---} &
U-immediate \\
\cline{1-7}
\\
\cline{1-7}
\multicolumn{2}{|c|}{--- inst[31] ---} &
\multicolumn{1}{c|}{inst[19:12]} &
\multicolumn{1}{c|}{inst[20]} &
\multicolumn{1}{c|}{inst[30:25]} &
\multicolumn{1}{c|}{inst[24:21]} &
\multicolumn{1}{c|}{0} &
J-immediate \\
\cline{1-7}
\end{tabular}
\end{center}
\caption{Types of immediate produced by RISC-V instructions.  The fields are labeled with the
  instruction bits used to construct their value.  Sign extension
  always uses inst[31].}
\label{fig:immtypes}
\end{figure}

\begin{commentary}
Sign-extension is one of the most critical operations on immediates
(particularly in RV64I), and in RISC-V the sign bit for all immediates
is always held in bit 31 of the instruction to allow sign-extension to
proceed in parallel with instruction decoding.

Although more complex implementations might have separate adders for
branch and jump calculations and so would not benefit from keeping the
location of immediate bits constant across types of instruction, we
wanted to reduce the hardware cost of the simplest implementations.
By rotating bits in the instruction encoding of B and J immediates
instead of using dynamic hardware muxes to multiply the immediate by
2, we reduce instruction signal fanout and immediate mux costs by
around a factor of 2.  The scrambled immediate encoding will add
negligible time to static or ahead-of-time compilation.  For dynamic
generation of instructions, there is some small additional
overhead, but the most common short forward branches have
straightforward immediate encodings.
\end{commentary}

\section{Integer Computational Instructions}

Most integer computational instructions operate on XLEN bits of values
held in the integer register file.  Integer computational instructions
are either encoded as register-immediate operations using the I-type
format or as register-register operations using the R-type format.
The destination is register {\em rd} for both register-immediate and
register-register instructions.  No integer computational instructions
cause arithmetic exceptions.

\begin{commentary}
We did not include special instruction set support for overflow checks
on integer arithmetic operations in the base instruction set, as many
overflow checks can be cheaply implemented using RISC-V branches.
Overflow checking for unsigned addition requires only a single
additional branch instruction after the addition:
\verb! add t0, t1, t2; bltu t0, t1, overflow!.

For signed addition, if one operand's sign is known, overflow checking
requires only a single branch after the addition:
\verb! addi t0, t1, +imm; blt t0, t1, overflow!.  This covers the
common case of addition with an immediate operand.

For general signed addition, three additional instructions after the
addition are required, leveraging the observation that the sum should
be less than one of the operands if and only if the other operand is
negative.
\begin{verbatim}
         add t0, t1, t2
         slti t3, t2, 0
         slt t4, t0, t1
         bne t3, t4, overflow
\end{verbatim}
In RV64, checks of 32-bit signed additions can be optimized further by
comparing the results of ADD and ADDW on the operands.
\end{commentary}

\subsubsection*{Integer Register-Immediate Instructions}
\vspace{-0.4in}
\begin{center}
\begin{tabular}{M@{}R@{}S@{}R@{}O}
\\
\instbitrange{31}{20} &
\instbitrange{19}{15} &
\instbitrange{14}{12} &
\instbitrange{11}{7} &
\instbitrange{6}{0} \\
\hline
\multicolumn{1}{|c|}{imm[11:0]} &
\multicolumn{1}{c|}{rs1} &
\multicolumn{1}{c|}{funct3} &
\multicolumn{1}{c|}{rd} &
\multicolumn{1}{c|}{opcode} \\
\hline
12 & 5 & 3 & 5 & 7 \\
I-immediate[11:0] & src & ADDI/SLTI[U]  & dest & OP-IMM \\
I-immediate[11:0] & src & ANDI/ORI/XORI & dest & OP-IMM \\
\end{tabular}
\end{center}
ADDI adds the sign-extended 12-bit immediate to register {\em rs1}.
Arithmetic overflow is ignored and the result is simply the low
XLEN bits of the result.  ADDI {\em rd, rs1, 0} is used to implement the
MV {\em rd, rs1} assembler pseudo-instruction.

SLTI (set less than immediate) places the value 1 in register {\em rd}
if register {\em rs1} is less than the sign-extended immediate when
both are treated as signed numbers, else 0 is written to {\em rd}.
SLTIU is similar but compares the values as unsigned numbers (i.e.,
the immediate is first sign-extended to XLEN bits then treated as an
unsigned number).  Note, SLTIU {\em rd}, {\em rs1}, 1 sets {\em rd}
to 1 if {\em rs1} equals zero, otherwise sets {\em rd} to 0 (assembler
pseudo-op SEQZ {\em rd, rs}).

ANDI, ORI, XORI are logical operations that perform bitwise AND, OR,
and XOR on register {\em rs1} and the sign-extended 12-bit immediate
and place the result in {\em rd}.  Note, XORI {\em rd, rs1, -1}
performs a bitwise logical inversion of register {\em rs1} (assembler
pseudo-instruction NOT {\em rd, rs}).

\vspace{-0.2in}
\begin{center}
\begin{tabular}{S@{}R@{}R@{}S@{}R@{}O}
\\
\instbitrange{31}{25} &
\instbitrange{24}{20} &
\instbitrange{19}{15} &
\instbitrange{14}{12} &
\instbitrange{11}{7} &
\instbitrange{6}{0} \\
\hline
\multicolumn{1}{|c|}{imm[11:5]} &
\multicolumn{1}{c|}{imm[4:0]} &
\multicolumn{1}{c|}{rs1} &
\multicolumn{1}{c|}{funct3} &
\multicolumn{1}{c|}{rd} &
\multicolumn{1}{c|}{opcode} \\
\hline
7 & 5 & 5 & 3 & 5 & 7 \\
0000000 & shamt[4:0]  & src & SLLI & dest & OP-IMM \\
0000000 & shamt[4:0]  & src & SRLI & dest & OP-IMM \\
0100000 & shamt[4:0]  & src & SRAI & dest & OP-IMM \\
\end{tabular}
\end{center}

Shifts by a constant are encoded as a specialization of the
I-type format.  The operand to be shifted is in {\em rs1}, and the
shift amount is encoded in the lower 5 bits of the I-immediate field.
The right shift type is encoded in bit 30.
SLLI is a logical left shift (zeros are shifted into the lower bits);
SRLI is a logical right shift (zeros are shifted into the upper bits);
and SRAI is an arithmetic right shift (the original sign bit is copied
into the vacated upper bits).

\vspace{-0.2in}
\begin{center}
\begin{tabular}{U@{}R@{}O}
\\
\instbitrange{31}{12} &
\instbitrange{11}{7} &
\instbitrange{6}{0} \\
\hline
\multicolumn{1}{|c|}{imm[31:12]} &
\multicolumn{1}{c|}{rd} &
\multicolumn{1}{c|}{opcode} \\
\hline
20 & 5 & 7 \\
U-immediate[31:12] & dest & LUI \\
U-immediate[31:12] & dest & AUIPC
\end{tabular}
\end{center}

LUI (load upper immediate) is used to build 32-bit constants and uses
the U-type format.  LUI places the U-immediate value in the top 20
bits of the destination register {\em rd}, filling in the lowest 12
bits with zeros.

AUIPC (add upper immediate to {\tt pc}) is used to build {\tt pc}-relative
addresses and uses the U-type format.  AUIPC forms a 32-bit offset from the
20-bit U-immediate, filling in the lowest 12 bits with zeros, adds this offset
to the {\tt pc}, then places the result in register {\em rd}.

\begin{commentary}
The AUIPC instruction supports two-instruction sequences to access
arbitrary offsets from the PC for both control-flow transfers and data
accesses.  The combination of an AUIPC and the 12-bit immediate in a
JALR can transfer control to any 32-bit PC-relative address, while an
AUIPC plus the 12-bit immediate offset in regular load or store
instructions can access any 32-bit PC-relative data address.

The current PC can be obtained by setting the U-immediate to 0.  Although
a JAL +4 instruction could also be used to obtain the PC, it might cause
pipeline breaks in simpler microarchitectures or pollute the BTB structures in
more complex microarchitectures.
\end{commentary}

\subsubsection*{Integer Register-Register Operations}

RV32I defines several arithmetic R-type operations.  All operations
read the {\em rs1} and {\em rs2} registers as source operands and
write the result into register {\em rd}.  The {\em funct7} and {\em
  funct3} fields select the type of operation.

\vspace{-0.2in}
\begin{center}
\begin{tabular}{S@{}R@{}R@{}S@{}R@{}O}
\\
\instbitrange{31}{25} &
\instbitrange{24}{20} &
\instbitrange{19}{15} &
\instbitrange{14}{12} &
\instbitrange{11}{7} &
\instbitrange{6}{0} \\
\hline
\multicolumn{1}{|c|}{funct7} &
\multicolumn{1}{c|}{rs2} &
\multicolumn{1}{c|}{rs1} &
\multicolumn{1}{c|}{funct3} &
\multicolumn{1}{c|}{rd} &
\multicolumn{1}{c|}{opcode} \\
\hline
7 & 5 & 5 & 3 & 5 & 7 \\
0000000 & src2 & src1 & ADD/SLT/SLTU & dest & OP    \\
0000000 & src2 & src1 & AND/OR/XOR  & dest & OP    \\
0000000 & src2 & src1 & SLL/SRL     & dest & OP    \\
0100000 & src2 & src1 & SUB/SRA     & dest & OP    \\
\end{tabular}
\end{center}

ADD and SUB perform addition and subtraction respectively.  Overflows
are ignored and the low XLEN bits of results are written to the
destination.  SLT and SLTU perform signed and unsigned compares
respectively, writing 1 to {\em rd} if $\mbox{\em rs1} < \mbox{\em
  rs2}$, 0 otherwise.  Note, SLTU {\em rd}, {\em x0}, {\em rs2} sets
{\em rd} to 1 if {\em rs2} is not equal to zero, otherwise sets {\em
  rd} to zero (assembler pseudo-op SNEZ {\em rd, rs}).  AND, OR, and
XOR perform bitwise logical operations.

SLL, SRL, and SRA perform logical left, logical right, and arithmetic
right shifts on the value in register {\em rs1} by the shift amount
held in the lower 5 bits of register {\em rs2}.

\subsubsection*{NOP Instruction}
\vspace{-0.4in}
\begin{center}
\begin{tabular}{M@{}R@{}S@{}R@{}O}
\\
\instbitrange{31}{20} &
\instbitrange{19}{15} &
\instbitrange{14}{12} &
\instbitrange{11}{7} &
\instbitrange{6}{0} \\
\hline
\multicolumn{1}{|c|}{imm[11:0]} &
\multicolumn{1}{c|}{rs1} &
\multicolumn{1}{c|}{funct3} &
\multicolumn{1}{c|}{rd} &
\multicolumn{1}{c|}{opcode} \\
\hline
12 & 5 & 3 & 5 & 7 \\
0 & 0 & ADDI & 0 & OP-IMM \\
\end{tabular}
\end{center}

The NOP instruction does not change any user-visible state, except
for advancing the {\tt pc}.  NOP is encoded as ADDI {\em x0, x0, 0}.

\begin{commentary}
NOPs can be used to align code segments to microarchitecturally
significant address boundaries, or to leave space for inline code
modifications.  Although there are many possible ways to encode a NOP,
we define a canonical NOP encoding to allow microarchitectural
optimizations as well as for more readable disassembly output.
\end{commentary}

\section{Control Transfer Instructions}

RV32I provides two types of control transfer instructions:
unconditional jumps and conditional branches.  Control transfer
instructions in RV32I do {\em not} have architecturally visible delay
slots.

\subsubsection*{Unconditional Jumps}

\vspace{-0.1in} The jump and link (JAL) instruction uses the J-type
format, where the J-immediate encodes a signed offset in multiples of
2 bytes.  The offset is sign-extended and added to the {\tt pc}
to form the jump target address.  Jumps can therefore target a
$\pm$\wunits{1}{MiB} range. JAL stores the address of the instruction
following the jump ({\tt pc}+4) into register {\em rd}.  The standard
software calling convention uses {\tt x1} as the return address
register and {\tt x5} as an alternate link register.

\begin{commentary}
The alternate link register supports calling millicode routines (e.g.,
those to save and restore registers in compressed code) while
preserving the regular return address register.  The register {\tt x5}
was chosen as the alternate link register as it maps to a temporary in
the standard calling convention, and has an encoding that is only one
bit different than the regular link register.
\end{commentary}

Plain unconditional jumps (assembler pseudo-op J) are encoded as a JAL
with {\em rd}={\tt x0}.

\vspace{-0.2in}
\begin{center}
\begin{tabular}{W@{}E@{}W@{}R@{}R@{}O}
\\
\multicolumn{1}{c}{\instbit{31}} &
\instbitrange{30}{21} &
\multicolumn{1}{c}{\instbit{20}} &
\instbitrange{19}{12} &
\instbitrange{11}{7} &
\instbitrange{6}{0} \\
\hline
\multicolumn{1}{|c|}{imm[20]} &
\multicolumn{1}{c|}{imm[10:1]} &
\multicolumn{1}{c|}{imm[11]} &
\multicolumn{1}{c|}{imm[19:12]} &
\multicolumn{1}{c|}{rd} &
\multicolumn{1}{c|}{opcode} \\
\hline
1 & 10 & \multicolumn{1}{c}{1} & 8 & 5 & 7 \\
\multicolumn{4}{c}{offset[20:1]} & dest & JAL \\
\end{tabular}
\end{center}

The indirect jump instruction JALR (jump and link register) uses the
I-type encoding.  The target address is obtained by adding the 12-bit
signed I-immediate to the register {\em rs1}, then setting the
least-significant bit of the result to zero.  The address of
the instruction following the jump ({\tt pc}+4) is written to register
{\em rd}.  Register {\tt x0} can be used as the destination if the
result is not required.
\vspace{-0.4in}
\begin{center}
\begin{tabular}{M@{}R@{}F@{}R@{}O}
\\
\instbitrange{31}{20} &
\instbitrange{19}{15} &
\instbitrange{14}{12} &
\instbitrange{11}{7} &
\instbitrange{6}{0} \\
\hline
\multicolumn{1}{|c|}{imm[11:0]} &
\multicolumn{1}{c|}{rs1} &
\multicolumn{1}{c|}{funct3} &
\multicolumn{1}{c|}{rd} &
\multicolumn{1}{c|}{opcode} \\
\hline
12 & 5 & 3 & 5 & 7 \\
offset[11:0] & base & 0 & dest & JALR \\
\end{tabular}
\end{center}

\begin{commentary}
The unconditional jump instructions all use PC-relative addressing to
help support position-independent code.  The JALR instruction was
defined to enable a two-instruction sequence to jump anywhere in a
32-bit absolute address range.  A LUI instruction can first load {\em
  rs1} with the upper 20 bits of a target address, then JALR can add
in the lower bits. Similarly, AUIPC then JALR can jump
anywhere in a 32-bit {\tt pc}-relative address range.

Note that the JALR instruction does not treat the 12-bit immediate as
multiples of 2 bytes, unlike the conditional branch instructions.
This avoids one more immediate format in hardware.  In
practice, most uses of JALR will have either a zero immediate or be
paired with a LUI or AUIPC, so the slight reduction in range is not
significant.

Clearing the least-significant bit when calculating the JALR target
address both simplifies the hardware slightly and allows the
low bit of function pointers to be used to store auxiliary
information.  Although there is potentially a slight loss of error
checking in this case, in practice jumps to an incorrect instruction
address will usually quickly raise an exception.

When used with a base {\em rs1}$=${\tt x0}, JALR can be used to implement
a single instruction subroutine call to the lowest \wunits{2}{KiB} or highest
\wunits{2}{KiB} address region from anywhere in the address space, which could
be used to implement fast calls to a small runtime library.
\end{commentary}

The JAL and JALR instructions will generate a misaligned instruction
fetch exception if the target address is not aligned to a four-byte
boundary.

\begin{commentary}
Instruction fetch misaligned exceptions are not possible on machines
that support extensions with 16-bit aligned instructions, such as the
compressed instruction set extension, C.
\end{commentary}

Return-address prediction stacks are a common feature of
high-performance instruction-fetch units, but require accurate
detection of instructions used for procedure calls and returns to be
effective.  For RISC-V, hints as to the instructions' usage are encoded
implicitly via the register numbers used.  A JAL instruction should
push the return address onto a return-address stack (RAS) only when
{\em rd}$=${\tt x1}/{\tt x5}.  JALR instructions should push/pop a
RAS as shown in the Table~\ref{rashints}.
\begin{table}[hbt]
\centering
\begin{tabular}{|c|c|c|l|}
  \hline
  \em rd & \em rs1 & {\em rs1}$=${\em rd} & RAS action \\
  \hline
  !{\em link} & !{\em link} & - & none \\
  !{\em link} &  {\em link} & - & pop \\
   {\em link} & !{\em link} & - & push  \\
   {\em link} &  {\em link} & 0 & pop, then push \\
   {\em link} &  {\em link} & 1 & push \\
   \hline
\end{tabular}
\caption{Return-address stack prediction hints encoded in register
  specifiers used in the instruction.  In the above, {\em link} is
  true when the register is either {\tt x1} or {\tt x5}.}
\label{rashints}
\end{table}

\begin{commentary}
Some other ISAs added explicit hint bits to their indirect-jump instructions
to guide return-address stack manipulation.  We use implicit hinting tied to
register numbers and the calling convention to reduce the encoding space used
for these hints.

When two different link registers ({\tt x1} and {\tt x5}) are given as
{\em rs1} and {\em rd}, then the RAS is both popped and pushed to
support coroutines.  If {\em rs1} and {\em rd} are the same link
register (either {\tt x1} or {\tt x5}), the RAS is only pushed to
enable macro-op fusion of the sequences:\linebreak
{\tt lui ra, imm20; jalr ra, ra, imm12} \ and \ 
{\tt auipc ra, imm20; jalr ra, ra, imm12}
\end{commentary}

\subsubsection*{Conditional Branches}

All branch instructions use the B-type instruction format.  The
12-bit B-immediate encodes signed offsets in multiples of 2, and is
added to the current {\tt pc} to give the target address.  The
conditional branch range is $\pm$\wunits{4}{KiB}.

\vspace{-0.2in}
\begin{center}
\begin{tabular}{W@{}R@{}F@{}F@{}R@{}R@{}F@{}S}
\\
\multicolumn{1}{c}{\instbit{31}} &
\instbitrange{30}{25} &
\instbitrange{24}{20} &
\instbitrange{19}{15} &
\instbitrange{14}{12} &
\instbitrange{11}{8} &
\multicolumn{1}{c}{\instbit{7}} &
\instbitrange{6}{0} \\
\hline
\multicolumn{1}{|c|}{imm[12]} &
\multicolumn{1}{c|}{imm[10:5]} &
\multicolumn{1}{c|}{rs2} &
\multicolumn{1}{c|}{rs1} &
\multicolumn{1}{c|}{funct3} &
\multicolumn{1}{c|}{imm[4:1]} &
\multicolumn{1}{c|}{imm[11]} &
\multicolumn{1}{c|}{opcode} \\
\hline
1 & 6 & 5 & 5 & 3 & 4 & 1 & 7 \\
\multicolumn{2}{c}{offset[12,10:5]} & src2 & src1 & BEQ/BNE & \multicolumn{2}{c}{offset[11,4:1]} & BRANCH \\
\multicolumn{2}{c}{offset[12,10:5]} & src2 & src1 & BLT[U] & \multicolumn{2}{c}{offset[11,4:1]} & BRANCH \\
\multicolumn{2}{c}{offset[12,10:5]} & src2 & src1 & BGE[U]  & \multicolumn{2}{c}{offset[11,4:1]} & BRANCH \\
\end{tabular}
\end{center}

Branch instructions compare two registers.  BEQ and BNE take the
branch if registers {\em rs1} and {\em rs2} are equal or unequal
respectively.  BLT and BLTU take the branch if {\em rs1} is less than
{\em rs2}, using signed and unsigned comparison respectively.  BGE and
BGEU take the branch if {\em rs1} is greater than or equal to {\em rs2},
using signed and unsigned comparison respectively. Note, BGT, BGTU,
BLE, and BLEU can be synthesized by reversing the operands to BLT,
BLTU, BGE, and BGEU, respectively.

\begin{commentary}
Signed array bounds may be checked with a single BLTU instruction, since
any negative index will compare greater than any nonnegative bound.
\end{commentary}

Software should be optimized such that the sequential code path is the
most common path, with less-frequently taken code paths placed out of
line.  Software should also assume that backward branches will be
predicted taken and forward branches as not taken, at least the
first time they are encountered.  Dynamic predictors should quickly
learn any predictable branch behavior.

Unlike some other architectures, the RISC-V jump (JAL with {\em
  rd}={\tt x0}) instruction should always be used for unconditional
branches instead of a conditional branch instruction with an always-true
condition.  RISC-V jumps are also PC-relative and support a much
wider offset range than branches, and will not pressure conditional
branch prediction tables.

\begin{commentary}
The conditional branches were designed to include arithmetic
comparison operations between two registers (as also done in PA-RISC
and Xtensa ISA), rather than use condition codes (x86, ARM, SPARC,
PowerPC), or to only compare one register against zero (Alpha, MIPS),
or two registers only for equality (MIPS).  This design was motivated
by the observation that a combined compare-and-branch instruction fits
into a regular pipeline, avoids additional condition code state or use
of a temporary register, and reduces static code size and dynamic
instruction fetch traffic.  Another point is that comparisons against
zero require non-trivial circuit delay (especially after the move to
static logic in advanced processes) and so are almost as expensive as
arithmetic magnitude compares.  Another advantage of a fused
compare-and-branch instruction is that branches are observed earlier
in the front-end instruction stream, and so can be predicted earlier.
There is perhaps an advantage to a design with condition codes in the
case where multiple branches can be taken based on the same condition
codes, but we believe this case to be relatively rare.

We considered but did not include static branch hints in the
instruction encoding.  These can reduce the pressure on dynamic
predictors, but require more instruction encoding space and
software profiling for best results, and can result in poor
performance if production runs do not match profiling runs.

We considered but did not include conditional moves or predicated
instructions, which can effectively replace unpredictable short
forward branches.  Conditional moves are the simpler of the two, but
are difficult to use with conditional code that might cause exceptions
(memory accesses and floating-point operations).  Predication adds
additional flag state to a system, additional instructions to set and
clear flags, and additional encoding overhead on every instruction.
Both conditional move and predicated instructions add complexity to
out-of-order microarchitectures, adding an implicit third source
operand due to the need to copy the original value of the destination
architectural register into the renamed destination physical register
if the predicate is false.  Also, static compile-time decisions to use
predication instead of branches can result in lower performance on
inputs not included in the compiler training set, especially given
that unpredictable branches are rare, and becoming rarer as branch
prediction techniques improve.

We note that various microarchitectural techniques exist to
dynamically convert unpredictable short forward branches into
internally predicated code to avoid the cost of flushing pipelines on
a branch mispredict~\cite{heil-tr1996,Klauser-1998,Kim-micro2005} and
have been implemented in commercial processors~\cite{ibmpower7}.
The simplest techniques just reduce the penalty of recovering from a
mispredicted short forward branch by only flushing instructions in the
branch shadow instead of the entire fetch pipeline, or by fetching
instructions from both sides using wide instruction fetch or idle
instruction fetch slots.  More complex techniques for out-of-order
cores add internal predicates on instructions in the branch shadow,
with the internal predicate value written by the branch instruction,
allowing the branch and following instructions to be executed
speculatively and out-of-order with respect to other code~\cite{ibmpower7}.
\end{commentary}

\section{Load and Store Instructions}

RV32I is a load-store architecture, where only load and store
instructions access memory and arithmetic instructions only operate on
CPU registers.  RV32I provides a 32-bit user address space that is
byte-addressed and little-endian.  The execution environment will
define what portions of the address space are legal to access.  Loads
with a destination of {\tt x0} must still raise any exceptions and
action any other side effects even though the load value is discarded.

\vspace{-0.4in}
\begin{center}
\begin{tabular}{M@{}R@{}F@{}R@{}O}
\\
\instbitrange{31}{20} &
\instbitrange{19}{15} &
\instbitrange{14}{12} &
\instbitrange{11}{7} &
\instbitrange{6}{0} \\
\hline
\multicolumn{1}{|c|}{imm[11:0]} &
\multicolumn{1}{c|}{rs1} &
\multicolumn{1}{c|}{funct3} &
\multicolumn{1}{c|}{rd} &
\multicolumn{1}{c|}{opcode} \\
\hline
12 & 5 & 3 & 5 & 7 \\
offset[11:0] & base & width & dest & LOAD \\
\end{tabular}
\end{center}

\vspace{-0.2in}
\begin{center}
\begin{tabular}{O@{}R@{}R@{}F@{}R@{}O}
\\
\instbitrange{31}{25} &
\instbitrange{24}{20} &
\instbitrange{19}{15} &
\instbitrange{14}{12} &
\instbitrange{11}{7} &
\instbitrange{6}{0} \\
\hline
\multicolumn{1}{|c|}{imm[11:5]} &
\multicolumn{1}{c|}{rs2} &
\multicolumn{1}{c|}{rs1} &
\multicolumn{1}{c|}{funct3} &
\multicolumn{1}{c|}{imm[4:0]} &
\multicolumn{1}{c|}{opcode} \\
\hline
7 & 5 & 5 & 3 & 5 & 7 \\
offset[11:5] & src & base & width & offset[4:0] & STORE \\
\end{tabular}
\end{center}

Load and store instructions transfer a value between the registers and
memory.  Loads are encoded in the I-type format and stores are
S-type.  The effective byte address is obtained by adding register
{\em rs1} to the sign-extended 12-bit offset.  Loads copy a value
from memory to register {\em rd}.  Stores copy the value in register
{\em rs2} to memory.

The LW instruction loads a 32-bit value from memory into {\em rd}.  LH
loads a 16-bit value from memory, then sign-extends to 32-bits before
storing in {\tt rd}. LHU loads a 16-bit value from memory but then
zero extends to 32-bits before storing in {\em rd}.  LB and LBU are
defined analogously for 8-bit values.  The SW, SH, and SB instructions
store 32-bit, 16-bit, and 8-bit values from the low bits of register
{\em rs2} to memory.

For best performance, the effective address for all loads and stores
should be naturally aligned for each data type (i.e., on a four-byte
boundary for 32-bit accesses, and a two-byte boundary for 16-bit
accesses).  The base ISA supports misaligned accesses, but these might
run extremely slowly depending on the implementation.  Furthermore,
naturally aligned loads and stores are guaranteed to execute
atomically, whereas misaligned loads and stores might not, and hence
require additional synchronization to ensure atomicity.

\begin{commentary}
Misaligned accesses are occasionally required when porting legacy
code, and are essential for good performance on many applications when
using any form of packed-SIMD extension.  Our rationale for supporting
misaligned accesses via the regular load and store instructions is to
simplify the addition of misaligned hardware support.  One option
would have been to disallow misaligned accesses in the base ISA and
then provide some separate ISA support for misaligned accesses, either
special instructions to help software handle misaligned accesses or a
new hardware addressing mode for misaligned accesses.  Special
instructions are difficult to use, complicate the ISA, and often add
new processor state (e.g., SPARC VIS align address offset register) or
complicate access to existing processor state (e.g., MIPS LWL/LWR
partial register writes).  In addition, for loop-oriented packed-SIMD
code, the extra overhead when operands are misaligned motivates
software to provide multiple forms of loop depending on operand
alignment, which complicates code generation and adds to loop startup
overhead.  New misaligned hardware addressing modes take considerable
space in the instruction encoding or require very simplified
addressing modes (e.g., register indirect only).

We do not mandate atomicity for misaligned accesses so simple
implementations can just use a machine trap and software handler to
handle some or all misaligned accesses.  If hardware misaligned support is
provided, software can exploit this by simply using regular load and
store instructions.  Hardware can then automatically optimize accesses
depending on whether runtime addresses are aligned.
\end{commentary}

\section{Control and Status Register Instructions}
\label{sec:csrinsts}

SYSTEM instructions are used to access system functionality that might
require privileged access and are encoded using the I-type instruction
format.  These can be divided into two main classes: those that
atomically read-modify-write control and status registers (CSRs), and
all other potentially privileged instructions. CSR instructions are
described in this section, with the two other user-level SYSTEM
instructions described in the following section.

\begin{commentary}
The SYSTEM instructions are defined to allow simpler implementations
to always trap to a single software trap handler.  More sophisticated
implementations might execute more of each system instruction in
hardware.
\end{commentary}

\subsubsection*{CSR Instructions}

We define the full set of CSR instructions here, although in the standard
user-level base ISA, only a handful of read-only counter CSRs are accessible.

\vspace{-0.2in}
\begin{center}
\begin{tabular}{M@{}R@{}F@{}R@{}S}
\\
\instbitrange{31}{20} &
\instbitrange{19}{15} &
\instbitrange{14}{12} &
\instbitrange{11}{7} &
\instbitrange{6}{0} \\
\hline
\multicolumn{1}{|c|}{csr} &
\multicolumn{1}{c|}{rs1} &
\multicolumn{1}{c|}{funct3} &
\multicolumn{1}{c|}{rd} &
\multicolumn{1}{c|}{opcode} \\
\hline
12 & 5 & 3 & 5 & 7 \\
source/dest  & source & CSRRW  & dest & SYSTEM \\
source/dest  & source & CSRRS  & dest & SYSTEM \\
source/dest  & source & CSRRC  & dest & SYSTEM \\
source/dest  & uimm[4:0]   & CSRRWI & dest & SYSTEM \\
source/dest  & uimm[4:0]   & CSRRSI & dest & SYSTEM \\
source/dest  & uimm[4:0]   & CSRRCI & dest & SYSTEM \\
\end{tabular}
\end{center}

The CSRRW (Atomic Read/Write CSR) instruction atomically swaps values
in the CSRs and integer registers. CSRRW reads the old value of the
CSR, zero-extends the value to XLEN bits, then writes it to integer
register {\em rd}.  The initial value in {\em rs1} is written to the
CSR.  If {\em rd}={\tt x0}, then the instruction shall not read the CSR
and shall not cause any of the side-effects that might occur on a CSR
read.

The CSRRS (Atomic Read and Set Bits in CSR) instruction reads the
value of the CSR, zero-extends the value to XLEN bits, and writes it
to integer register {\em rd}.  The initial value in integer register
{\em rs1} is treated as a bit mask that specifies bit positions to be
set in the CSR.  Any bit that is high in {\em rs1} will cause the
corresponding bit to be set in the CSR, if that CSR bit is writable.
Other bits in the CSR are unaffected (though CSRs might have side
effects when written).

The CSRRC (Atomic Read and Clear Bits in CSR) instruction reads the
value of the CSR, zero-extends the value to XLEN bits, and writes it
to integer register {\em rd}.  The initial value in integer register
{\em rs1} is treated as a bit mask that specifies bit positions to be
cleared in the CSR.  Any bit that is high in {\em rs1} will cause the
corresponding bit to be cleared in the CSR, if that CSR bit is
writable.  Other bits in the CSR are unaffected.

For both CSRRS and CSRRC, if {\em rs1}={\tt x0}, then the instruction
will not write to the CSR at all, and so shall not cause any of the
side effects that might otherwise occur on a CSR write, such as
raising illegal instruction exceptions on accesses to read-only CSRs.
Note that if {\em rs1} specifies a register holding a zero value other
than {\tt x0}, the instruction will still attempt to write the
unmodified value back to the CSR and will cause any attendant side effects.

The CSRRWI, CSRRSI, and CSRRCI variants are similar to CSRRW, CSRRS,
and CSRRC respectively, except they update the CSR using an XLEN-bit
value obtained by zero-extending a 5-bit unsigned immediate (uimm[4:0]) field
encoded in the {\em rs1} field instead of a value from an integer
register.  For CSRRSI and CSRRCI, if the uimm[4:0] field is zero, then
these instructions will not write to the CSR, and shall not cause any
of the side effects that might otherwise occur on a CSR write.  For
CSRRWI, if {\em rd}={\tt x0}, then the instruction shall not read the
CSR and shall not cause any of the side-effects that might occur on a
CSR read.

Some CSRs, such as the instructions retired counter, {\tt instret}, may be
modified as side effects of instruction execution.  In these cases, if a CSR
access instruction reads a CSR, it reads the value prior to the execution of
the instruction.  If a CSR access instruction writes a CSR, the update occurs
after the execution of the instruction.  In particular, a value written to
{\tt instret} by one instruction will be the value read by the following
instruction (i.e., the increment of {\tt instret} caused by the first
instruction retiring happens before the write of the new value).

The assembler pseudo-instruction to read a CSR, CSRR {\em rd, csr}, is
encoded as CSRRS {\em rd, csr, x0}.  The assembler pseudo-instruction
to write a CSR, CSRW {\em csr, rs1}, is encoded as CSRRW {\em x0, csr,
  rs1}, while CSRWI {\em csr, uimm}, is encoded as CSRRWI {\em x0,
  csr, uimm}.

Further assembler pseudo-instructions are defined to set and clear
bits in the CSR when the old value is not required: CSRS/CSRC {\em
  csr, rs1}; CSRSI/CSRCI {\em csr, uimm}.

\subsubsection*{Timers and Counters}

\vspace{-0.2in}
\begin{center}
\begin{tabular}{M@{}R@{}F@{}R@{}S}
\\
\instbitrange{31}{20} &
\instbitrange{19}{15} &
\instbitrange{14}{12} &
\instbitrange{11}{7} &
\instbitrange{6}{0} \\
\hline
\multicolumn{1}{|c|}{csr} &
\multicolumn{1}{c|}{rs1} &
\multicolumn{1}{c|}{funct3} &
\multicolumn{1}{c|}{rd} &
\multicolumn{1}{c|}{opcode} \\
\hline
12 & 5 & 3 & 5 & 7 \\
RDCYCLE[H]   & 0 & CSRRS  & dest & SYSTEM \\
RDTIME[H]    & 0 & CSRRS  & dest & SYSTEM \\
RDINSTRET[H] & 0 & CSRRS  & dest & SYSTEM \\
\end{tabular}
\end{center}

RV32I provides a number of 64-bit read-only user-level counters, which
are mapped into the 12-bit CSR address space and accessed in 32-bit
pieces using CSRRS instructions.

The RDCYCLE pseudo-instruction reads the low XLEN bits of the {\tt
  cycle} CSR which holds a count of the number of clock cycles
executed by the processor core on which the hart is running from
an arbitrary start time in the past.  RDCYCLEH is
an RV32I-only instruction that reads bits 63--32 of the same cycle
counter.  The underlying 64-bit counter should never overflow in
practice.  The rate at which the cycle counter advances will depend on
the implementation and operating environment.  The execution
environment should provide a means to determine the current rate
(cycles/second) at which the cycle counter is incrementing.

The RDTIME pseudo-instruction reads the low XLEN bits of the {\tt
  time} CSR, which counts wall-clock real time that has passed from an
arbitrary start time in the past.  RDTIMEH is an RV32I-only instruction
that reads bits 63--32 of the same real-time counter.  The underlying 64-bit
counter should never overflow in practice.  The execution environment
should provide a means of determining the period of the real-time
counter (seconds/tick).  The period must be constant.  The
real-time clocks of all harts in a single user application
should be synchronized to within one tick of the real-time clock.  The
environment should provide a means to determine the accuracy of the
clock.

The RDINSTRET pseudo-instruction reads the low XLEN bits of the {\tt
  instret} CSR, which counts the number of instructions retired by
this hart from some arbitrary start point in the past.  RDINSTRETH is
an RV32I-only instruction that reads bits 63--32 of the same
instruction counter. The underlying 64-bit counter that should never
overflow in practice.

The following code sequence will read a valid 64-bit cycle counter value into
{\tt x3}:{\tt x2}, even if the counter overflows between reading its upper
and lower halves.

\begin{figure}[h!]
\begin{center}
\begin{verbatim}
    again:
        rdcycleh     x3
        rdcycle      x2
        rdcycleh     x4
        bne          x3, x4, again
\end{verbatim}
\end{center}
\caption{Sample code for reading the 64-bit cycle counter in RV32.}
\label{critical}
\end{figure}

\begin{commentary}
We mandate these basic counters be provided in all implementations as
they are essential for basic performance analysis, adaptive and
dynamic optimization, and to allow an application to work with
real-time streams.  Additional counters should be provided to help
diagnose performance problems and these should be made accessible from
user-level application code with low overhead.

We required the counters be 64 bits wide, even on RV32, as otherwise
it is very difficult for software to determine if values have
overflowed.  For a low-end implementation, the upper 32 bits of each
counter can be implemented using software counters incremented by a
trap handler triggered by overflow of the lower 32 bits.  The sample
code described above shows how the full 64-bit width value can be
safely read using the individual 32-bit instructions.

In some applications, it is important to be able to read multiple
counters at the same instant in time.  When run under a multitasking
environment, a user thread can suffer a context switch while
attempting to read the counters.  One solution is for the user thread
to read the real-time counter before and after reading the other
counters to determine if a context switch occurred in the middle of the
sequence, in which case the reads can be retried.  We considered
adding output latches to allow a user thread to snapshot the counter
values atomically, but this would increase the size of the user
context, especially for implementations with a richer set of counters.
\end{commentary}


\section{Environment Call and Breakpoints}

\vspace{-0.2in}
\begin{center}
\begin{tabular}{M@{}R@{}F@{}R@{}S}
\\
\instbitrange{31}{20} &
\instbitrange{19}{15} &
\instbitrange{14}{12} &
\instbitrange{11}{7} &
\instbitrange{6}{0} \\
\hline
\multicolumn{1}{|c|}{funct12} &
\multicolumn{1}{c|}{rs1} &
\multicolumn{1}{c|}{funct3} &
\multicolumn{1}{c|}{rd} &
\multicolumn{1}{c|}{opcode} \\
\hline
12 & 5 & 3 & 5 & 7 \\
ECALL   & 0 & PRIV & 0 & SYSTEM \\
EBREAK  & 0 & PRIV & 0 & SYSTEM \\
\end{tabular}
\end{center}

The ECALL instruction is used to make a request to the supporting
execution environment, which is usually an operating system.  The ABI
for the system will define how parameters for the environment request
are passed, but usually these will be in defined locations in the
integer register file.

The EBREAK instruction is used by debuggers to cause control to be
transferred back to a debugging environment.

\begin{commentary}
ECALL and EBREAK were previously named SCALL and SBREAK.  The
instructions have the same functionality and encoding, but were
renamed to reflect that they can be used more generally than to call a
supervisor-level operating system or debugger.
\end{commentary}

\section{Memory Consistency Model}

This section defines the RISC-V memory consistency model.
A memory consistency model is a set of rules that specifies which values can be legally returned by loads of memory.
The RISC-V ISA by default uses a memory model called ``RVWMO'' (RISC-V Weak Memory Ordering).
RVWMO is designed to provide flexibility for architects to build high-performance scalable designs, while simultaneously supporting a tractable programming model.

The RISC-V ISA also provides the optional Ztso extension which imposes the stronger RVTSO (RISC-V Total Store Ordering) memory model for hardware that chooses to provide it.
RVTSO provides a simpler programming model but is more restrictive on the implementations that can be legally built.
%Other models (e.g., IBM Power) provide a model which allows for more aggressive and flexible implementations, but at the cost of a much more complicated programming model.
%RVWMO, a variant of release consistency, balances these two extremes.
%The RVWMO memory model also addresses known shortcomings of memory models of the past (e.g., by enforcing memory ordering on address dependencies and on most same-address load-load pairs).

Under both models, code running on a single hart will appear to execute in order from the perspective of other memory operations in the same hart, but memory instructions from another hart may observe the memory instructions from the first hart being executed in a different order.
Therefore, under both models, multithreaded code may require explicit synchronization to guarantee ordering between memory operations from different harts.
The base RISC-V ISA provides a {\tt fence} instruction for this purpose, described in Section~\ref{sec:fence}, while the optional ``A'' atomics extension defines load-reserved/store-conditional and atomic read-modify-write operations.

Appendix~\ref{sec:explanation} provides a formal model and additional explanatory material about the memory model.

%This chapter provides a precise formal specification of the RVWMO memory consistency model.
%Section~\ref{sec:definition} provides a natural language specification of the model.
%Section~\ref{sec:explanation} provides explanation and intuition for the design of RVWMO,
%The sections that follow provide other explanatory and reference material, and then Sections~\ref{sec:alloy}--\ref{sec:operational} provide a suite of formalisms that enable rigorous analysis.
%Finally, Chapter~\ref{sec:tso} defines RVTSO and the Ztso extension.

\begin{commentary}
  The RVWMO and RVTSO models are formalized as presented, but the interaction of the memory model with I/O, instruction fetches, page table walks, and {\tt sfence.vma} is not formalized.  We may formalize some or all of them in a future revision.  The ``V'' vector, transactional memory, and ``J'' JIT extensions will need to be incorporated into a future revision as well.

  Memory models which permit memory accesses of different sizes are an active and open area of research, and the specifics of how the RISC-V memory model deals with mixed-size memory accesses is subject to change.  Nevertheless, we provide an educated guess at sane rules for such situations as a guideline for architects and implementers.
\end{commentary}

\subsection{Definition of the RVWMO Memory Model}
\label{sec:definition}

The RVWMO memory model is defined in terms of the {\em global memory order}, a total ordering of the memory accesses produced by all harts.
In general, a multithreaded program will have many different possible executions, and each execution will have its its own corresponding global memory order.

The global memory order is defined in terms of the primitive load(s) and/or store(s) generated by each memory instruction.
It is then subject to the constraints defined in the rest of this chapter.
Any execution which satisfies all of the memory model constraints is a legal execution (as far as the memory model is concerned).

\subsubsection*{Memory Model Primitives}
The RVWMO memory model is specified in terms of the memory accesses generated by each instruction.
The {\em program order} over memory accesses reflects the order in which the instructions that generate each load and store are originally laid out in that hart's dynamic instruction stream; i.e., the order in which a simple in-order processor would execute the instructions of that hart.
Table~\ref{tab:memoryaccesses} summarizes the memory accesses generated by each type of memory instruction.

\begin{table}[h!]
  \centering
  \begin{tabular}{|l|l|}
    \hline
    RISC-V Instruction & Memory Accesses \\
    \hline
    \hline
    \tt l\{b|h|w|d\}    & load$^*$               \\
    \tt s\{b|h|w|d\}    & store$^*$              \\
    \hline
    \hline
    \tt lr              & load                   \\
    \tt lr.aq           & load-acquire-RCpc      \\
    \tt lr.aqrl         & load-acquire-RCsc      \\
    \tt lr.rl           & (deprecated)           \\
    \hline
    \hline
    \tt sc$^\dagger$      & store                  \\
    \tt sc.rl$^\dagger$   & store-release-RCpc     \\
    \tt sc.aqrl$^\dagger$ & store-release-RCsc     \\
    \tt sc.aq$^\dagger$   & (deprecated)           \\
    \hline
    \hline
    \tt amo<op>         & load; {\em $<$op$>$}; store \\
    \tt amo<op>.aq      & load-acquire-RCpc; {\em $<$op$>$}; store \\
    \tt amo<op>.rl      & load; {\em $<$op$>$}; store-release-RCpc \\
    \tt amo<op>.aqrl    & load-SC; {\em $<$op$>$}; store-SC \\
    \hline
    \multicolumn{2}{l}{$^*$: possibly multiple if misaligned} \\
    \multicolumn{2}{l}{$^\dagger$: if successful} \\
  \end{tabular}
  \caption{Mapping of instructions into memory accesses, for the purposes of describing the memory model.  ``{\tt.sc}'' is a synonym for ``{\tt.aqrl}''.}
  \label{tab:memoryaccesses}
\end{table}

Every aligned load or store instruction gives rise to exactly one memory access that executes in a single-copy atomic fashion: it can never be observed in a partially-incomplete state.
Every misaligned load or store may be decomposed into a set of component loads or stores at any granularity.
The memory accesses generated by misaligned loads and stores are not ordered with respect to each other in program order, but they are ordered with respect to the memory accesses generated by preceding and subsequent instructions in program order.

\begin{commentary}
  The legal decomposition of unaligned memory operations down to even byte granularity facilitates emulation on implementations that do not natively support unaligned accesses.
  Such implementations might, for example, simply iterate over the bytes of a misaligned access one by one.
\end{commentary}

AMOs give rise to exactly two memory accesses: one load and one store.  These accesses are said to be {\em paired}.
Every {\tt lr} instruction gives rise to exactly one load.
Every {\tt sc} instruction gives rise to either zero stores or one store, depending on whether the store conditional succeeds.
A successful {\tt sc} instruction is said to be paired with the last {\tt lr} instruction that precedes it in program order; the corresponding memory operations are said to be paired as well.
Both {\tt lr} and {\tt sc} instructions may be unpaired if they do not meet the condition above.
The complete list of conditions determining the success or failure of store conditional instructions is defined in the ``A'' extension.

Loads and stores generated by atomics may carry ordering annotations.
Loads may carry ``acquire-RCpc'' or ``acquire-RCsc'' annotations.
The term ``load-acquire'' without further annotation refers to both collectively.
Stores may carry ``release-RCpc'' or ``release-RCsc'' annotations, and once again ``store-release'' without further annotation refers to both together.
In the memory model literature, the term ``RCpc'' stands for release consistency with processor-consistent synchronization operations, and the term ``RCsc'' stands for release consistency with sequentially-consistent synchronization operations.
Finally, AMOs with both {\tt .aq} and {\tt .rl} set are sequentially-consistent in an even stronger sense: they do not allow any reordering in either direction.
The precise semantics of these annotations as they apply to RVWMO are described by the memory model rules below.

\begin{commentary}
  Although the ISA does not currently contain {\tt l\{b|h|w|d\}.aq[rl]} or {\tt s\{b|h|w|d\}.[aq]rl} instructions, we may add them as assembler pseudoinstructions to facilitate forwards compatibility with their potential future addition into the ISA.  These pseudoinstructions will generally assemble per the fence-based mappings of Section~\ref{sec:porting} until if and when the instructions are added to the ISA.  The RVWMO memory model is also designed to easily accommodate the possible future inclusion of such instructions.
\end{commentary}

\subsubsection*{Dependencies}
The definition of the RVWMO memory model depends in part on the notion of a syntactic dependency.
A register $r$ read by an instruction $b$ has a syntactic dependency on an instruction $a$ if $a$ precedes $b$ in program order, $r$ is not {\tt x0}, and either of the following hold:
\begin{enumerate}
  \item $r$ is written by $a$ and read by $b$, and no other instruction between $a$ and $b$ in program order modifies $r$
  \item There is some other instruction $i$ such that $r$ has a syntactic dependency on $i$, a register $s$ read by $i$ has a syntactic dependency on $a$, and $s$ {\em carries a dependency} from $r$
\end{enumerate}

The question of whether a given instruction carries a dependency from {\em rs1} and/or {\em rs2} to {\em rd} is specified as part of the definition of each instruction; however, we also provide a summary here.
For all instructions which read {\em rs1} and write {\em rd}, {\em rd} carries a dependency from {\em rs1}, with the exception of {\tt lb}, {\tt lh}, {\tt lw}, {\tt ld}, {\tt lr}, {\tt jalr}, and {\tt csrrw}.
For all instructions which read {\em rs1} and {\em rs2} and write {\em rd}, {\em rd} carries a dependency from both {\em rs1} and {\em rs2}, with the exception of {\tt amoswap}, {\tt amoadd}, {\tt amoand}, {\tt amoor}, {\tt amoxor}, and {\tt amomax}, and {\tt amomin}.

The specification of how dependencies are carried through CSRs will be added in a future revision.

Specific types of dependencies are defined as follows.
\begin{itemize}
\item For two instructions $a$ and $b$, $b$ has a syntactic address dependency on $a$ if a register used to calculate the address accessed by $b$ has a syntactic dependency on $a$.
\item $b$ has a syntactic data dependency on $a$ if $b$ is a store and a register used to calculate the data being written by $b$ has a syntactic dependency on $a$.
\item $b$ has a syntactic control dependency on $a$ if either:
  \begin{itemize}
  \item There exists a branch $m$ between $a$ and $b$ in program order such that a register checked as part of the condition of $m$ has a syntactic dependency on $a$, or
  \item There exists an indirect jump $m$ between $a$ and $b$ in program order such that the register used to calculate the jump address of $m$ has a syntactic dependency on $a$.
  \end{itemize}
\end{itemize}


\subsubsection*{Preserved Program Order}
The global memory order for any given execution of a program respects some but not necessarily all of each hart's program order.
The subset of program order respected by the global memory order for all executions is known as {\em preserved program order}.

\newcommand{\ppost}{$a$ and $b$ are accesses to overlapping memory addresses, and $b$ is a store}
\newcommand{\ppofence}{$a$ and $b$ are separated in program order by a fence, $a$ is in the predecessor set of the fence, and $b$ is in the successor set of the fence}
\newcommand{\ppoacquire}{$a$ is a load-acquire}
\newcommand{\ppoamoload}{$b$ is a load-SC}
\newcommand{\ppoamostore}{$a$ is a store-SC}
%\newcommand{\ppoloadtoacq}{$a$ is a load, $b$ is a load-acquire, and $a$ and $b$ are accesses to overlapping memory accesses}
%\newcommand{\ppoloadtoacq}{$a$ is a paired load, $b$ is a load-acquire, $a$ and $b$ are accesses to overlapping memory accesses, and there is no other store to an overlapping address between the store paired with $a$ and $b$}
\newcommand{\ppoloadtoacq}{$a$ is a paired store, $b$ is a load-acquire, $a$ and $b$ are accesses to overlapping memory accesses, and there is no other store to an overlapping address between the store paired with $a$ and $b$}
\newcommand{\pposuccess}{$a$ is a load, and there exists some $m$ such that $m$ has an address or data dependency on $a$ and $b$ has a success dependency on $m$}
\newcommand{\pporelease}{$b$ is a store-release}
\newcommand{\ppostrongacqrel}{$a$ is a store-release-RCsc and $b$ is a load-acquire-RCsc}
\newcommand{\ppoaddr}{$a$ is a load, and $b$ has a syntactic address dependency on $a$}
\newcommand{\ppodata}{$a$ is a load, $b$ is a store, and $b$ has a syntactic data dependency on $a$}
\newcommand{\ppoctrl}{$a$ is a load, $b$ is a store, and $b$ has a syntactic control dependency on $a$}
\newcommand{\ppordw}{$a$ and $b$ are loads, $x$ is a byte read by both $a$ and $b$, there is no store to $x$ between $a$ and $b$ in program order, and $a$ and $b$ return values for $x$ from different stores}
%\newcommand{\ppormwrfi}{$a$ is a paired load, $b$ is a load to an overlapping address, $b$ is a load-acquire, and there is no other store to an overlapping address between the store paired with $a$ and $b$}
\newcommand{\pporfiaq}{$a$ is a paired store, $b$ is a load to an overlapping address, $b$ is a load-acquire, and there is no store to overlapping memory location(s) between $a$ and $b$ in program order}
\newcommand{\ppoldstld}{$a$ and $b$ are loads, and there exists some store $m$ between $a$ and $b$ in program order such that $m$ has an address or data dependency on $a$, and $b$ reads a value written by $m$}
\newcommand{\ppoaddrpo}{$a$ is a load, $b$ is a store, and there exists some instruction $m$ between $a$ and $b$ in program order such that $m$ has an address dependency on $a$}
\newcommand{\ppoctrlcfence}{$a$ and $b$ are loads, $b$ has a syntactic control dependency on $a$, and there exists a {\tt fence.i} between the branch used to form the control dependency and $b$ in program order}
\newcommand{\ppoaddrpocfence}{$a$ is a load, there exists an instruction $m$ which has a syntactic address dependency on $a$, and there exists a {\tt fence.i} between $m$ and $b$ in program order}

The complete definition of preserved program order is as follows:
memory access $a$ precedes memory access $b$ in preserved program order (and hence also in the global memory order) if $a$ precedes $b$ in program order, $a$ and $b$ are both accesses to normal memory (i.e., not to I/O regions), and any of the following hold:

\begin{itemize}
  \item Basic same-address orderings:
    \begin{enumerate}
      \item\label{ppo:->st} \ppost
    \end{enumerate}
  \item Explicit synchronization
    \begin{enumerate}[resume]
      \item\label{ppo:fence} \ppofence
      \item\label{ppo:acquire} \ppoacquire
      %\item\label{ppo:loadtoacq} \ppoloadtoacq
      \item\label{ppo:release} \pporelease
      \item\label{ppo:strongacqrel} \ppostrongacqrel
      \item\label{ppo:amostore} \ppoamostore
      \item\label{ppo:amoload} \ppoamoload
    \end{enumerate}
  \item Dependencies
    \begin{enumerate}[resume]
      \item\label{ppo:addr} \ppoaddr
      \item\label{ppo:data} \ppodata
      \item\label{ppo:ctrl} \ppoctrl
      %\item\label{ppo:success} \pposuccess
    \end{enumerate}
  \item Same-address load-load ordering
    \begin{enumerate}[resume]
      \item\label{ppo:rdw} \ppordw
    \end{enumerate}
  %\item Orderings for atomics
  %  \begin{enumerate}[resume]
  %    %\item\label{ppo:rmwrfi} \ppormwrfi
  %    \item\label{ppo:rfiaq} \pporfiaq
  %  \end{enumerate}
  \item Pipeline dependency artifacts
    \begin{enumerate}[resume]
      \item\label{ppo:ld->st->ld} \ppoldstld
      \item\label{ppo:addrpo} \ppoaddrpo
      %\item\label{ppo:ctrlcfence} \ppoctrlcfence
      %\item\label{ppo:addrpocfence} \ppoaddrpocfence
    \end{enumerate}
\end{itemize}

\subsubsection*{Memory Model Axioms}

An execution of a RISC-V program obeys the RVWMO memory consistency model only if there exists a memory order conforming to {\em preserved program order} and satisfying the {\em load value axiom}, the {\em atomicity axiom}, and the {\em progress axiom}.

\newcommand{\loadvalueaxiom}{
  Each byte of each load returns the corresponding byte written by the whichever of the following two stores comes later in the global memory order:
  \begin{enumerate}
    \item the latest store to the same address and preceding the load in the global memory order
    \item the latest store to the same address and preceding the load in program order
  \end{enumerate}
}

\newcommand{\atomicityaxiom}{If $r$ and $w$ are a paired load and store, and if $s$ is any store to byte $x$ from which $r$ returns a value, then there can be no store from another hart to byte $x$ following $s$ and preceding $w$ in the global memory order.}

\newcommand{\progressaxiom}{No event may be preceded in the global memory order by an infinite sequence of other events.}

\textbf{Load Value Axiom:} \loadvalueaxiom

\textbf{Atomicity Axiom:} \atomicityaxiom

\textbf{Progress Axiom:} \progressaxiom

\subsection{Definition of the RVTSO Memory Model}
\label{sec:tso}

RISC-V cores which implement Ztso impose RVTSO onto all memory accesses.
RVTSO behaves just like RVWMO but with the following modifications:

\begin{itemize}
  \item All {\tt l\{b|h|w|d|r\}} instructions behave as if {\tt .aq} is set
  \item All {\tt s\{b|h|w|d|c\}} instructions behave as if {\tt .rl} is set
  \item All AMO instructions behave as if {\tt .aq} and {\tt .rl} are both set
\end{itemize}

These rules render PPO rules \ref{ppo:->st} and \ref{ppo:addr}--\ref{ppo:addrpo} redundant.
They also make redundant any non-I/O fences that do not have both {\tt .pw} and {\tt .sr} set.
Finally, they also imply that all AMO instructions are fully-fenced; nothing will be reordered past an AMO.

\begin{commentary}
  The definitions of RVTSO and Ztso are new and have not yet been reviewed as carefully as the RVWMO model.  For example, it is not yet properly specified how LR/SC will behave under RVTSO.
\end{commentary}

%% Operational Memory Model
\subsection{An Operational Memory Model}
%The formal model is defined in
%executable higher-order logic, written in the \texttt{lem} specification language.

This is an alternative exposition of the RVWMO.
\fixme{give some incentives for having another model: microarchitecture-like, more intuitive to some, executable, incremental...}
The operational model is expressed as a state machine, with states that are an abstract representation of hardware machine states.

An interactive version of the model, together with a library of litmus tests,
is provided online: \url{http://www.cl.cam.ac.uk/~pes20/rmem}

In this subsection, {\em load} means any instruction that can perform a memory load operation (including AMOs), and {\em store} means any instruction that can perform a memory store operation (including AMOs).
% {\em load-acquire} means any load that has {\tt .aq} set (including load-acquire-RCsc), and {\em store-release} means any store that has {\tt .rl} set (including store-release-RCsc).

\paragraph{Model states}
A model state consists of a shared memory and a tuple of thread model states.
% \begin{center}
%   \begin{tikzpicture} [node distance=1cm, font=\sffamily, >=latex]
%     \node[draw,text width=5cm,align=center] (mem) {Shared Memory};
%
%     \node[draw,text width=2cm,align=center] (thread 1) [above=of mem.north west, anchor=south west] {Thread 0};
%     \node[draw,text width=2cm,align=center] (thread n) [above=of mem.north east, anchor=south east] {Thread n};
%
%     \node[font=\bf] (thread dots) at ($(thread 1.east)!0.5!(thread n.west)$) [anchor=center] {\dots};
%     \node[font=\bf] (arrow dots) [node distance=7mm, below=of thread dots, anchor=center] {\dots};
%
%     \draw[->] ($(thread 1.south west)!0.33!(thread 1.south east)$) -- ++(-90:1cm);
%     \draw[<-] ($(thread 1.south west)!0.67!(thread 1.south east)$) -- ++(-90:1cm);
%
%     \draw[->] ($(thread n.south west)!0.33!(thread n.south east)$) -- ++(-90:1cm);
%     \draw[<-] ($(thread n.south west)!0.67!(thread n.south east)$) -- ++(-90:1cm);
%   \end{tikzpicture}
% \end{center}
The shared memory state records the most recent memory store operation to each location.
To handle atomic memory accesses ({\em  lr}, {\em sc} and AMOs), the memory is extended with a map (the atomics map) from load requests to sets of store slices, that associates a load request of an atomic load with the store slices it reads from (excluding stores that have been forwarded to the load and have not reached memory yet).

Each thread model state consists principally of a tree of instruction instances, some of which have been finished, and some of which have not.
Non-finished instruction instances can be subject to restart, e.g.~if they depend on an out-of-order or speculative load that turns out to be unsound.
The load part of AMOs can be marked as finished before the entire instruction is finished; when such instruction is restarted it is rolled back to the state it was in when the load part was marked as finished.
Conditional branch and indirect jump instructions may have multiple successors in the instructions tree.
When such instruction is finished, any un-taken alternative paths are discarded.

Each instruction instance state includes an execution state of the instruction's ISA pseudocode, which one can think of as a representation of the pseudocode control state, pseudocode call stack, and local variable values. \fixme{change ``pseudocode'' to something more RISC-V appropriate}
%
An instruction instance state also includes information, detailed below, about the instruction instance's memory and register footprints, its register and memory reads and writes, whether it is finished, etc.

\paragraph{Model transitions}
For any state, the model defines the set of allowed transitions, each of which is a single atomic step to a new abstract machine state.
Each transition arises from a single instruction instance; it will change the state of that instance, and it may depend on or change the rest of its thread state and the shared memory state.
% Instructions cannot be treated
% as atomic units: complete execution of a single instruction instance may
% involve many transitions, which can be interleaved with those of other
% instances in the same or other threads, and some of this is programmer-visible.
The transitions are introduced below and defined in Section~\ref{sec:omm:thread_trans}, with a precondition and a construction of the post-transition model state for each.

\noindent Transitions for all instructions:
\begin{itemize}
\item \nameref{omm:thread:fetch}: This transition represents a fetch and decode of a new instruction instance, as a program-order successor of a previously fetched instruction instance, or at the initial fetch address for a thread.
\item[$\circ$] \nameref{omm:thread:reg_read}: This is a read of a register value from the most recent program-order predecessor instruction instance that writes to that register.
\item[$\circ$] \nameref{omm:thread:reg_write}
\item[$\circ$] \nameref{omm:thread:sail_interp}: This covers pseudocode internal computation, function calls, etc.
\item[$\circ$] \nameref{omm:thread:finish}: At this point the instruction pseudocode is done, the instruction cannot be restarted or discarded, and all memory effects have taken place. For a conditional branches and indirect jump any non-taken program order successor branches are discarded.
\end{itemize}

\noindent Load instructions:
\begin{itemize}
\item[$\circ$] \nameref{omm:thread:initiate_mem_read}: At this point the memory footprint of the load is provisionally known and its individual memory load operations can start being satisfied.
\item \nameref{omm:thread:sat_by_forwarding}: This partially or entirely satisfies a single memory load operation by forwarding from its program order previous writes.
\item \nameref{omm:thread:sat_from_mem}: This entirely satisfies the outstanding slices of a single memory load operation, from memory.
\item[$\circ$] \nameref{omm:thread:complete_load}: At this point all the memory load operations of the load have been entirely satisfied and the instruction pseudocode can continue executing.
A load instruction can be subject to being restarted until the \nameref{omm:thread:finish} transition or \nameref{omm:thread:finish_load_part}.
But, under some conditions, the model might treat a load instruction as non-restartable if the \nameref{restart_condition} does not indicate the load can be restarted.
\end{itemize}

\noindent Store instructions:
\begin{itemize}
\item[$\circ$] \nameref{omm:thread:announce_mem_write_footprint}: At this point the memory footprint of the store is provisionally known.
\item[$\circ$] \nameref{omm:thread:initiate_mem_write}: At this point the memory store operations have their values and program order subsequent memory load operations can be satisfied by forwarding from them.
\item[$\circ$] \nameref{omm:thread:commit_store}: At this point the store is guaranteed to happen (it cannot be restarted or discarded), and the memory store operations can start being propagated to memory.
\item \nameref{omm:thread:prop_mem_write}: This propagates a single memory store operation to memory.
\item[$\circ$] \nameref{omm:thread:complete_store}: At this point all memory store operations have been propagated to memory, and the instruction pseudocode can continue executing.
\end{itemize}

\noindent {\em  sc.w/d} instructions:
\begin{itemize}
\item \nameref{omm:thread:excl_success}: This commits to the success of the {\em sc.w/d}.
\item \nameref{omm:thread:excl_fail}: This commits to the failure of the {\em sc.w/d}.
\end{itemize}

\noindent AMO instructions:
\begin{itemize}
\item[$\circ$] \nameref{omm:thread:finish_load_part}: At this point the load part of an AMO instruction is done, and the load part cannot be restarted or discarded. If the AMO instruction is restarted after the transition is taken, the instruction rolls back to its state right after this transition.
\end{itemize}

\noindent Fence instructions:
\begin{itemize}
\item[$\circ$] \nameref{omm:thread:commit_barrier}
\end{itemize}

\begin{commentary}
The transitions labelled~$\circ$ can always be taken eagerly, as soon as they are enabled, without excluding other behaviour; the $\bullet$ cannot.
\end{commentary}

\subsubsection{Intra-instruction Pseudocode Execution}
The intra-instruction semantics for each instruction instance is expressed as a state machine, essentially running the instruction pseudocode.
Given a pseudocode execution state, it computes the next state, as one of the following:

\begin{center}
\begin{tabular}{ll}
{\sc Load\_mem}($load\_kind$, $address$, $size$, $load\_continuation$)
    & Load request\\
{\sc Atomic\_res}($res\_continuation$)
    & {\em sc.w/d} result\\
{\sc Store\_ea}($store\_kind$, $address$, $size$, $next\_state$)
    & Store effective address\\
{\sc Store\_memv}($memory\_value$, $store\_continuation$)
    & Store value\\
{\sc Fence}($fence\_kind$, $next\_state$)
    & Fence\\
{\sc Read\_reg}($reg\_name$, $read\_continuation$)
    & Register memory load operation\\
{\sc Write\_reg}($reg\_name$, $register\_value$, $next\_state$)
    & Register write\\
{\sc Internal}($next\_state$)
    & Pseudocode internal step\\
{\sc Done}
    & End of pseudocode\\
\end{tabular}
\end{center}
Here memory values are lists of bytes, addresses are 64-bit \fixme{what about 32-bit?} numbers, load and store kinds identify whether they are regular, atomic, acquire, acquire-RCsc, release and/or release-RCsc operations \fixme{<- these should use the proper names}, register names identify a register and slice thereof (start and end bit indices), and the continuations describe how the instruction instance will continue for each value that might be provided by the surrounding memory model.
Stores are split into two steps, {\sc Store\_ea} and {\sc Store\_memv}.
We ensure these are paired in the pseudocode, but there may be other steps between them: it is observable that the {\sc Store\_ea} can occur before the value to be written is determined, because the potential memory footprint of the instruction becomes provisionally known then.

The pseudocode of each instruction performs at most one store, load, or fence, except for AMOs that preform exactly one load and one store.
Those memory accesses are then split apart into the architecturally atomic units by the thread semantics (see \nameref{omm:thread:initiate_mem_read} and \nameref{omm:thread:announce_mem_write_footprint} below).

Informally, each bit of a register read should be satisfied from a register write by the most recent (in program order) instruction instance that can write that bit, or from the thread's initial register state if there is no such.
That instance may not have executed its register write yet, in which case the register read should block.
Hence, it is essential to know the register write footprint of each instruction instance, which we calculate when the instruction instance is created (see the action of \nameref{omm:thread:fetch} below).
We ensure in the pseudocode that each instruction does at most one register write to each register bit, and also that it does not try to read a register value it just wrote.

Data-flow dependencies in the model emerge from the fact that register read has to wait for the appropriate register write to be executed (as described above).

\subsubsection{Instruction Instance States}\label{sec:omm:inst_state}
Each instruction instance $i$ has a state comprising:
\begin{itemize}
\item $program\_loc$, the memory address from which the instruction was fetched;
\item $instruction\_kind$, identifying whether this is a load, store, AMO, or fence instruction, each with the associated kind; or a conditional branch; or a `simple' instruction;
\item $regs\_in$, the set of input $reg\_name$s, as statically determined from the opcode;
\item $regs\_out$, the output $reg\_name$s, as statically determined from the opcode;
\item $pseudocode\_state$ (or sometimes just `state' for short), one of
  \begin{itemize}
  \item {\sc Plain} $next\_state$, ready to make a pseudocode transition;
  \item {\sc Pending\_mem\_loads} $load\_cont$, performing the memory load operation(s) of a load instruction; or
  \item {\sc Pending\_mem\_stores} $store\_cont$, performing the memory store operation(s) of a store instruction;
  \item {\sc Pending\_exception} $exception$, performing an exception;
  \end{itemize}
\item $reg\_reads$, the accumulated register reads, including their sources and values, of this instance's execution so far;
\item $reg\_writes$, the accumulated register writes, including dependency information to identify the register reads and memory load operations (by this instruction) that might have affected each;
\item $mem\_loads$, a set of memory load operations.
Each operation includes a memory footprint (an address and size) and, if the operation has already been satisfied, the set of store slices (each consisting of a memory store operation and a set of its byte indices) that satisfied it.
\item $mem\_stores$, a set of memory store operations.
Each operation includes a memory footprint and, when available, the memory value to be stored.
In addition, each operation has a flag that indicates whether it has been propagated (passed to the memory) or not.
\item $successful\_atomic$, for {\em sc.w/d}, indicates whether the instruction is committed to succeed or fail, or no commitment has been made yet; for AMOs this will always be set to true (committed to succeed).
\item information recording whether the instance is committed, finished, etc.
\end{itemize}

Memory load operations include their load kind and their memory footprint (their address and size), the as-yet-unsatisfied slices (the byte indices that have not been satisfied), and, for the satisfied slices, information about the memory store operation(s) that they were satisfied from.
%
Memory store operations include their store kind, their memory footprint, and their value.
When we refer to a memory store or load operation without mentioning the kind we mean the operation can be of any kind.
%
A load instruction which has initiated (so its load operation list $mem\_loads$ is not empty) and for which all its load operations are satisfied (i.e.~there are no unsatisfied slices) is said to be
\emph{entirely satisfied}.
%
A {\em lr.w/d} instruction instance is called {\em successful} if it is paired with a {\em sc.w/d} and that {\em sc.w/d} was committed to succeed.
If a successful {\em lr.w/d} has a memory load operation that is mapped, in the atomics map, to a memory store operation slice $ws$, we say the {\em lr.w/d} has an outstanding lock on $ws$.


\subsubsection{Thread States}
The model state of a single hardware thread includes:
\begin{itemize}
\item $thread\_id$, a unique identifier of the thread;
\item $register\_data$, the name, bit width, and start bit index for each register;
\item $initial\_register\_state$, the initial register value for each register;
\item $initial\_fetch\_address$, the initial fetch address for this thread;
\item $instruction\_tree$, a tree of the instruction instances that have been fetched (and not discarded), in program order.
\end{itemize}
\fixme{the above has more details than needed; only $thread\_id$ and $instruction\_tree$ are needed}


\subsubsection{Model Transitions}\label{sec:omm:thread_trans}
\fixme{say something about how we describe transitions?}

\paragraph{Fetch instruction}\label{omm:thread:fetch}
A possible program-order successor of instruction instance $i$ can be fetched from address $loc$ if:
\begin{enumerate}
\item it has not already been fetched, i.e., none of the immediate successors of $i$ in the thread's $instruction\_tree$ are from $loc$; and
\item $loc$ is a possible next fetch address for $i$:
  \begin{enumerate}
  \item for a conditional branch, either the successor address or the branch target address;
  \item for indirect jump instruction ({\em jalr}), when the target address is not yet determined, any address; and
  \item for any other instruction, the value written to the program counter register ({\em pc}); \fixme{<- this is tide to our Sail model}
  \end{enumerate}
\end{enumerate}

\begin{commentary}
The possible next fetch addresses are available immediately after fetching $i$ and the model does not need to wait for the pseudocode to write to {\em pc} (as is the case with GPRs), this allows out of order execution, and speculation past conditional branches and jumps.
For most instructions these addresses are easily attained from a static analysis of the instruction pseudocode.
The only exception to that is the indirect jump instruction ({\em jalr}).
%
The exhaustive search in the {\tt rmem} tool handles this instruction by running the exhaustive search multiple times with a growing set of possible next fetch addresses for each indirect jump.
The initial search uses empty sets, hence there is no fetch after indirect jump instruction until the pseudocode of the instruction writes to {\em pc}, and then we use that value for fetching the next instruction.
Before starting the next iteration of exhaustive search, we collect for each indirect jump (grouped by code location) the set of values it wrote to {\em pc} in all the executions in the previous search iteration, and use that as possible next fetch addresses of the instruction.
This process terminates when it reaches a fixed-point (no new fetch addresses are detected).
\end{commentary}

Action: construct a freshly initialized instruction instance $i'$ for the instruction in the program memory at $loc$, with state ``{\sc Plain} $next\_state$'', including the static information available from the pseudocode such as its $instruction\_kind$, $regs\_in$, and $regs\_out$, and add $i'$ to the thread's $instruction\_tree$ as a successor of $i$. If the instruction fails to decode, set the state of $i'$ to ``{\sc Pending\_exception} $exception$'' with $exception$ that describes the decoding error.

\begin{commentary}
This involves only the thread, not the storage subsystem, as we assume a fixed program rather than modelling fetches with memory reads; we do not model self-modifying code.
\end{commentary}


\paragraph{Initiate memory load operations}\label{omm:thread:initiate_mem_read}
An instruction instance $i$ with next pseudocode state {\sc Load\_mem}($load\_kind$, $address$, $size$, $load\_cont$) can initiate the corresponding memory reads if:
\begin{enumerate}
\item all program order previous {\em fence} instructions with {\em .sr} set are finished;
\item all program order previous {\em fence.i} instructions are finished; \fixme{it was decided that fence.i does not have any memory model effects; remove this}
\item if $i$ is a load-acquire-RCsc, all program order previous store-releases-RCsc are finished; and
\item all non-finished program order previous load-acquire instructions are entirely satisfied.
\end{enumerate}
Action:
\begin{enumerate}
\item Construct the appropriate memory load operations $mlos$:
  \begin{itemize}
  \item if $address$ is aligned to $size$ then $mlos$ is a single memory load operation of $size$ bytes from $address$;
  \item otherwise, $mlos$ is a set of $size$ memory load operations, each of one byte, from the addresses $address\ldots address+size-1$.
  \end{itemize}
\item set $i.mem\_loads$ to $mlos$; and
\item update the state of $i$ to ``{\sc Pending\_mem\_loads} $load\_cont$''.
\end{enumerate}


\paragraph{Satisfy memory load operation by forwarding from stores}\label{omm:thread:sat_by_forwarding}
For a load instruction instance $i$ in state ``{\sc Pending\_mem\_loads} $load\_cont$'', and a memory load operation, $mlo$ in $i.mem\_loads$ that has unsatisfied slices, the memory load operation can be partially or entirely satisfied by forwarding from unpropagated memory store operations by store instruction instances that are program order before $i$.

Let $msoss$ be the maximal set of unpropagated memory store operation slices from store instruction instances that are program order before $i$ (if $i$ is a load-acquire, exclude {\em sc.w/d} and AMO memory store operations \fixme{?}), that overlap with the unsatisfied slices of $mlo$, and which are not superseded by intervening stores that are either propagated or read from by this thread.
The last condition requires, for each memory store operation slice $msos$ in $msoss$ from instruction $i'$:
\begin{itemize}
\item that there is no store instruction program order between $i$ and $i'$ with a memory store operation overlapping $msos$; and
\item that there is no load instruction program order between $i$ and $i'$ that was satisfied from an overlapping memory store operation slice from a different thread.
\end{itemize}
Action:
\begin{enumerate}
\item update $mlo$ to indicate that it was satisfied by $msoss$; and
\item restart any speculative instructions which have violated coherence as a result of this, i.e., for every non-finished instruction $i'$ that is a program order successor of $i$, and every memory load operation $mlo'$ of $i'$ that was satisfied from $msoss'$, if there exists a memory store operation slice $msos'$ in $msoss'$, and an overlapping memory store operation slice from a different memory store operation in $msoss$, and $msos'$ is not from an instruction that is a program order successor of $i$, restart $i'$ and its data-flow dependents (including program order successors of restarted load-acquire instructions).
\end{enumerate}

\begin{commentary}
Forwarding memory store operations to a memory load operation might satisfy only some slices of the request, leaving other slices unsatisfied.

A consequence of the above is that store-release-RCsc memory store operations cannot be forwarded to load-acquires-RCsc:
a load-acquire-RCsc instruction cannot be in state ``{\sc Pending\_mem\_loads} $load\_cont$'' before all the program order previous store-release-RCsc instructions are finished, and $msoss$ does not include memory store operations from finished stores (as those must be propagated memory store operations).
\end{commentary}


\paragraph{Satisfy memory load operation from memory}\label{omm:thread:sat_from_mem}
For a load instruction instance $i$ in state ``{\sc Pending\_mem\_loads} $load\_cont$'', and a memory load operation $mlo$ in $i.mem\_loads$, that has unsatisfied slices, the memory load operation can be satisfied from memory under the condition that if $i$ is an AMO or a successful {\em lr.w/d} then no other AMO or successful {\em lr.w/d} from a different thread has an outstanding lock on the memory store operations $mlo$ is trying to read from.
Action: let $msoss$ be the memory store operation slices from memory covering the unsatisfied slices of $mlo$, and apply the action of \nameref{omm:thread:sat_by_forwarding}.
In addition, if $i$ is an AMO or a successful {\em lr.w/d}, union $msoss$ with the set of memory store operation slices $mlo$ is mapped to in the atomics map.

\begin{commentary}
Note that \nameref{omm:thread:sat_by_forwarding} might leave some slices of the memory load operation unsatisfied.
\nameref{omm:thread:sat_from_mem}, on the other hand, will always satisfy all the unsatisfied slices of the memory load operation.
\end{commentary}


\paragraph{Complete load instruction}\label{omm:thread:complete_load}
A load instruction instance $i$ in state ``{\sc Pending\_mem\_loads} $load\_cont$'' can be completed (not to be confused with finished) if all the memory load operations $i.mem\_loads$ are entirely satisfied (i.e.~there are no unsatisfied slices).
Action: update the state of $i$ to ``{\sc Plain} $load\_cont(memory\_value)$'', where $memory\_value$ is assembled from all the memory store operation slices that satisfied $i.mem\_loads$.


\paragraph{Finish load part of AMO instruction}\label{omm:thread:finish_load_part}
An AMO instruction instance $i$ that has completed the load part, i.e., \nameref{omm:thread:complete_load} has been taken, can be marked as such if the conditions of \nameref{omm:thread:finish} are satisfied, except for the fully determined data condition for register {\em rs2}.
Action: mark the load part of $i$ as finished. If later on $i$ has to be restarted, reset its state to the current state.


% \paragraph{Guarantee the success of {\em sc.w/d}}\label{omm:thread:excl_success}
% A {\em sc.w/d} instruction instance $i$ with next pseudocode state {\sc Atomic\_res}($res\_cont$) can be guaranteed to succeed if:
% \begin{enumerate}
% \item $i$ has not been made to fail (as recorded in $i.successful\_atomic$);
% \item $i$ is paired with a {\em lr.w/d} $i'$; and
% \item if $i'$ has already been satisfied (not necessarily entirely), let $msoss$ be the set of propagated write slices $i'$ has read from, then, no slice in $msoss$ has been overwritten (in memory) by a write from a different thread, and no other AMO or successful {\em lr.w/d} from a different thread has an outstanding lock on a write slice from $msoss$.
% \end{enumerate}
% Action:
% \begin{enumerate}
% \item record in $i.successful\_atomic$ that $i$ will be successful;
% \item if $i'$ has already been satisfied, union $msoss$ with the set of write slices the memory load operation of $i'$ is mapped to in the atomics map, where $msoss$ is as above; and
% \item update the state of $i$ to ``{\sc Plain} $res\_cont(true)$''.
% \end{enumerate}
%
%
% \paragraph{Make a {\em sc.w/d} fail}\label{omm:thread:excl_fail}
% A {\em sc.w/d} instruction instance $i$ with next pseudocode state {\sc Atomic\_res}($res\_cont$) can be made to fail if the {\em sc.w/d} has not been guaranteed to succeed (as recorded in $i.successful\_atomic$).
% Action:
% \begin{enumerate}
% \item record in $i.successful\_atomic$ that the {\em sc.w/d} was made to fail; and
% \item update the state of $i$ to ``{\sc Plain} $res\_cont(false)$''.
% \end{enumerate}
%
% \begin{commentary}
% Note that the promise-success transition is enabled before the {\em sc.w/d} commits, and we do not require it to have a fully-determined address or to be non-restartable.
% As a result, a {\em sc.w/d} that has already promised its success might be restarted.
% Since other instructions may rely on its promise, the restart will not affect the value of $i.successful\_atomic$.
% Instead, when the {\em sc.w/d} is restarted it will take the same promise/failure transition as before its restart --- based on the value of $i.successful\_atomic$.
% \end{commentary}

\paragraph{Initiate memory store operation footprints of store instruction}\label{omm:thread:announce_mem_write_footprint}
An instruction instance $i$ with next pseudocode state {\sc Store\_ea}($store\_kind$, $address$, $size$, $next\_state$) can announce its pending memory store operation footprint.
Action:
\begin{enumerate}
\item construct the appropriate memory store operations $msos$:
  \begin{itemize}
  \item if $address$ is aligned to $size$ then $msos$ is a single memory store operation of $size$ bytes to $address$;
  \item otherwise $msos$ is a set of $size$ memory store operations, each of one byte size, to the addresses $address\ldots address+size-1$.
  \end{itemize}
\item set $i.mem\_stores$ to $msos$; and
\item update the state of $i$ to ``{\sc Plain} $next\_state$''.
\end{enumerate}

\begin{commentary}
Note that after taking the transition above the memory store operations do not yet have their values, but other, non-overlapping program order following memory store operations, can already propagate.
\end{commentary}


\paragraph{Instantiate memory store operation values of store instruction}\label{omm:thread:initiate_mem_write}
An instruction instance $i$ with next pseudocode state {\sc Store\_memv}($memory\_value$, $store\_cont$) can initiate the values of the memory store operations $i.mem\_stores$.
Action:
\begin{enumerate}
\item split $memory\_value$ between the memory store operations $i.mem\_stores$; and
\item update the state of $i$ to ``{\sc Pending\_mem\_stores} $store\_cont$''.
\end{enumerate}


\paragraph{Commit store instruction}\label{omm:thread:commit_store}
For an uncommitted store instruction $i$ in state ``{\sc Pending\_mem\_stores} $store\_cont$'', $i$ can be committed if:
\begin{enumerate}
\item $i$ has fully determined data (i.e.~the register reads cannot change, see Section~\ref{sec:aux});
\item all program order previous conditional branch instructions are finished;
\item all program order previous {\em fence} instructions with {\em .sw} set are finished;
\item all program order previous {\em fence.i} instructions are finished; \fixme{remove?}
\item all program order previous load-acquire instructions are finished;
\item  if $i$ is a store-release, all program order previous memory access instructions are finished;
\item\label{omm:commit_store:prev_addrs} all program order previous memory access instructions have a fully determined memory footprint;
\item\label{omm:commit_store:prev_stores} all program order previous store instructions, except for {\em sc.w/d} that failed, have initiated and so have non-empty $mem\_stores$; and
\item\label{omm:commit_store:prev_loads} all program order previous load instructions have initiated and so have non-empty $mem\_loads$.
\end{enumerate}
Action: record $i$ as committed.

\begin{commentary}
Notice that if condition \ref{omm:commit_store:prev_addrs} is satisfied the conditions \ref{omm:commit_store:prev_stores} and \ref{omm:commit_store:prev_loads} are also satisfied, or will be satisfied after taking just eager transitions.
Hence requiring them does not strengthen the model.
It guarantees that previous memory access instructions have taken enough transitions to make the memory operations visible for the condition checks of the \nameref{omm:thread:prop_mem_write} transition.
\end{commentary}


\paragraph{Propagate memory store operation}\label{omm:thread:prop_mem_write}
For an instruction $i$ in state ``{\sc Pending\_mem\_stores} $store\_cont$'', and an unpropagated memory store operation $mso$ in $i.mem\_stores$, the memory store operation can be propagated if:
\begin{enumerate}
\item all memory memory store operations of program order previous store instructions that overlap with $mso$ have already propagated;
\item all memory load operations of program order previous load instructions that overlap with $mso$ have already been satisfied, and (the load instructions) are non-restartable (see \nameref{restart_condition});
\item all memory load operations satisfied by forwarding $mso$ are entirely satisfied; and
\item no AMO or successful {\em lr.w/d} from a different thread has an outstanding lock on a memory store operation slice that overlaps with $mso$.
\end{enumerate}
Action:
\begin{enumerate}
\item update the memory with $mso$;
\item record $mso$ as propagated;
\item restart any speculative instructions which have violated coherence as a result of this, i.e., for every non-finished instruction $i'$ program order after $i$ and every memory load operation $mlo'$ of $i'$ that was satisfied from $msoss'$, if there exists a memory store operation slice $msos'$ in $msoss'$ that overlaps with $mso$ and is not from $mso$, and $msos'$ is not from a program order successor of $i$, restart $i'$ and its data-flow dependents; and
\item for every AMO and successful {\em lr.w/d} that has read from $mso$ (by forwarding), add the slices of $mso$ this load reads from to the set of memory store operation slices the memory load operation of the load is mapped to in the exclusives map.
\end{enumerate}

\paragraph{Complete store instruction}\label{omm:thread:complete_store}
A store instruction $i$ in state ``{\sc Pending\_mem\_stores} $store\_cont$'', for which all the memory store operations in $i.mem\_stores$ have been propagated, can be completed (not to be confused with finished).
Action: update the state of $i$ to ``{\sc Plain} $store\_cont(true)$''.


\paragraph{Commit fence}\label{omm:thread:commit_barrier}
A fence instruction $i$ in state ``{\sc Plain} $next\_state$'' where $next\_state$ is {\sc Fence}($fence\_kind$, $next\_state'$) can be committed if:
\begin{enumerate}
\item all program order previous conditional branch instructions are finished;
\fixme{this looks stronger than intended, but actually it has no observable effect for most fences; the exception is ``fence w,r'', e.g., MP+fence.w.w+ctrlfence.w.r is allowed by Daniel's model but forbidden as a consequence of this condition. Fixing this means changing the invariant that finished instructions are never discarded, to finished load/store instructions are never discarded. Is RISC-V going to include ``fence w,r''?}
\item if $i$ has {\em .pr} set, all program order previous load instructions are finished;
\item if $i$ has {\em .pw} set, all program order previous store instructions are finished; and
\item if $i$ is a {\em fence.i} instruction, all program order previous memory access instructions have fully determined memory footprints. \fixme{remove?}
\end{enumerate}
Action: update the state of $i$ to ``{\sc Plain} $next\_state'$''.


\paragraph{Register read}\label{omm:thread:reg_read}
An instruction instance $i$ with next pseudocode state {\sc Read\_reg}($reg\_name$, $read\_cont$) can do a $reg\_name$ register read of if every instruction instance that it needs to read from has already performed the expected $reg\_name$ register write.

Let $read\_sources$ include, for each bit of $reg\_name$, the write to that bit by the most recent (in program order) instruction instance that can write to that bit, if any. If there is no such instruction, the source is the initial register value from $initial\_register\_state$.
Let  $register\_value$ be the value assembled from $read\_sources$.

Action:
\begin{enumerate}
\item add $reg\_name$ to $i.reg\_reads$ with $read\_sources$ and $register\_value$; and
\item update the state of $i$ to ``{\sc Plain} $read\_cont(register\_value)$''.
\end{enumerate}


\paragraph{Register write}\label{omm:thread:reg_write}
An instruction instance $i$ with next pseudocode state {\sc Write\_reg}($reg\_name$, $register\_value$, $next\_state'$) can do a $reg\_name$ register write.
Action:
\begin{enumerate}
\item add $reg\_name$ to $i.reg\_writes$ with $write\_deps$ and $register\_value$; and
\item update the state of $i$ to ``{\sc Plain} $next\_state'$''.
\end{enumerate}
where $write\_deps$ is the set of all $read\_sources$ from $i.reg\_reads$ and a flag that is set to true if $i$ is a load instruction that has already been entirely satisfied.


\paragraph{Pseudocode internal step}\label{omm:thread:sail_interp}
An instruction instance $i$ with next pseudocode state {\sc Internal}($next\_state'$) can do that pseudocode-internal step.
Action: Update the state of $i$ to ``{\sc Plain} $next\_state'$''.


\paragraph{Finish instruction}\label{omm:thread:finish}
A non-finished instruction $i$ with next pseudocode state {\sc Done} can be finished if:
\begin{enumerate}
\item if $i$ is a load instruction:
  \begin{enumerate}
  \item all program order previous load-acquire instructions are finished; and
  \item it is guaranteed that the values read by the memory load operations of $i$ will not cause coherence violations, i.e., for any program order previous instruction instance $i'$, let $cfp$ be the combined footprint of propagated memory store operations from store instructions program order between $i$ and $i'$ and fixed memory store operations that were forwarded to $i$ from store instructions program order between $i$ and $i'$ including $i'$, and let $cfp'$ be the complement of $cfp$ in the memory footprint of $i$.
  If $cfp'$ is not empty:
    \begin{enumerate}
    \item $i'$ has a fully determined memory footprint;
    \item $i'$ has no unpropagated memory store operations that overlap with $cfp'$; and
    \item if $i'$ is a load with a memory footprint that overlaps with $cfp'$, then all the memory load operations of $i'$ that overlap with $cfp'$ are satisfied and $i'$ can not be restarted (see \nameref{restart_condition}).
    \end{enumerate}
  Here, a memory store operation is called fixed if the store instruction has a fully determined data.
  \end{enumerate}
\item $i$ has a fully determined data; and
\item all program order previous conditional branch and indirect jump instructions are finished.
\end{enumerate}
Action:
\begin{enumerate}
\item if $i$ is a conditional branch or indirect jump instruction, discard any untaken path of execution, i.e., remove any (non-finished) instructions that are not reachable by the branch/jump taken in $instruction\_tree$; and
\item record the instruction as finished, i.e., set $finished$ to $true$.
\end{enumerate}



%%> \subsubsection{Auxiliary Definitions}\label{sec:aux}
%%>
%%> \paragraph{Fully determined}
%%> % This is the
%%> Informally, an instruction is said to have \emph{fully determined data}
%%> if the load instructions feeding its input registers are finished.
%%> Similarly, it is said to have \emph{fully determined memory footprint}
%%> if the load instructions feeding its memory location register are finished.
%%> %
%%> Formally, we first define the notion of \emph{fully determined register write}:
%%> a register write $mso$, of instruction $i$, with the associated
%%> \lem{write\_deps} from \lem{i.reg\_writes} is said to be \emph{fully determined}
%%> if one of the following conditions hold:
%%> \begin{enumerate}
%%>   \item $i$ is finished\ifexcl{\ (just the load part for AMOs)}; or
%%>   \item the load flag in \lem{write\_deps} is \lem{false} and every register
%%>   write in \lem{write\_deps} is fully determined.
%%> \end{enumerate}
%%> %
%%> Now, an instruction $i$ is said to have a \emph{fully determined data} if all the
%%> register writes of \lem{read\_sources} in \lem{i.reg\_reads} are fully determined;
%%> %
%%> and, $i$ is said to have a \emph{fully determined memory footprint}
%%> if all the register writes of \lem{read\_sources} in \lem{i.reg\_reads} that are
%%> associated with registers that feed into $i$'s memory access footprint are fully
%%> determined.
%%>
%%>
%%> \paragraph{Restart condition}\label{restart_condition}
%%> To determine if instruction $i$ might be restarted we use the
%%> following condition:
%%> $i$ is a non-finished instruction and at least one of the following holds,
%%> \begin{enumerate}
%%>     \item there exists a store instruction \lem{s} and an unpropagated write $mso$ of \lem{s} such that applying the
%%>     action of the \nameref{omm:thread:prop_mem_write} transition to $mso$ will
%%>     result in the restart of $i$;
%%>
%%>     \item there exists a non-finished load instruction \lem{l} and a memory load operation \lem{rr} of \lem{l} such that applying the
%%>     action of the \nameref{omm:thread:sat_from_mem} transition to \lem{rr}
%%>     will result in the restart of $i$ (even if \lem{rr} is already satisfied); or
%%>
%%>     \item there exists a non-finished instruction $i'$ that might be restarted and
%%>     $i$ is in its data-flow dependents\ifrelacq{, or $i'$ is a load-acquire}.
%%> \end{enumerate}
%%>
%%>
%%> \comment[Shaked]{
%%> \subsubsection{to do to the model}
%%>
%%> push ``potential mem write'' -> ``pending mem write'' everywhere
%%>
%%> shaked to fix fully-determined-address (and turn it into footprint)
%%>
%%> }


\section{Memory Ordering Instructions}
\label{sec:fence}

\vspace{-0.2in}
\begin{center}
\begin{tabular}{F@{}IIIIIIIIF@{}F@{}F@{}S}
\\
\instbitrange{31}{28} &
\multicolumn{1}{c}{\instbit{27}} &
\multicolumn{1}{c}{\instbit{26}} &
\multicolumn{1}{c}{\instbit{25}} &
\multicolumn{1}{c}{\instbit{24}} &
\multicolumn{1}{c}{\instbit{23}} &
\multicolumn{1}{c}{\instbit{22}} &
\multicolumn{1}{c}{\instbit{21}} &
\multicolumn{1}{c}{\instbit{20}} &
\instbitrange{19}{15} &
\instbitrange{14}{12} &
\instbitrange{11}{7} &
\instbitrange{6}{0} \\
\hline
\multicolumn{1}{|c|}{0} &
\multicolumn{1}{c|}{PI} &
\multicolumn{1}{c|}{PO} &
\multicolumn{1}{c|}{PR} &
\multicolumn{1}{c|}{PW} &
\multicolumn{1}{|c|}{SI} &
\multicolumn{1}{c|}{SO} &
\multicolumn{1}{c|}{SR} &
\multicolumn{1}{c|}{SW} &
\multicolumn{1}{c|}{rs1} &
\multicolumn{1}{c|}{funct3} &
\multicolumn{1}{c|}{rd} &
\multicolumn{1}{c|}{opcode} \\
\hline
4 & 1 & 1 & 1 & 1 & 1 & 1 & 1 & 1 & 5 & 3 & 5 & 7 \\
0 & \multicolumn{4}{c}{predecessor} & \multicolumn{4}{c}{successor} & 0 & FENCE & 0 & MISC-MEM \\
\end{tabular}
\end{center}

The FENCE instruction is used to order device I/O and
memory accesses as viewed by other RISC-V harts and external devices
or coprocessors.  Any combination of device input (I), device output
(O), memory reads (R), and memory writes (W) may be ordered with
respect to any combination of the same.  Informally, no other RISC-V
hart or external device can observe any operation in the {\em
  successor} set following a FENCE before any operation in the {\em
  predecessor} set preceding the FENCE.  The execution environment
will define what I/O operations are possible, and in particular, which
load and store instructions might be treated and ordered as device
input and device output operations respectively rather than memory
reads and writes.  For example, memory-mapped I/O devices will
typically be accessed with uncached loads and stores that are ordered
using the I and O bits rather than the R and W bits.  Instruction-set
extensions might also describe new coprocessor I/O instructions that
will also be ordered using the I and O bits in a FENCE.

The unused fields in the FENCE instruction, {\em imm[11:8]}, {\em rs1}, and
{\em rd}, are reserved for finer-grain fences in future extensions.  For
forward compatibility, base implementations shall ignore these fields, and
standard software shall zero these fields.

\begin{commentary}
We chose a relaxed memory model to allow high performance from simple
machine implementations and from likely future
coprocessor or accelerator extensions.  We separate out I/O ordering
from memory R/W ordering to avoid unnecessary serialization within a
device-driver hart and also to support alternative non-memory paths
to control added coprocessors or I/O devices.  Simple implementations
may additionally ignore the {\em predecessor} and {\em successor}
fields and always execute a conservative fence on all operations.
\end{commentary}

\vspace{-0.4in}
\begin{center}
\begin{tabular}{M@{}R@{}S@{}R@{}O}
\\
\instbitrange{31}{20} &
\instbitrange{19}{15} &
\instbitrange{14}{12} &
\instbitrange{11}{7} &
\instbitrange{6}{0} \\
\hline
\multicolumn{1}{|c|}{imm[11:0]} &
\multicolumn{1}{c|}{rs1} &
\multicolumn{1}{c|}{funct3} &
\multicolumn{1}{c|}{rd} &
\multicolumn{1}{c|}{opcode} \\
\hline
12 & 5 & 3 & 5 & 7 \\
0 & 0 & FENCE.I & 0 & MISC-MEM \\
\end{tabular}
\end{center}

The FENCE.I instruction is used to synchronize the instruction and
data streams.  RISC-V does not guarantee that stores to instruction
memory will be made visible to instruction fetches on the same RISC-V
hart until a FENCE.I instruction is executed.  A FENCE.I instruction
only ensures that a subsequent instruction fetch on a RISC-V hart
will see any previous data stores already visible to the same RISC-V
hart.  FENCE.I does {\em not} ensure that other RISC-V harts'
instruction fetches will observe the local hart's stores in a
multiprocessor system. To make a store to instruction memory visible
to all RISC-V harts, the writing hart has to execute a data FENCE
before requesting that all remote RISC-V harts execute a FENCE.I.

The unused fields in the FENCE.I instruction, {\em imm[11:0]}, {\em rs1}, and
{\em rd}, are reserved for finer-grain fences in future extensions.  For
forward compatibility, base implementations shall ignore these fields, and
standard software shall zero these fields.

\begin{commentary}
Because FENCE.I only orders stores with a hart's own instruction fetches,
application code should only rely upon FENCE.I if the application thread will
not be migrated to a different hart.  The ABI will provide mechanisms for
multiprocessor instruction-stream synchronization.
\end{commentary}

\begin{commentary}
The FENCE.I instruction was designed to support a wide variety of
implementations.  A simple implementation can flush the local
instruction cache and the instruction pipeline when the FENCE.I is
executed.  A more complex implementation might snoop the instruction
(data) cache on every data (instruction) cache miss, or use an
inclusive unified private L2 cache to invalidate lines from the
primary instruction cache when they are being written by a local store
instruction.  If instruction and data caches are kept coherent in this
way, then only the pipeline needs to be flushed at a FENCE.I.

We considered but did not include a ``store instruction word''
instruction (as in MAJC~\cite{majc}).  JIT compilers may generate a
large trace of instructions before a single FENCE.I, and amortize any
instruction cache snooping/invalidation overhead by writing translated
instructions to memory regions that are known not to reside in the
I-cache.
\end{commentary}
