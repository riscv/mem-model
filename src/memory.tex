\newcommand{\diagram}{\textcolor{red}{(picture coming soon)}}

\chapter{RVWMO Explanatory Material}
\label{sec:explanation}
This section provides more explanation for the RVWMO memory model, using more informal language and concrete examples.
These are intended to clarify the meaning and intent of the axioms and preserved program order rules.
%In case of any discrepancy between the informal descriptions here and the formal descriptions elsewhere, the formal definitions should be considered authoritative.

\section{Why RVWMO?}
\label{sec:whynottso}

Memory consistency models fall along a loose spectrum from weak to strong.
Weak memory models allow more hardware implementation flexibility and deliver arguably better performance, performance per watt, power, scalability, and hardware verification overheads than strong models, at the expense of a more complex programming model.
Strong models provide simpler programming models, but at the cost of imposing more restrictions on the kinds of (non-speculative) hardware optimizations that can be performed in the pipeline and in the memory system, and in turn imposing some cost in terms of power, area overhead, and verification burden.

RISC-V has chosen the RVWMO memory model, a variant of release consistency.
This places it in between the two extremes of the memory model spectrum.
The RVWMO memory model enables architects to build simple implementations, aggressive implementations, implementations embedded deeply inside a much larger system and subject to complex memory system interactions, or any number of other possibilities, all while simultaneously being strong enough to support programming language memory models at high performance.

To facilitate the porting of code from other architectures, some hardware implementations may choose to implement the Ztso extension, which provides stricter RVTSO ordering semantics by default.
Code written for RVWMO is automatically and inherently compatible with RVTSO, but code written assuming RVTSO is not guaranteed to run correctly on RVWMO implementations.
In fact, most RVWMO implementations will (and should) simply refuse to run RVTSO-only binaries.
Each implementation must therefore choose whether to prioritize compatibility with RVTSO code (e.g., to facilitate porting from x86) or whether to instead prioritize compatibility with other RISC-V cores implementing RVWMO.

Some fences and/or memory ordering annotations in code written for RVWMO may become redundant under RVTSO; the cost that the default of RVWMO imposes on Ztso implementations is the incremental overhead of fetching those fences (e.g., {\tt fence~r,rw} and {\tt fence rw,w}) which become no-ops on that implementation.
However, these fences must remain present in the code if compatibility with non-Ztso implementations is desired.

\section{Litmus Tests}
The explanations in this chapter make use of {\em litmus tests}, or small programs designed to test or highlight one particular aspect of a memory model.
Figure~\ref{fig:litmus:sample} shows an example of a litmus test with two harts.
As a convention for this figure and for all figures that follow in this chapter, we assume that {\tt s0}--{\tt s2} are pre-set to the same value in all harts and that {\tt s0} holds the address labeled {\tt x}, {\tt s1} holds {\tt y}, and {\tt s2} holds {\tt z}, where {\tt x}, {\tt y}, and {\tt z} are disjoint memory locations aligned to 8 byte boundaries.
Each figure shows the litmus test code on the left, and a visualization of one particular valid or invalid execution on the right.

\begin{figure}[h!]
  \centering
  {
    \tt\small
    \begin{tabular}{cl||cl}
    \multicolumn{2}{c}{Hart 0} & \multicolumn{2}{c}{Hart 1} \\
    \hline
          & $\vdots$    &     & $\vdots$    \\
          & li t1,1     &     & li t4,4     \\
      (a) & sw t1,0(s0) & (e) & sw t4,0(s0) \\
          & $\vdots$    &     & $\vdots$    \\
          & li t2,2     &     &             \\
      (b) & sw t2,0(s0) &     &             \\
          & $\vdots$    &     & $\vdots$    \\
      (c) & lw a0,0(s0) &     &             \\
          & $\vdots$    &     & $\vdots$    \\
          & li t3,3     &     & li t5,5     \\
      (d) & sw t3,0(s0) & (f) & sw t5,0(s0) \\
          & $\vdots$    &     & $\vdots$    \\
    \end{tabular}
  }
  ~~~~
  \diagram
  \caption{A sample litmus test}
  \label{fig:litmus:sample}
\end{figure}

Litmus tests are used to understand the implications of the memory model in specific concrete situations.
For example, in the litmus test of Figure~\ref{fig:litmus:sample}, the final value of {\tt a0} in the first hart can be either 2, 4, or 5, depending on the dynamic interleaving of the instruction stream from each hart at runtime.
However, in this example, the final value of {\tt a0} in Hart 0 will never be 1 or 3; intuitively, the value 1 will no longer be visible at the time the load executes, and the value 3 will not yet be visible by the time the load executes.

We analyze this test and many others below.

\section{Explaining the RVWMO Rules}
In this section, we provide explanation and examples for all of the RVWMO rules and axioms.

\subsection{Preserved Program Order and Global Memory Order}
Preserved program order represents the subset of program order that must be respected within the global memory order.
Conceptually, events from the same hart that are ordered by preserved program order must appear in that order from the perspective of other harts and/or observers.
Events from the same hart that are not ordered by preserved program order, on the other hand, may appear reordered from the perspective of other harts and/or observers.

Informally, the global memory order represents the order in which loads and stores perform.
The formal memory model literature has moved away from specifications built around the concept of performing, but the idea is still useful for building up informal intuition.
A load is said to have performed when its return value is determined.
A store is said to have performed not when it has executed inside the pipeline, but rather only when its value has been propagated to globally visible memory.
In this sense, the global memory order also represents the contribution of the coherence protocol and/or the rest of the memory system to interleave the (possibly reordered) memory accesses being issued by each hart into a single total order agreed upon by all harts.

The order in which loads perform does not always directly correspond to the relative age of the values those two loads return.
In particular, a load $b$ may perform before another load $a$ to the same address (i.e., $b$ may execute before $a$, and $b$ may appear before $a$ in the global memory order), but $a$ may nevertheless return an older value than $b$.
This discrepancy captures (among other things) the reordering effects of buffering placed between the core and memory.
For example, $b$ may have returned a value from a store in the store buffer, while $a$ may have ignored that younger store and read an older value from memory instead.
To account for this, at the time each load performs, the value it returns is determined by the load value axiom, not just strictly by determining the most recent store to the same address in the global memory order, as described below.

\subsection{\nameref*{rvwmo:ax:load}}
\label{sec:memory:loadvalueaxiom}
\begin{tabular}{p{1cm}|p{12cm}} &
\nameref{rvwmo:ax:load}: \loadvalueaxiom
\end{tabular}

Preserved program order is {\em not} required to respect the ordering of a store followed by a load to an overlapping address.
This complexity arises due to the ubiquity of store buffers in nearly all implementations.
Informally, the load may perform (return a value) by forwarding from the store while the store is still in the store buffer, and hence before the store itself performs (writes back to globally visible memory).
Any other hart will therefore observe the load as performing before the store.

\begin{figure}[h!]
  \centering
  {
    \tt\small
    \begin{tabular}{cl||cl}
    \multicolumn{2}{c}{Hart 0} & \multicolumn{2}{c}{Hart 1} \\
    \hline
          & li t1, 1    &     & li t1, 1    \\
      (a) & sw t1,0(s0) & (e) & sw t1,0(s1) \\
      (b) & lw a0,0(s0) & (f) & lw a2,0(s1) \\
      (c) & fence r,r   & (g) & fence r,r   \\
      (d) & lw a1,0(s1) & (h) & lw a3,0(s0) \\
    \end{tabular}
  }
  ~~~~
  \diagram
  \caption{A store buffer forwarding litmus test}
  \label{fig:litmus:storebuffer}
\end{figure}

Consider the litmus test of Figure~\ref{fig:litmus:storebuffer}.
When running this program on an implementation with store buffers, it is possible to arrive at the final outcome
{\tt a0=1}, {\tt a1=0}, {\tt a2=1}, {\tt a3=0}
as follows:
\begin{itemize}
  \item (a) executes and enters the first hart's private store buffer
  \item (b) executes and forwards its return value 1 from (a) in the store buffer
  \item (c) executes since all previous loads (i.e., (b)) have completed
  \item (d) executes and reads the value 0 from memory
  \item (e) executes and enters the second hart's private store buffer
  \item (f) executes and forwards its return value 1 from (d) in the store buffer
  \item (g) executes since all previous loads (i.e., (f)) have completed
  \item (h) executes and reads the value 0 from memory
  \item (a) drains from the first hart's store buffer to memory
  \item (e) drains from the second hart's store buffer to memory
\end{itemize}
Therefore, the memory model must be able to account for this behavior.

To put it another way, suppose the definition of preserved program order did include the following hypothetical rule:
memory access $a$ precedes memory access $b$ in preserved program order (and hence also in the global memory order) if $a$ precedes $b$ in program order and $a$ and $b$ are accesses to the same memory location, $a$ is a write, and $b$ is a read.  Call this ``Rule X''.  Then we get the following:

\begin{itemize}
  \item (a) precedes (b): by rule X
  \item (b) precedes (d): by rule \ref{ppo:fence}
  \item (d) precedes (e): by the load value axiom.  Otherwise, if (e) preceded (d), then (d) would be required to return the value 1.  (This is a perfectly legal execution; it's just not the one in question)
  \item (e) precedes (f): by rule X
  \item (f) precedes (h): by rule \ref{ppo:fence}
  \item (h) precedes (a): by the load value axiom, as above.
\end{itemize}
The global memory order must be a total order and cannot be cyclic, because a cycle would imply that every event in the cycle happens before itself, which is impossible.
Therefore, the execution proposed above would be forbidden, and hence the addition of rule X would forbid implementations with store buffer forwarding, which would clearly be undesirable.

Nevertheless, even if (b) precedes (a) and/or (f) precedes (e) in the global memory order, the only sensible possibility in this example is for (b) to return the value written by (a), and likewise for (f) and (e).  This combination of circumstances is what leads to the second option in the definition of the load value axiom.
Even though (b) precedes (a) in the global memory order, (a) will still be visible to (b) by virtue of sitting in the store buffer at the time (b) executes.
Therefore, even if (b) precedes (a) in the global memory order, (b) should return the value written by (a) because (a) precedes (b) in program order.
Likewise for (e) and (f).

\subsection{\nameref*{rvwmo:ax:atom}}
\label{sec:memory:atomicityaxiom}
\begin{tabular}{p{1cm}|p{12cm}} &
\nameref{rvwmo:ax:atom} (for Aligned Atomics): \atomicityaxiom
\end{tabular}

The RISC-V architecture decouples the notion of atomicity from the notion of ordering.  Unlike architectures such as TSO, RISC-V atomics under RVWMO do not impose any ordering requirements by default.  Ordering semantics are only guaranteed by the PPO rules that otherwise apply.

RISC-V contains two types of atomics: AMOs and LR/SC pairs.
These conceptually behave differently, in the following way.
LR/SC behave as if the old value is brought up to the core, modified, and written back to memory, all while a reservation is held on that memory location.
AMOs on the other hand conceptually behave as if they are performed directly in memory.
AMOs are therefore inherently atomic, while LR/SC pairs are atomic in the slightly different sense that the memory location in question will not be modified by another hart during the time the original hart holds the reservation.

\begin{figure}[h!]
  \centering\small
  \begin{verbbox}
  (a) lr.d a0, 0(s0)
  (b) sd   t1, 0(s0)
  (c) sc.d t2, 0(s0)
  \end{verbbox}
  \theverbbox
  ~~~~~~
  \begin{verbbox}
  (a) lr.d a0, 0(s0)
  (b) sw   t1, 4(s0)
  (c) sc.d t2, 0(s0)
  \end{verbbox}
  \theverbbox
  ~~~~~~
  \begin{verbbox}
  (a) lr.w a0, 0(s0)
  (b) sw   t1, 4(s0)
  (c) sc.w t2, 0(s0)
  \end{verbbox}
  \theverbbox
  ~~~~~~
  \begin{verbbox}
  (a) lr.w a0, 0(s0)
  (b) sw   t1, 4(s0)
  (c) sc.w t2, 8(s0)
  \end{verbbox}
  \theverbbox
%  \caption{Store-conditional (c) may succeed on some implementations}
  \caption{In all four (independent) code snippets, the store-conditional (c) is permitted but not guaranteed to succeed}
  \label{fig:litmus:lrsdsc}
\end{figure}

The atomicity axiom forbids from other harts from being interleaved in global memory order between an LR and the SC paired with that LR.
The atomicity axiom does not forbid loads from being interleaved between the paired operations in program order or in the global memory order, nor does it forbid stores from the same hart or stores to non-overlapping locations from appearing between the paired operations in either program order or in the global memory order.
For example, the SC instructions in Figure~\ref{fig:litmus:lrsdsc} may (but are not guaranteed to) succeed.
None of those successes would violate the atomicity axiom, because the intervening non-conditional stores are from the same hart as the paired load-reserved and store-conditional instructions.
This way, a memory system that tracks memory accesses at cache line granularity (and which therefore will see the four snippets of Figure~\ref{fig:litmus:lrsdsc} as identical) will not be forced to fail a store conditional instruction that happens to (falsely) share another portion of the same cache line as the memory location being held by the reservation.

The atomicity axiom also technically supports cases in which the LR and SC touch different addresses and/or use different access sizes; however, use cases for such behaviors are expected to be rare in practice.
Likewise, scenarios in which stores from the same hart between an LR/SC pair actually overlap the memory location(s) referenced by the LR or SC are expected to be rare compared to scenarios where the intervening store may simply fall onto the same cache line.

\subsection{\nameref*{rvwmo:ax:prog}}
\label{sec:progress}
\begin{tabular}{p{1cm}|p{12cm}} &
\nameref{rvwmo:ax:prog}: \progressaxiom
\end{tabular}

The progress axiom ensures a minimal forward progress guarantee.
It ensures that stores from one hart will eventually be made visible to other harts in the system in a finite amount of time, and that loads from other harts will eventually be able to read those values (or successors thereof).
Without this rule, it would be legal, for example, for a spinlock to spin infinitely on a value, even with a store from another hart waiting to unlock the spinlock.

The progress axiom is intended not to impose any other notion of fairness, latency, or quality of service onto the harts in a RISC-V implementation.
Any stronger notions of fairness are up to the rest of the ISA and/or up to the platform and/or device to define and implement.

The forward progress axiom will in almost all cases be naturally satisfied by any standard cache coherence protocol.
Implementations with non-coherent caches may have to provide some other mechanism to ensure the eventual visibility of all stores (or successors thereof) to all harts.

\subsection{Overlapping-Address Orderings (Rules~\ref{ppo:->st}--\ref{ppo:amoforward})}
\label{sec:memory:overlap}
\begin{tabular}{p{1cm}|p{12cm}}
  & Rule \ref{ppo:->st}: \ppost \\
  & Rule \ref{ppo:rdw}: \ppordw \\
  & Rule \ref{ppo:amoforward}: \ppoamoforward \\
\end{tabular}

Same-address orderings where the latter is a store are straightforward: a load or store can never be reordered with a later store to an overlapping memory location.  From a microarchitecture perspective, generally speaking, it is difficult or impossible to undo a speculatively reordered store if the speculation turns out to be invalid, so such behavior is simply disallowed by the model.
Same-address orderings from a store to a later load, on the other hand, do not need to be enforced.
As discussed in Section~\ref{sec:memory:loadvalueaxiom}, this reflects the observable behavior of implementations that forward values from buffered stores to later loads.

Same-address load-load ordering requirements are far more subtle.
The basic requirement is that a younger load must not return a value that is older than a value returned by an older load in the same hart to the same address.  This is often known as ``CoRR'' (Coherence for Read-Read pairs), or as part of a broader ``coherence'' or ``sequential consistency per location'' requirement.
Some architectures in the past have relaxed same-address load-load ordering, but in hindsight this is generally considered to complicate the programming model too much, and so RVWMO requires CoRR ordering to be enforced.
However, because the global memory order corresponds to the order in which loads perform rather than the ordering of the values being returned, capturing CoRR requirements in terms of the global memory order requires a bit of indirection.

\begin{figure}[h!]
  \center
  {
    \tt\small
    \begin{tabular}{cl||cl}
    \multicolumn{2}{c}{Hart 0} & \multicolumn{2}{c}{Hart 1} \\
    \hline
          & li t1, 1    &     & li~ t2, 2    \\
      (a) & sw t1,0(s0) & (d) & lw~ a0,0(s1) \\
      (b) & fence w, w  & (e) & sw~ t2,0(s1) \\
      (c) & sw t1,0(s1) & (f) & lw~ a1,0(s1) \\
          &             & (g) & xor t3,a1,a1 \\
          &             & (h) & add s0,s0,t3 \\
          &             & (i) & lw~ a2,0(s0) \\
    \end{tabular}
  }
  ~~~~
  \diagram
  \caption{Litmus test MP+fence.w.w+fri-rfi-addr}
  \label{fig:litmus:frirfi}
\end{figure}

Consider the litmus test of Figure~\ref{fig:litmus:frirfi}, which is one particular instance of the more general ``fri-rfi'' pattern.
The term ``fri-rfi'' refers to the sequence (d), (e), (f): (d) ``from-reads'' (i.e., reads from an earlier write than) (e) which is the same hart, and (f) reads from (e) which is in the same hart.

From a microarchitectural perspective, outcome {\tt a0=1}, {\tt a1=2}, {\tt a2=0} is legal (as are various other less subtle outcomes).  Intuitively, the following would produce the outcome in question:
\begin{itemize}
  \item (d) stalls (for whatever reason; perhaps it's stalled waiting for some other preceding instruction)
  \item (e) executes and enters the store buffer (but does not yet drain to memory)
  \item (f) executes and forwards from (e) in the store buffer
  \item (g), (h), and (i) execute
  \item (a) executes and drains to memory, (b) executes, and (c) executes and drains to memory
  \item (d) unstalls and executes
  \item (e) drains from the store buffer to memory
\end{itemize}
This corresponds to a global memory order of (f), (i), (a), (c), (d), (e).
Note that even though (f) performs before (d), the value returned by (f) is newer than the value returned by (d).
Therefore, this execution is legal and does not violate the CoRR requirements.

Likewise, if two back-to-back loads return the values written by the same store, then they may also appear out-of-order in the global memory order without violating CoRR.  Note that this is not the same as saying that the two loads return the same value, since two different stores may write the same value.

\begin{figure}[h!]
  \centering
  {
    \tt\small
    \begin{tabular}{cl||cl}
    \multicolumn{2}{c}{Hart 0} & \multicolumn{2}{c}{Hart 1} \\
    \hline
          & li t1, 1    & (d) & lw~ a0,0(s1) \\
      (a) & sw t1,0(s0) & (e) & xor t2,a0,a0 \\
      (b) & fence w, w  & (f) & add s2,s2,t2 \\
      (c) & sw t1,0(s1) & (g) & lw~ a1,0(s2) \\
          &             & (h) & lw~ a2,0(s2) \\
          &             & (i) & xor t3,a2,a2 \\
          &             & (j) & add s0,s0,t3 \\
          &             & (k) & lw~ a3,0(s0) \\
    \end{tabular}
  }
  ~~~~
  \diagram
  \caption{Litmus test RSW}
  \label{fig:litmus:rsw}
\end{figure}

Consider the litmus test of Figure~\ref{fig:litmus:rsw}.
The outcome {\tt a0=1}, {\tt a1=a2}, {\tt a3=0} can be observed by allowing (g) and (h) to be reordered.  This might be done speculatively, and the speculation can be justified by the microarchitecture (e.g., by snooping for cache invalidations and finding none) because replaying (h) after (g) would return the value written by the same store anyway.
Hence assuming {\tt a1} and {\tt a2} would end up with the same value written by the same store anyway, (g) and (h) can be legally reordered.
The global memory order corresponding to this execution would be (h),(k),(a),(c),(d),(g).

Executions of the test in Figure~\ref{fig:litmus:rsw} in which {\tt a1} does not equal {\tt a2} do in fact require that (g) appears before (h) in the global memory order.
Allowing (h) to appear before (g) in the global memory order would in that case result in a violation of CoRR, because then (h) would return an older value than that returned by (g).
Therefore, PPO rule~\ref{ppo:rdw} forbids this CoRR violation from occurring.
As such, PPO rule~\ref{ppo:rdw} strikes a careful balance between enforcing CoRR in all cases while simultaneously being weak enough to permit ``RSW'' and ``fri-rfi'' patterns that commonly appear in real microarchitectures.

There is one more overlapping-address rule: PPO rule~\ref{ppo:amoforward} simply states that a value cannot be returned from an AMO or SC to a subsequent load until the AMO or SC has (in the case of the SC, successfully) performed globally.
This follows somewhat naturally from the conceptual view that both AMOs and SC instructions are meant to be performed atomically in memory.
However, notably, PPO rule~\ref{ppo:amoforward} states that hardware may not even non-speculatively forward the value being stored by an AMOSWAP to a subsequent load, even though for AMOSWAP that store value is not actually semantically dependent on the previous value in memory, as is the case for the other AMOs.
The same holds true for SC store values that are not semantically dependent on the value returned by the paired LR.

The three PPO rules above also apply when the memory accesses in question only overlap partially.
This can occur, for example, when accesses of different sizes are used to access the same object.
Note also that the base addresses of two overlapping memory operations need not necessarily be the same for two memory accesses to overlap.
When misaligned memory accesses are being used, the overlapping-address PPO rules apply to each of the component memory accesses independently.

\begin{comment}
The formal model captures this as follows:
\begin{itemize}
  \item (a) precedes (b) in preserved program order because both are stores to the same address, and (b) is a store (Rule~\ref{ppo:->st}).  Therefore, (c) cannot return the value written by (a), because (b) is a later store to the same address in both program order and the global memory order, and so returning the value written by (a) would violate the load value axiom.
  \item (c) precedes (d) in preserved program order because both are accesses to the same address, and (d) is a store.  (c) also precedes (d) in program order.  Therefore, (c) is not able to return the value written by (d), because neither option in the load value axiom applies.
\end{itemize}
\end{comment}

\subsection{Fences (Rule~\ref{ppo:fence})}\label{sec:mm:fence}
\begin{tabular}{p{1cm}|p{12cm}} &
Rule \ref{ppo:fence}: \ppofence
\end{tabular}

By default, the FENCE instruction ensures that all memory accesses from instructions preceding the fence in program order (the ``predecessor set'') appear earlier in the global memory order than memory accesses from instructions appearing after the fence in program order (the ``successor set'').
However, fences can optionally further restrict the predecessor set and/or the successor set to  a smaller set of memory accesses in order to provide some speedup.
Specifically, fences have PR, PW, SR, and SW bits which restrict the predecessor and/or successor sets.
The predecessor set includes loads (resp.\@ stores) if and only if PR (resp.\@ PW) is set.
Similarly, the successor set includes loads (resp.\@ stores) if and only if SR (resp.\@ SW) is set.

The FENCE encoding currently has nine non-trivial combinations of the four bits PR, PW, SR, and SW, plus one extra encoding FENCE.TSO which facilitates mapping of ``acquire+release'' or RVTSO semantics.
The remaining seven combinations have empty predecessor and/or successor sets and hence are no-ops.
Of the ten non-trivial options, only six are commonly used in practice:
{\tt
\begin{itemize}
  \item fence rw,rw
  \item fence.tso \textrm{(i.e., a combined {\tt fence r,rw} $+$ {\tt fence rw,w})}
  \item fence rw,w
  \item fence r,rw
  \item fence r,r
  \item fence w,w
\end{itemize}
}
FENCE instructions using any other combination of PR, PW, SR, and SW are reserved.  We strongly recommend that programmers stick to these six.
Other combinations may have unknown or unexpected interactions with the memory model.

Finally, we note that since RISC-V uses a multi-copy atomic memory model, programmers can reason about fences bits in a thread-local manner.  There is no complex notion of ``fence cumulativity'' as found in memory models that are not multi-copy atomic.

\subsection{Acquire/Release Ordering (Rules~\ref{ppo:acquire}--\ref{ppo:rcsc})}\label{sec:acqrel}
\begin{tabular}{p{1cm}|p{12cm}}
  & Rule \ref{ppo:acquire}: \ppoacquire \\
  & Rule \ref{ppo:release}: \pporelease \\
  & Rule \ref{ppo:rcsc}: \pporcsc \\
\end{tabular}

An {\em acquire} operation, as would be used at the start of a critical section, requires all memory operations following the acquire in program order to also follow the acquire in the global memory order.
This ensures, for example, that all loads and stores inside the critical section are up to date with respect to the synchronization variable being used to protect it.
Acquire ordering can be enforced in one of two ways: with an acquire annotation, which enforces ordering with respect to just the synchronization variable itself, or with a {\tt fence r,rw}, which enforces ordering with respect to all previous loads.  

\begin{figure}[h!]
  \centering\small
  \begin{verbatim}
          sd           x1, (a1)     # Arbitrary unrelated store
          ld           x2, (a2)     # Arbitrary unrelated load
          li           t0, 1        # Initialize swap value.
      again:
          amoswap.w.aq t0, t0, (a0) # Attempt to acquire lock.
          bnez         t0, again    # Retry if held.
          # ...
          # Critical section.
          # ...
          amoswap.w.rl x0, x0, (a0) # Release lock by storing 0.
          sd           x3, (a3)     # Arbitrary unrelated store
          ld           x4, (a4)     # Arbitrary unrelated load
  \end{verbatim}
  \caption{A spinlock with atomics}
  \label{fig:litmus:spinlock_atomics}
\end{figure}

Consider Figure~\ref{fig:litmus:spinlock_atomics}.
Because this example uses {\em aq}, the loads and stores in the critical section are guaranteed to appear in the global memory order after the AMOSWAP used to acquire the lock.  However, assuming {\tt a0}, {\tt a1}, and {\tt a2} point to different memory locations, the loads and stores in the critical section may or may not appear after the ``Arbitrary unrelated load'' at the beginning of the example in the global memory order.

\begin{figure}[h!]
  \centering\small
  \begin{verbatim}
          sd           x1, (a1)     # Arbitrary unrelated store
          ld           x2, (a2)     # Arbitrary unrelated load
          li           t0, 1        # Initialize swap value.
      again:
          amoswap.w    t0, t0, (a0) # Attempt to acquire lock.
          fence        r, rw        # Enforce "acquire" memory ordering
          bnez         t0, again    # Retry if held.
          # ...
          # Critical section.
          # ...
          fence        rw, w        # Enforce "release" memory ordering
          amoswap.w    x0, x0, (a0) # Release lock by storing 0.
          sd           x3, (a3)     # Arbitrary unrelated store
          ld           x4, (a4)     # Arbitrary unrelated load
  \end{verbatim}
  \caption{A spinlock with fences}
  \label{fig:litmus:spinlock_fences}
\end{figure}

Now, consider the alternative in Figure~\ref{fig:litmus:spinlock_fences}.
In this case, even though the AMOSWAP does not enforce ordering with an {\em aq} bit, the fence nevertheless enforces that the acquire AMOSWAP appears earlier in the global memory order than all loads and stores in the critical section.
Note, however, that in this case, the fence also enforces additional orderings: it also requires that the ``Arbitrary unrelated load'' at the start of the program also appears earlier in the global memory order than the loads and stores of the critical section.  (This particular fence does not, however, enforce any ordering with respect to the ``Arbitrary unrelated store'' at the start of the snippet.)
In this way, fence-enforced orderings are slightly coarser than orderings enforced by {\em .aq}.

Release orderings work exactly the same as acquire orderings, just in the opposite direction.  Release semantics require all loads and stores preceding the release operation in program order to also precede the release operation in the global memory order.
This ensures, for example, that memory accesses in a critical section appear before the lock-releasing store in the global memory order.  Just as for acquire semantics, release semantics can be enforced using release annotations or with a {\tt fence rw,w} operations.  Using the same examples, the ordering between the loads and stores in the critical section and the ``Arbitrary unrelated store'' at the end of the code snippet is enforced only by the {\tt fence rw,w} in Figure~\ref{fig:litmus:spinlock_fences}, not by the {\em rl} in Figure~\ref{fig:litmus:spinlock_fences}.

With RCpc annotations alone, store-release-to-load-acquire ordering is not enforced.  This facilitates the porting of code written under the TSO and/or RCpc memory models.  
To enforce store-release-to-load-acquire ordering, the code must use store-release-RCsc and load-acquire-RCsc operations so that PPO rule \ref{ppo:rcsc} applies.
RCpc alone is sufficient for many uses cases in C/C++ but is insufficient for many other use cases in C/C++, Java, and Linux, to name just a few examples; see Section~\ref{sec:porting} for details.

Lastly, we note that just as with fences, programmers need not worry about ``cumulativity'' when analyzing ordering annotations.

\subsection{Syntactic Dependencies (Rules~\ref{ppo:addr}--\ref{ppo:ctrl})}
\label{sec:depspart1}
\begin{tabular}{p{1cm}|p{12cm}}
  & Rule \ref{ppo:addr}: \ppoaddr \\
  & Rule \ref{ppo:data}: \ppodata \\
  & Rule \ref{ppo:ctrl}: \ppoctrl \\
\end{tabular}

Dependencies from a load to a later memory operation in the same hart are respected by the RVWMO memory model.
The Alpha memory model was notable for choosing {\em not} to enforce the ordering of such dependencies, but most modern hardware and software memory models consider allowing dependent instructions to be reordered too confusing and counterintuitive.
Furthermore, modern code sometimes intentionally uses such dependencies as a particularly lightweight ordering enforcement mechanism.

The terms in Section~\ref{sec:deps} work as follows.
Instructions are said to carry dependencies from their source register(s) to their destination register(s) whenever the value written into each destination register is a function of the source register(s).
For most instructions, this means that the destination register(s) carry a dependency from all source register(s).
However, there are a few notable exceptions.
In the case of memory instructions, the value written into the destination register ultimately comes from the memory system rather than from the source register(s) directly, and so this breaks the chain of dependencies carried from the source register(s).
In the case of unconditional jumps, the value written into the destination register comes from the current {\tt pc} (which is never considered a source register by the memory model), and so likewise, JALR (the only jump with a source register) does not carry a dependency from {\em rs1} to {\em rd}.

\begin{verbbox}
(a) fadd  f3,f1,f2
(b) fadd  f6,f4,f5
(c) csrrs a0,fflags,x0
\end{verbbox}
\begin{figure}[h!]
  \centering\small
  \theverbbox
  \caption{(c) has a syntactic dependency on both (a) and (b) via {\tt fflags}, a destination register that both (a) and (b) implicitly accumulate into}
  \label{fig:litmus:fflags}
\end{figure}

The notion of accumulating into a destination register rather than writing into it reflects the behavior of CSRs such as {\tt fflags}.
In particular, an accumulation into a register does not clobber any previous writes or accumulations into the same register.
For example, in Figure~\ref{fig:litmus:fflags}, (c) has a syntactic dependency on both (a) and (b).

Like other modern memory models, the RVWMO memory model uses syntactic rather than semantic dependencies.
In other words, this definition depends on the identities of the
registers being accessed by different instructions, not the actual
contents of those registers.  This means that an address, control, or
data dependency must be enforced even if the calculation could seemingly
be ``optimized away''.
This choice ensures that RVWMO remains compatible with code that uses these false syntactic dependencies as a lightweight ordering mechanism.

\begin{verbbox}
ld  a1,0(s0)
xor a2,a1,a1
add s1,s1,a2
ld  a5,0(s1)
\end{verbbox}
\begin{figure}[h!]
  \centering\small
  \theverbbox
  \caption{A syntactic address dependency}
  \label{fig:litmus:address}
\end{figure}

For example, there is a syntactic address
dependency from the memory operation generated by the first instruction to the memory operation generated by the last instruction in
Figure~\ref{fig:litmus:address}, even though {\tt a1} XOR {\tt a1} is zero and
hence has no effect on the address accessed by the second load.

The benefit of using dependencies as a lightweight synchronization mechanism is that the ordering enforcement requirement is limited only to the specific two instructions in question.
Other non-dependent instructions may be freely-reordered by aggressive implementations.
One alternative would be to use a load-acquire, but this would enforce ordering for the first load with respect to {\em all} subsequent instructions.
Another would be to use a {\tt fence r,r}, but this would include all previous and all subsequent loads, making this option more expensive.

\begin{verbbox}
      lw  x1,0(x2)
      bne x1,x0,NEXT
      sw  x3,0(x4)
next: sw  x5,0(x6)
\end{verbbox}
\begin{figure}[h!]
  \centering\small
  \theverbbox
  \caption{A syntactic control dependency}
  \label{fig:litmus:control1}
\end{figure}

Control dependencies behave differently from address and data dependencies in the sense that a control dependency always extends to all instructions following the original target in program order.
Consider Figure~\ref{fig:litmus:control1}: the instruction at {\tt next} will always execute, but memory operation generated by that last instruction nevertheless still has control dependency from the memory operation generated by the first instruction.

\begin{verbbox}
        lw  x1,0(x2)
        bne x1,x0,NEXT
  next: sw  x3,0(x4)
\end{verbbox}
\begin{figure}[h!]
  \centering\small
  \theverbbox
  \caption{Another syntactic control dependency}
  \label{fig:litmus:control2}
\end{figure}

Likewise, consider Figure~\ref{fig:litmus:control2}.
Even though both branch outcomes have the same target, there is still a control dependency from the memory operation generated by the first instruction in this snippet to the memory operation generated by the last instruction.
This definition of control dependency is subtly stronger than what might be seen in other contexts (e.g., C++), but it conforms with standard definitions of control dependencies in the literature.

Notably, PPO rules \ref{ppo:addr}--\ref{ppo:ctrl} are also intentionally designed to respect dependencies that originate from the output of a successful store conditional instruction.
In general, an SC instruction will be followed by a branch checking whether the outcome was successful; this implies that there will be a control dependency from the store operation generated by the SC instruction to any memory operations following the branch.
PPO rule~\ref{ppo:ctrl} in turn implies that any subsequent store operations will appear later in the global memory order than the store operation generated by the SC.
However, a dependency cannot originate or be used to enforce ordering from the {\em rd} destination register of an unsuccessful SC instruction.

\begin{figure}[h!]
  \center
  {
    \tt\small
    \begin{tabular}{cl||cl}
    \multicolumn{2}{c}{Hart 0} & \multicolumn{2}{c}{Hart 1} \\
    \hline
      (a) & ld a0,0(s0)    & (e) & ld a3,0(s2) \\
      (b) & lr a1,0(s1)    & (f) & sd a3,0(s0) \\
      (c) & sc a2,a0,0(s1) &                    \\
      (d) & sd a2,0(s2)    &                    \\
    \end{tabular}
  }
  ~~~~
  \diagram
  \caption{A variant of the LB litmus test}
  \label{fig:litmus:successdeps}
\end{figure}

In addition, the choice to respect dependencies originating at store-conditional instructions ensures that certain out-of-thin-air-like behaviors will be prevented.
Consider Figure~\ref{fig:litmus:successdeps}.
Suppose a hypothetical implementation could occasionally make some early guarantee that a store-conditional operation will succeed.
In this case, (c) could return 0 to {\tt a2} early (before actually executing), allowing the sequence (d), (e), (f), (a), and then (b) to execute, and then (c) might execute (successfully) only at that point.
This would imply that (c) writes its own success value to {\tt 0(s1)}!
Fortunately, this situation and others like it are prevented by the fact that RVWMO respects dependencies originating at the stores generated by successful SC instructions.

\subsection{Pipeline Dependencies (Rules~\ref{ppo:addrdatarfi}--\ref{ppo:addrpo})}
\label{sec:ppopipeline}
\begin{tabular}{p{1cm}|p{12cm}}
  & Rule \ref{ppo:addrdatarfi}: \ppoaddrdatarfi \\
  & Rule \ref{ppo:addrpo}: \ppoaddrpo \\
%  & Rule \ref{ppo:ctrlcfence}: \ppoctrlcfence \\
%  & Rule \ref{ppo:addrpocfence}: \ppoaddrpocfence \\
\end{tabular}

\begin{figure}[h!]
  \centering
  {
    \tt\small
    \begin{tabular}{cl}
      (a) & lw a0, 0(s0)   \\
      (b) & sw a0, 0(s1)   \\
      (c) & lw a1, 0(s1)   \\
    \end{tabular}
  }
  ~~~~
  \diagram
  \caption{Because of the data dependency from (a) to (b), (a) also precedes (c)}
  \label{fig:litmus:addrdatarfi}
\end{figure}

PPO rules~\ref{ppo:addrdatarfi} and \ref{ppo:addrpo} reflect behaviors of almost all real processor pipeline implementations.
Rule~\ref{ppo:addrdatarfi} states that a load cannot forward from a store until the address and data for that store are known.
Consider Figure~\ref{fig:litmus:addrdatarfi}:
(c) cannot be executed until the data for (b) has been resolved, because (c) must return the value written by (b) (or by something even later in the global memory order), and the old value must not be clobbered by the writeback of (b) before (a) has had a chance to perform.
Therefore, (c) will never perform before (a) has performed.

\begin{figure}[h!]
  \centering
  {
    \tt\small
    \begin{tabular}{cl}
          & li t1, 1       \\
      (a) & lw a0, 0(s0)   \\
      (b) & sw a0, 0(s1)   \\
          & sw t1, 0(s1)   \\
      (c) & lw a1, 0(s1)   \\
    \end{tabular}
  }
  ~~~~
  \diagram
  \caption{Because of the extra store between (b) and (c), (a) no longer necessarily precedes (c)}
  \label{fig:litmus:addrdatarfi_no}
\end{figure}

If there were another store to the same address in between (b) and (c), as in Figure~\ref{fig:litmus:addrdatarfi_no}, then (c) would no longer be dependent on the data of (b) being resolved, and hence the dependency of (c) on (a), which produces the data for (b), would be broken.

Rule~\ref{ppo:addrpo} makes a similar observation to the previous rule: a store cannot be performed at memory until all previous loads that might access the same address have themselves been performed.
Such a load must appear to execute before the store, but it cannot do so if the store were to overwrite the value in memory before the load had a chance to read the old value.
Likewise, a store generally cannot be performed until it is known that preceding instructions will not cause an exception due to failed address resolution, and in this sense, rule~\ref{ppo:addrpo} can be seen as somewhat of a special case of rule~\ref{ppo:ctrl}.

\begin{figure}[h!]
  \centering
  {
    \tt\small
    \begin{tabular}{cl}
        & li t1, 1       \\
    (a) & lw a0, 0(s0)   \\
    (b) & lw a1, 0(a0)   \\
    (c) & sw t1, 0(s1)   \\
    \end{tabular}
  }
  ~~~~
  \diagram
  \caption{Because of the address dependency from (a) to (b), (a) also precedes (c)}
  \label{fig:litmus:addrpo}
\end{figure}

Consider Figure~\ref{fig:litmus:addrpo}:
(c) cannot be executed until the address for (b) is resolved, because it may turn out that the addresses match; i.e., that {\tt a0=s1}.  Therefore, (c) cannot be sent to memory before (a) has executed and confirmed whether the addresses to indeed overlap.


\section{Beyond Normal Memory}

RVWMO does not currently attempt to formally describe how FENCE.I, SFENCE.VMA, I/O fences, and PMAs behave.
All of these behaviors will be described by future formalizations.
In the meantime, the behavior of FENCE.I is described in Section~\ref{sec:fence}, the behavior of SFENCE.VMA is described in the RISC-V Instruction Set Privileged Architecture Manual, and the behavior of I/O fences and the effects of PMAs are described below.

\subsection{Coherence and Cacheability}

The RISC-V Privileged ISA defines Physical Memory Attributes (PMAs) which specify, among other things, whether portions of the address space are coherent and/or cacheable.
See the RISC-V Privileged ISA Specification for the complete details.
Here, we simply discuss how the various details in each PMA relate to the memory model:

\begin{itemize}
  \item Main memory vs.\@ I/O, and I/O memory ordering PMAs: the memory model as defined applies to main memory regions.  I/O ordering is discussed below.
  \item Supported access types and atomicity PMAs: the memory model is simply applied on top of whatever primitives each region supports.
  \item Cacheability PMAs: the cacheability PMAs in general do not affect the memory model.  Non-cacheable regions may have more restrictive behavior than cacheable regions, but the set of allowed behaviors does not change regardless.  However, some platform-specific and/or device-specific cacheability settings may differ.
  \item Coherence PMAs: The memory consistency model for memory regions marked as non-coherent in PMAs is currently platform-specific and/or device-specific: the load-value axiom, the atomicity axiom, and the progress axiom all may be violated with non-coherent memory.  Note however that coherent memory does not require a hardware cache coherence protocol.  The RISC-V Privileged ISA Specification suggests that hardware-incoherent regions of main memory are discouraged, but the memory model is compatible with hardware coherence, software coherence, implicit coherence due to read-only memory, implicit coherence due to only one agent having access, or otherwise.
  \item Idempotency PMAs: Idempotency PMAs are used to specify memory regions for which loads and/or stores may have side effects, and this in turn is used by the microarchitecture to determine, e.g., whether prefetches are legal.  This distinction does not affect the memory model.
\end{itemize}


\subsection{I/O Ordering}

For I/O, the load value axiom and atomicity axiom in general do not apply, as both reads and writes might have device-specific side effects and may return values other than the value ``written'' by the most recent store to the same address.
Nevertheless, the following preserved program order rules still generally apply for accesses to I/O memory:
memory access $a$ precedes memory access $b$ in global memory order if $a$ precedes $b$ in program order and one or more of the following holds:
\begin{enumerate}
  \item $a$ precedes $b$ in preserved program order as defined in Section~\ref{sec:memorymodel}, with the exception that acquire and release ordering annotations apply only from one memory operation to another memory operation and from one I/O operation to another I/O operation, but not from a memory operation to an I/O nor vice versa
  \item $a$ and $b$ are accesses to overlapping addresses in an I/O region
  \item $a$ and $b$ are accesses to the same strongly-ordered I/O region
  \item $a$ and $b$ are accesses to I/O regions, and the channel associated with the I/O region accessed by either $a$ or $b$ is channel 1
  \item $a$ and $b$ are accesses to I/O regions associated with the same channel (except for channel 0)
\end{enumerate}

Note that the FENCE instruction distinguishes between memory operations and I/O operations in its predecessor and successor sets.
To enforce ordering between I/O operations and memory operations, code must use a FENCE with PI, PO, SI, and/or SO, plus PR, PW, SR, and/or SW.
For example, to enforce ordering between a write to normal memory and an MMIO write to a device register, a {\tt fence w,o} or stronger is needed.

\begin{verbbox}
  sd t0, 0(a0)
  fence w,o
  sd a0, 0(a1)
\end{verbbox}
\begin{figure}[h!]
  \centering\small
  \theverbbox
  \caption{Ordering memory and I/O accesses}
  \label{fig:litmus:wo}
\end{figure}

When a fence is in fact used, implementations must assume that the device may attempt to access memory immediately after receiving the MMIO signal, and subsequent memory accesses from that device to memory must observe the effects of all accesses ordered prior to that MMIO operation.
In other words, in Figure~\ref{fig:litmus:wo}, suppose {\tt 0(a0)} is in normal memory and {\tt 0(a1)} is the address of a device register in I/O memory.
If the device accesses {\tt 0(a0)} upon receiving the MMIO write, then that load must conceptually appear after the first store to {\tt 0(a0)} according to the rules of the RVWMO memory model.
In some implementations, the only way to ensure this will be to require that the first store does in fact complete before the MMIO write is issued.
Other implementations may find ways to be more aggressive, while others still may not need to do anything different at all for I/O and normal memory accesses.
Nevertheless, the RVWMO memory model does not distinguish between these options; it simply provides an implementation-agnostic mechanism to specify the orderings that must be enforced.

Many architectures include separate notions of ``ordering'' and ``completion'' fences, especially as it relates to I/O (as opposed to normal memory).
Ordering fences simply ensure that memory operations stay in order, while completion fences ensure that predecessor accesses have all completed before any successors are made visible.
RISC-V does not explicitly distinguish between ordering and completion fences.
Instead, this distinction is simply inferred from different uses of the FENCE bits.

For implementations that conform to the RISC-V Unix Platform Specification, I/O devices and DMA operations are required to access memory coherently and via strongly-ordered I/O channels.
Therefore, accesses to normal memory regions that are shared with I/O devices can also use the standard synchronization mechanisms.
Implementations that do not conform to the Unix Platform Specification and/or in which devices do not access memory coherently will need to use mechanisms (which are currently platform-specific or device-specific) to enforce coherency.

I/O regions in the address space should be considered non-cacheable regions in the PMAs for those regions.  Such regions can be considered coherent by the PMA if they are not cached by any agent.

The ordering guarantees in this section may not apply beyond a platform-specific boundary between the RISC-V cores and the device.  In particular, I/O accesses sent across an external bus (e.g., PCIe) may be reordered before they reach their ultimate destination.  Ordering must be enforced in such situations according to the platform-specific rules of those external devices and buses.

\section{Code Porting and Mapping Guidelines}
\label{sec:porting}

\begin{table}[h!]
  \centering
  \begin{tabular}{|l|l|}
    \hline
    x86/TSO Operation & RVWMO Mapping \\
    \hline
    \hline
    Load              & \tt l\{b|h|w|d\}; fence r,rw               \\
    \hline
    Store             & \tt fence rw,w; s\{b|h|w|d\}               \\
    \hline
    Atomic RMW        & \tt amo<op>.\{w|d\}.aqrl \textrm{OR} lr.\{w|d\}.aq; <op>; sc.\{w|d\}.aqrl \\
    \hline
    Fence             & \tt fence rw,rw \\
    \hline
  \end{tabular}
  \caption{Mappings from TSO operations to RISC-V operations}
  \label{tab:tsomappings}
\end{table}

Table~\ref{tab:tsomappings} provides a mapping from TSO memory operations onto RISC-V memory instructions.
Normal x86 loads and stores are all inherently acquire-RCpc and release-RCpc operations: TSO enforces all load-load, load-store, and store-store ordering by default.
Therefore, under RVWMO, all TSO loads must be mapped onto {\tt l\{b|h|w|d\}; fence r,rw}, and all TSO stores must either be mapped onto {\tt amoswap.\{w|d\}.rl~x0} or onto {\tt fence rw,w; s\{b|h|w|d\}}.
TSO atomic read-modify-writes and x86 instructions using the LOCK prefix are fully-ordered and can be implemented either with {\tt amo<op>.aqrl} or {\tt lr.aq; <op>; sc.aqrl}.
In the latter case, the {\em rl} annotation on the LR turns out (for non-obvious reasons) to be redundant and can be omitted.

\begin{table}[h!]
  \centering
  \begin{tabular}{|l|l|}
    \hline
    Power Operation & RVWMO Mapping \\
    \hline
    \hline
    Load              & \tt l\{b|h|w|d\}  \\
    \hline
    Load-Reserve      & \tt lr.\{w|d\}  \\
    \hline
    Store             & \tt s\{b|h|w|d\}  \\
    \hline
    Store-Conditional & \tt sc.\{w|d\}  \\
    \hline
    \tt lwsync        & \tt fence.tso \\
    \hline
    \tt sync          & \tt fence rw,rw \\
    \hline
    \tt isync         & \tt fence.i; fence r,r \\
    \hline
  \end{tabular}
  \caption{Mappings from Power operations to RISC-V operations}
  \label{tab:powermappings}
\end{table}

Table~\ref{tab:powermappings} provides a mapping from Power memory operations onto RISC-V memory instructions.
Power {\tt isync} maps to RISC-V {\tt fence.i; fence r,r}; the latter fence is needed because {\tt isync} is used to define a ``control+control fence'' dependency that is not present in RVWMO.

\begin{table}[h!]
  \centering
  \begin{tabular}{|l|l|}
    \hline
    ARM Operation             & RVWMO Mapping \\
    \hline
    \hline
    Load                      & \tt l\{b|h|w|d\}  \\
    \hline
    Load-Acquire              & \tt l\{b|h|w|d\}; fence r,rw  \\
    \hline
    Load-Exclusive            & \tt lr.\{w|d\}  \\
    \hline
    Load-Acquire-Exclusive    & \tt lr.\{w|d\}.aq \\
    \hline
    Store                     & \tt s\{b|h|w|d\}  \\
    \hline
    Store-Release             & \tt fence rw,w; s\{b|h|w|d\}  \\
    \hline
    Store-Exclusive           & \tt sc.\{w|d\}  \\
    \hline
    Store-Release-Exclusive   & \tt sc.\{w|d\}.rl  \\
    \hline
    \tt dmb                   & \tt fence rw,rw \\
    \hline
    \tt dmb.ld                & \tt fence r,rw \\
    \hline
    \tt dmb.st                & \tt fence w,w \\
    \hline
    \tt isb                   & \tt fence.i; fence r,r \\
    \hline
  \end{tabular}
  \caption{Mappings from ARM operations to RISC-V operations}
  \label{tab:armmappings}
\end{table}

Table~\ref{tab:armmappings} provides a mapping from ARM memory operations onto RISC-V memory instructions.
Since RISC-V does not currently have {\tt l\{b|h|w|d\}.aq[rl]} or  {\tt s\{b|h|w|d\}.[aq]rl} opcodes, ARM load-acquire and store-release operations should be mapped onto {\tt l\{b|h|w|d\}; fence r,rw} and {\tt fence rw,w; s\{b|h|w|d\}}, respectively.
ARM load-reserved and store-conditional instructions can likewise map onto their RISC-V LR and SC equivalents.
ARM {\tt isb} maps to RISC-V {\tt fence.i;~fence~r,r} similarly to how {\tt isync} maps for Power.

\begin{table}[h!]
  \centering
  \begin{tabular}{|l|l|}
    \hline
    Linux Operation           & RVWMO Mapping \\
    \hline
    \hline
    \tt smp\_mb()             & \tt fence rw,rw \\
    \hline
    \tt smp\_rmb()            & \tt fence r,r \\
    \hline
    \tt smp\_wmb()            & \tt fence w,w \\
    \hline
    \tt dma\_rmb()            & \tt fence r,r \\
    \hline
    \tt dma\_wmb()            & \tt fence w,w \\
    \hline
    \tt mb()                  & \tt fence iorw,iorw \\
    \hline
    \tt rmb()                 & \tt fence ri,ri \\
    \hline
    \tt wmb()                 & \tt fence wo,wo \\
    \hline
    \tt smp\_load\_acquire()   & \tt l\{b|h|w|d\}; fence r,rw \\
    \hline
    \tt smp\_store\_release()  & \tt fence.tso; s\{b|h|w|d\}  \\
    \hline
  \end{tabular}

  ~

  \begin{tabular}{|l|l|l|}
    \hline
    Linux Construct            & RVWMO AMO Mapping       & RVWMO LR/SC Mapping\\
    \hline
    \tt atomic\_<op>\_relaxed  & \tt amo<op>.\{w|d\}      & \tt lr.\{w|d\}; <op>; sc.\{w|d\} \\
    \hline
    \tt atomic\_<op>\_acquire  & \tt amo<op>.\{w|d\}.aq   & \tt lr.\{w|d\}.aq; <op>; sc.\{w|d\} \\
    \hline
    \tt atomic\_<op>\_release  & \tt amo<op>.\{w|d\}.rl   & \tt lr.\{w|d\}; <op>; sc.\{w|d\}.aqrl$^*$ \\
    \hline
    \tt atomic\_<op>           & \tt amo<op>.\{w|d\}.aqrl & \tt lr.\{w|d\}.aq; <op>; sc.\{w|d\}.aqrl \\
    \hline
    \multicolumn{3}{p{0.9\textwidth}}{$^*$an alternative mapping of ``{\tt fence.tso;~lr.\{w|d\};~<op>;~sc.\{w|d\}}'' uses more instructions but may have higher performance on some systems, as FENCE.TSO is lighterweight than the implicit full fence in {\tt sc.\{w|d\}.aqrl}.}
  \end{tabular}
  \caption{Mappings from Linux memory primitives to RISC-V primitives.  Other constructs (such as spinlocks) should follow accordingly.  Platforms or devices with non-coherent DMA may need additional synchronization (such as cache flush or invalidate mechanisms); currently any such extra synchronization will be device-specific.}
  \label{tab:linuxmappings}
\end{table}

Table~\ref{tab:linuxmappings} provides a mapping of Linux memory ordering macros onto RISC-V memory instructions.
%The now-deprecated fence {\tt smp\_read\_barrier\_depends()} map to a no-op due to preserved program order rules \ref{ppo:addr}--\ref{ppo:ctrl}.
The Linux fences {\tt dma\_rmb()} and {\tt dma\_wmb()} map onto {\tt fence r,r} and {\tt fence w,w}, respectively, since the RISC-V Unix Platform requires coherent DMA, but would be mapped onto {\tt fence ri,ri} and {\tt fence wo,wo}, respectively, on a platform with non-coherent DMA.
Platforms with non-coherent DMA may also require a mechanism by which cache lines can be flushed and/or invalidated.
Such mechanisms will be device-specific and/or standardized in a future extension to the ISA.

The Linux mappings for release operations may seem stronger than necessary, but these mappings are needed to cover some cases in which Linux requires stronger orderings than the more intuitive mappings would provide.
In particular, as of the time this text is being written, Linux is actively debating whether to require load-load, load-store, and store-store orderings between accesses in one critical section and accesses in a subsequent critical section in the same hart and protected by the same synchronization object.
Not all combinations of {\tt fence rw,w}/{\tt fence r,rw} mappings with {\em aq}/{\em rl} mappings combine to provide such orderings.
There are a few ways around this problem, including:
\begin{enumerate}
  \item Always use {\tt fence rw,w}/{\tt fence r,rw}, and never use {\em aq}/{\em rl}.  This suffices but is undesirable, as it defeats the purpose of the {\em aq}/{\em rl} modifiers.
  \item Always use {\em aq}/{\em rl}, and never use {\tt fence rw,w}/{\tt fence r,rw}.  This does not currently work due to the lack of {\tt l\{b|h|w|d\}.aq[rl]} and {\tt s\{b|h|w|d\}.[aq]rl} opcodes.
  \item Strengthen the mappings of release operations such that they would enforce sufficient orderings in the presence of either type of acquire mapping.  This is the currently-recommended solution, and the one shown in Table~\ref{tab:linuxmappings}.
\end{enumerate}

\begin{figure}[h!]
  \centering\small
  \begin{verbbox}
Linux code:
(a)  int r0 = *x;
(bc) spin_unlock(y, 0);
     ...
     ...
(d)  spin_lock(y);
(e)  int r1 = *z;
  \end{verbbox}
  \theverbbox
  ~~~~~~~~~~
  \begin{verbbox}
RVWMO Mapping:
(a) lw           a0, 0(s0)
(b) fence.tso  // vs. fence rw,w
(c) sd           x0,0(s1)
    ...
(d) amoswap.d.aq a1,t1,0(s1)
(e) lw           a2,0(s2)
  \end{verbbox}
  \theverbbox
  \caption{Orderings between critical sections in Linux}
  \label{fig:litmus:lkmm_ll}
\end{figure}

For example, the critical section ordering rule currently being debated by the Linux community would require (a) to be ordered before (e) in Figure~\ref{fig:litmus:lkmm_ll}.
If that will indeed be required, then it would be insufficient for (b) to map as {\tt fence rw,w}.
That said, these mappings are subject to change as the Linux Kernel Memory Model evolves.

\begin{table}[h!]
  \centering
  \begin{tabular}{|l|l|}
    \hline
    C/C++ Construct                            & RVWMO Mapping \\
    \hline
    \hline
    Non-atomic load                            & \tt l\{b|h|w|d\}               \\
    \hline
    \tt atomic\_load(memory\_order\_relaxed)   & \tt l\{b|h|w|d\}               \\
    \hline
    %\tt atomic\_load(memory\_order\_consume)   & \multicolumn{2}{l|}{\tt l\{b|h|w|d\}; fence r,rw}   \\
    %\hline
    \tt atomic\_load(memory\_order\_acquire)   & \tt l\{b|h|w|d\}; fence r,rw    \\
    \hline
    \tt atomic\_load(memory\_order\_seq\_cst)  & \tt fence rw,rw; l\{b|h|w|d\}; fence r,rw       \\
    \hline
    \hline
    Non-atomic store                           & \tt s\{b|h|w|d\}               \\
    \hline
    \tt atomic\_store(memory\_order\_relaxed)  & \tt s\{b|h|w|d\}               \\
    \hline
    \tt atomic\_store(memory\_order\_release)  & \tt fence rw,w; s\{b|h|w|d\}  \\
    \hline
    \tt atomic\_store(memory\_order\_seq\_cst) & \tt fence rw,rw; s\{b|h|w|d\} \\
    \hline
    \hline
    \tt atomic\_thread\_fence(memory\_order\_acquire)  & \tt fence r,rw \\
    \hline
    \tt atomic\_thread\_fence(memory\_order\_release)  & \tt fence rw,w \\
    \hline
    \tt atomic\_thread\_fence(memory\_order\_acq\_rel) & {\tt fence.tso} \\
    \hline
    \tt atomic\_thread\_fence(memory\_order\_seq\_cst) & \tt fence rw,rw \\
    \hline
  \end{tabular}

~

  \footnotesize\centering
  \begin{tabular}{|l|l|l|}
    \hline
    C/C++ Construct                               & RVWMO AMO Mapping       & RVWMO LR/SC Mapping\\
    \hline
    \tt atomic\_<op>(memory\_order\_relaxed)  & \tt amo<op>.\{w|d\}      & \tt lr.\{w|d\}; <op>; sc.\{w|d\} \\
    \hline
    \tt atomic\_<op>(memory\_order\_acquire)  & \tt amo<op>.\{w|d\}.aq   & \tt lr.\{w|d\}.aq; <op>; sc.\{w|d\} \\
    \hline
    \tt atomic\_<op>(memory\_order\_release)  & \tt amo<op>.\{w|d\}.rl   & \tt lr.\{w|d\}; <op>; sc.\{w|d\}.rl \\
    \hline
    \tt atomic\_<op>(memory\_order\_acq\_rel) & \tt amo<op>.\{w|d\}.aqrl & \tt lr.\{w|d\}.aq; <op>; sc.\{w|d\}.rl \\
    \hline
    \tt atomic\_<op>(memory\_order\_seq\_cst) & \tt amo<op>.\{w|d\}.aqrl & \tt lr.\{w|d\}.aqrl; <op>; sc.\{w|d\}.rl \\
    \hline
  \end{tabular}
  \caption{Mappings from C/C++ primitives to RISC-V primitives.}
  \label{tab:c11mappings}
\end{table}

\begin{table}[h!]
  \centering
  \begin{tabular}{|l|l|}
    \hline
    C/C++ Construct                            & RVWMO Mapping \\
    \hline
    \hline
    Non-atomic load                            & \tt l\{b|h|w|d\}               \\
    \hline
    \tt atomic\_load(memory\_order\_relaxed)   & \tt l\{b|h|w|d\}               \\
    \hline
    \tt atomic\_load(memory\_order\_acquire)   & \tt l\{b|h|w|d\}.aq  \\
    \hline
    \tt atomic\_load(memory\_order\_seq\_cst)  & \tt l\{b|h|w|d\}.aq  \\
    \hline
    \hline
    Non-atomic store                           & \tt s\{b|h|w|d\}               \\
    \hline
    \tt atomic\_store(memory\_order\_relaxed)  & \tt s\{b|h|w|d\}               \\
    \hline
    \tt atomic\_store(memory\_order\_release)  & \tt s\{b|h|w|d\}.rl  \\
    \hline
    \tt atomic\_store(memory\_order\_seq\_cst) & \tt s\{b|h|w|d\}.rl \\
    \hline
    \hline
    \tt atomic\_thread\_fence(memory\_order\_acquire)  & \tt fence r,rw \\
    \hline
    \tt atomic\_thread\_fence(memory\_order\_release)  & \tt fence rw,w \\
    \hline
    \tt atomic\_thread\_fence(memory\_order\_acq\_rel) & {\tt fence.tso} \\
    \hline
    \tt atomic\_thread\_fence(memory\_order\_seq\_cst) & \tt fence rw,rw \\
    \hline
  \end{tabular}

~

  \footnotesize
  \begin{tabular}{|l|l|l|}
    \hline
    C/C++ Construct                               & RVWMO AMO Mapping       & RVWMO LR/SC Mapping\\
    \hline
    \tt atomic\_<op>(memory\_order\_relaxed)  & \tt amo<op>.\{w|d\}      & \tt lr.\{w|d\}; <op>; sc.\{w|d\} \\
    \hline
    \tt atomic\_<op>(memory\_order\_acquire)  & \tt amo<op>.\{w|d\}.aq   & \tt lr.\{w|d\}.aq; <op>; sc.\{w|d\} \\
    \hline
    \tt atomic\_<op>(memory\_order\_release)  & \tt amo<op>.\{w|d\}.rl   & \tt lr.\{w|d\}; <op>; sc.\{w|d\}.rl \\
    \hline
    \tt atomic\_<op>(memory\_order\_acq\_rel) & \tt amo<op>.\{w|d\}.aqrl & \tt lr.\{w|d\}.aq; <op>; sc.\{w|d\}.rl \\
    \hline
    \tt atomic\_<op>(memory\_order\_seq\_cst) & \tt amo<op>.\{w|d\}.aqrl & \tt lr.\{w|d\}.aq$^*$; <op>; sc.\{w|d\}.rl \\
    \hline
    \multicolumn{3}{l}{$^*$must be {\tt lr.\{w|d\}.aqrl} in order to interoperate with code mapped per Table~\ref{tab:c11mappings}}
  \end{tabular}
  \caption{Hypothetical mappings from C/C++ primitives to RISC-V primitives, if native load-acquire and store-release opcodes are introduced.}
  \label{tab:c11mappings_hypothetical}
\end{table}

Table~\ref{tab:c11mappings} provides a mapping of C11/C++11 atomic operations onto RISC-V memory instructions.
If {\tt l\{b|h|w|d\}.aq[rl]} and {\tt s\{b|h|w|d\}.[aq]rl} opcodes are introduced, then the mappings in Figure~\ref{tab:c11mappings_hypothetical} will suffice.
Note however that the two mappings only interoperate correctly when the original LR/SC mapping of {\tt atomic\_<op>(memory\_order\_seq\_cst)} onto {\tt lr.\{w|d\}.aqrl;~<op>;~sc.\{w|d\}.rl} is used.

%The acquire-RCsc annotation can be emulated via fences by following the relevant AMO or LR/SC pair with a {\tt fence r,rw}.
%The release-RCsc annotation can be emulated via fences by preceding the relevant AMO or LR/SC pair with a {\tt fence rw,w}.
%A combined acquire-RCsc and release-RCsc annotation can be emulated via fences by both preceding the relevant AMO or LR/SC pair with a {\tt fence rw,w} and following the relevant AMO or LR/SC pair with a {\tt fence rw,rw}.

Any AMO can be emulated by an LR/SC pair, but care must be taken to ensure that any PPO orderings that originate from the LR are also made to originate from the SC, and that any PPO orderings that terminate at the SC are also made to terminate at the LR.
For example, the LR must also be made to respect any data dependencies that the AMO has, given that load operations do not otherwise have any notion of a data dependency.
Likewise, the effect a {\tt fence r,r} elsewhere in the same hart must also be made to apply to the SC, which would not otherwise respect that fence.
The emulator may achieve this effect by simply mapping AMOs onto {\tt lr.aq;~<op>;~sc.aqrl}, matching the mapping used elsewhere for fully-ordered atomics.

\section{Implementation Guidelines}

The RVWMO and RVTSO memory models by no means preclude microarchitectures from employing sophisticated speculation techniques or other forms of optimization in order to deliver higher performance.
The models also do not impose any requirement to use any one particular cache hierarchy, nor even to use a cache coherence protocol at all.
Instead, these models only specify the behaviors that can be exposed to software.
Microarchitectures are free to use any pipeline design, any coherent or non-coherent cache hierarchy, any on-chip interconnect, etc., as long as the design only admits executions that satisfy the memory model rules.
That said, to help people understand the actual implementations of the memory model, in this section we provide some guidelines on how architects and programmers should interpret the models' rules.

Both RVWMO and RVTSO are multi-copy atomic (or ``other-multi-copy-atomic''): any store value that is visible to a hart other than the one that originally issued it must also be conceptually visible to all other harts in the system.
In other words, harts may forward from their own previous stores before those stores have become globally visible to all harts, but no early inter-hart forwarding is permitted.
Multi-copy atomicity may be enforced in a number of ways.
It might hold inherently due to the physical design of the caches and store buffers, it may be enforced via a single-writer/multiple-reader cache coherence protocol, or it might hold due to some other mechanism.

Although multi-copy atomicity does impose some restrictions on the microarchitecture, it is one of the key properties keeping the memory model from becoming extremely complicated.
For example, a hart may not legally forward a value from a neighbor hart's private store buffer (unless of course it is done in such a way that no new illegal behaviors become architecturally visible).
Nor may a cache coherence protocol forward a value from one hart to another until the coherence protocol has invalidated all older copies from other caches.
Of course, microarchitectures may (and high-performance implementations likely will) violate these rules under the covers through speculation or other optimizations, as long as any non-compliant behaviors are not exposed to the programmer.

As a rough guideline for interpreting the PPO rules in RVWMO, we expect the following from the software perspective:
\begin{itemize}
  \item programmers will use PPO rules \ref{ppo:->st} and \ref{ppo:fence}--\ref{ppo:rcsc} regularly and actively.
  \item expert programmers will use PPO rules \ref{ppo:addr}--\ref{ppo:ctrl} to speed up critical paths of important data structures.
  \item even expert programmers will rarely if ever use PPO rules \ref{ppo:rdw}--\ref{ppo:amoforward} and \ref{ppo:addrdatarfi}--\ref{ppo:addrpo} directly.  These are included to facilitate common microarchitectural optimizations (rule~\ref{ppo:rdw}) and the operational formal modeling approach (rules \ref{ppo:amoforward} and \ref{ppo:addrdatarfi}--\ref{ppo:addrpo}) described in Section~\ref{sec:operational}.  They also facilitate the process of porting code from other architectures that have similar rules.
\end{itemize}

We also expect the following from the hardware perspective:
\begin{itemize}
  \item PPO rules \ref{ppo:->st} and \ref{ppo:amoforward}--\ref{ppo:release} reflect well-understood rules that should pose few surprises to architects.
  \item PPO rule \ref{ppo:rdw} reflects a natural and common hardware optimization, but one that is very subtle and hence is worth double checking carefully.
  \item PPO rule \ref{ppo:rcsc} may not be immediately obvious to architects, but it is a standard memory model requirement
  \item The load value axiom, the atomicity axiom, and PPO rules \ref{ppo:addr}--\ref{ppo:addrpo} reflect rules that most hardware implementations will enforce naturally, unless they contain extreme optimizations.  Of course, implementations should make sure to double check these rules nevertheless.  Hardware must also ensure that syntactic dependencies are not ``optimized away''.
\end{itemize}

Architectures are free to implement any of the memory model rules as conservatively as they choose.  For example, a hardware implementation may choose to do any or all of the following:
  \begin{itemize}
    \item interpret all fences as if they were {\tt fence rw,rw} (or {\tt fence iorw,iorw}, if I/O is involved), regardless of the bits actually set
    \item implement all fences with PW and SR as if they were {\tt fence~rw,rw} (or {\tt fence~iorw,iorw}, if I/O is involved), as ``{\tt w,r}'' is the most expensive of the four possible normal memory orderings anyway
    \item emulate {\em aq} and {\em rl} as described in Section~\ref{sec:porting}
    \item enforcing all same-address load-load ordering, even in the presence of patterns such as ``fri-rfi'' and ``RSW''
    \item forbid any forwarding of a value from a store in the store buffer to a subsequent AMO or LR to the same address
    \item forbid any forwarding of a value from an AMO or SC in the store buffer to a subsequent load to the same address
    \item implement TSO on all memory accesses, and ignore any normal memory fences that do not include ``{\tt w,r}'' ordering (e.g., as Ztso implementations will do)
    \item implement all atomics to be RCsc or even fully-ordered, regardless of annotation
  \end{itemize}

Architectures that implement RVTSO can safely do the following:
\begin{itemize}
  \item Ignore all fences that do not have both PW and SR (unless the fence also orders I/O)
  \item Ignore all PPO rules except for rules \ref{ppo:fence} and \ref{ppo:rcsc}, since the rest are redundant with other PPO rules under RVTSO assumptions
\end{itemize}

Other general notes:

\begin{itemize}
  \item Silent stores (i.e., stores that write the same value that already exists at a memory location) behave like any other store from a memory model point of view.  Microarchitectures that attempt to implement silent stores must take care to ensure that the memory model is still obeyed, particularly in cases such as RSW (Section~\ref{sec:memory:overlap}) which tend to be incompatible with silent stores.
  \item Writes may be merged (i.e., two consecutive writes to the same address may be merged) or subsumed (i.e., the earlier of two back-to-back writes to the same address may be elided) as long as the resulting behavior does not otherwise violate the memory model semantics.
\end{itemize}

The question of write subsumption can be understood from the following example:
\begin{figure}[h!]
  \centering
  {
    \tt\small
    \begin{tabular}{cl||cl}
    \multicolumn{2}{c}{Hart 0} & \multicolumn{2}{c}{Hart 1} \\
    \hline
        & li t1, 3    &     & li  t3, 2    \\
        & li t2, 1    &     &              \\
    (a) & sw t1,0(s0) & (d) & lw  a0,0(s1) \\
    (b) & fence w, w  & (e) & sw  a0,0(s0) \\
    (c) & sw t2,0(s1) & (f) & sw  t3,0(s0) \\
    \end{tabular}
  }
  ~~~~
  \diagram
  \caption{Write subsumption litmus test}
  \label{fig:litmus:subsumption}
\end{figure}

As written, (a) must precede (f) in the global memory order:
\begin{itemize}
  \item (a) precedes (c) in the global memory order because of rule 2
  \item (c) precedes (d) in the global memory order because of the Load Value axiom
  \item (d) precedes (e) in the global memory order because of rule 7
  \item (e) precedes (f) in the global memory order because of rule 1
\end{itemize}

A very aggressive microarchitecture might erroneously decide to discard (e), as (f) supersedes it, and this may in turn lead the microarchitecture to break the now-eliminated dependency between (d) and (f) (and hence also between (a) and (f)).
This would violate the memory model rules, and hence it is forbidden.
Write subsumption may in other cases be legal, if for example there were no data dependency between (d) and (e).

\subsection{Possible Future Extensions}

We expect that any or all of the following possible future extensions would be compatible with the RVWMO memory model:

\begin{itemize}
  \item `V' vector ISA extensions
  \item A transactional memory subset of the `T' ISA extension
  \item `J' JIT extension
  \item Native encodings for {\tt l\{b|h|w|d\}.aq}/{\tt s\{b|h|w|d\}.rl}
  \item Fences limited to certain addresses
  \item Cache writeback/flush/invalidate/etc.\@ instructions
\end{itemize}

\chapter{Formal Memory Model Specifications}

To facilitate formal analysis of RVWMO, this chapter presents a set of formalizations using different tools and modeling approaches.  Any discrepancies are unintended; the expectation is that the models describe exactly the same sets of legal behaviors.

\clearpage
\section{Formal Axiomatic Specification in Alloy}
\label{sec:alloy}

\lstdefinelanguage{alloy}{
  morekeywords={abstract, sig, extends, pred, fun, fact, no, set, one, lone, let, not, all, iden, some, run, for},
  morecomment=[l]{//},
  morecomment=[s]{/*}{*/},
  commentstyle=\color{green!40!black},
  keywordstyle=\color{blue!40!black},
  moredelim=**[is][\color{red}]{@}{@},
  escapeinside={!}{!},
}
\lstset{language=alloy}
\lstset{aboveskip=0pt}
\lstset{belowskip=0pt}

We present a formal specification of the RVWMO memory model in Alloy (\url{http://alloy.mit.edu}).
This model is available online at \url{https://github.com/daniellustig/riscv-memory-model}.

The online material also contains some litmus tests and some examples of how Alloy can be used to model check some of the mappings in Section~\ref{sec:porting}.

\begin{figure}[h!]
  {
  \tt\bfseries\centering\footnotesize
  \begin{lstlisting}
////////////////////////////////////////////////////////////////////////////////
// =RVWMO PPO=

// Preserved Program Order
fun ppo : Event->Event {
  // same-address ordering
  po_loc :> Store
  + rdw
  + (AMO + StoreConditional) <: rfi

  // explicit synchronization
  + ppo_fence
  + Acquire <: ^po :> MemoryEvent
  + MemoryEvent <: ^po :> Release
  + RCsc <: ^po :> RCsc

  // syntactic dependencies
  + addrdep
  + datadep
  + ctrldep :> Store

  // pipeline dependencies
  + (addrdep+datadep).rfi
  + addrdep.^po :> Store
}

// the global memory order respects preserved program order
fact { ppo in ^gmo }
\end{lstlisting}}
  \caption{The RVWMO memory model formalized in Alloy (1/5: PPO)}
  \label{fig:alloy1}
\end{figure}
\begin{figure}[h!]
  {
  \tt\bfseries\centering\footnotesize
  \begin{lstlisting}
////////////////////////////////////////////////////////////////////////////////
// =RVWMO axioms=

// Load Value Axiom
fun candidates[r: MemoryEvent] : set MemoryEvent {
  (r.~^gmo & Store & same_addr[r]) // writes preceding r in gmo
  + (r.^~po & Store & same_addr[r]) // writes preceding r in po
}

fun latest_among[s: set Event] : Event { s - s.~^gmo }

pred LoadValue {
  all w: Store | all r: Load |
    w->r in rf <=> w = latest_among[candidates[r]]
}

// Atomicity Axiom
pred Atomicity {
  all r: Store.~pair |            // starting from the lr,
    no x: Store & same_addr[r] |  // there is no store x to the same addr
      x not in same_hart[r]       // such that x is from a different hart,
      and x in r.~rf.^gmo         // x follows (the store r reads from) in gmo,
      and r.pair in x.^gmo        // and r follows x in gmo
}

// Progress Axiom - implicit

pred RISCV_mm { LoadValue and Atomicity /* and Progress */ }

\end{lstlisting}}
  \caption{The RVWMO memory model formalized in Alloy (2/5: Axioms)}
  \label{fig:alloy2}
\end{figure}
\begin{figure}[h!]
  {
  \tt\bfseries\centering\footnotesize
  \begin{lstlisting}
////////////////////////////////////////////////////////////////////////////////
// Basic model of memory

sig Hart {  // hardware thread
  start : one Event
}
sig Address {}
abstract sig Event {
  po: lone Event // program order
}

// .aq and .rl represent the opcode bits.  The ordering annotations assigned
// to those bits are described further down.
abstract sig MemoryEvent extends Event {
  address: one Address,
  acquireRCpc: lone MemoryEvent,
  acquireRCsc: lone MemoryEvent,
  releaseRCpc: lone MemoryEvent,
  releaseRCsc: lone MemoryEvent,
  addrdep: set MemoryEvent,
  ctrldep: set Event,
  datadep: set MemoryEvent,
  gmo: set MemoryEvent,  // global memory order
  rf: set MemoryEvent
}
sig LoadNormal extends MemoryEvent {} // l{b|h|w|d}
sig LoadReserve extends MemoryEvent { // lr
  pair: lone StoreConditional
}
sig StoreNormal extends MemoryEvent {}       // s{b|h|w|d}
// all StoreConditionals in the model are assumed to be successful
sig StoreConditional extends MemoryEvent {}  // sc
sig AMO extends MemoryEvent {}               // amo
sig NOP extends Event {}

fun Load : Event { LoadNormal + LoadReserve + AMO }
fun Store : Event { StoreNormal + StoreConditional + AMO }

sig Fence extends Event {
  pr: lone Fence, // opcode bit
  pw: lone Fence, // opcode bit
  sr: lone Fence, // opcode bit
  sw: lone Fence  // opcode bit
}
sig FenceTSO extends Fence {}

/* Alloy encoding detail: opcode bits are either set (encoded, e.g.,
 * as f.pr in iden) or unset (f.pr not in iden).  The bits cannot be used for
 * anything else */
fact { pr + pw + sr + sw in iden }
// likewise for ordering annotations
fact { acquireRCpc + acquireRCsc + releaseRCpc + releaseRCsc in iden }
// don't try to encode FenceTSO via pr/pw/sr/sw; just use it as-is
fact { no FenceTSO.(pr + pw + sr + sw) }
\end{lstlisting}}
  \caption{The RVWMO memory model formalized in Alloy (3/5: model of memory)}
  \label{fig:alloy3}
\end{figure}

\begin{figure}[h!]
  {
  \tt\bfseries\centering\footnotesize
  \begin{lstlisting}
////////////////////////////////////////////////////////////////////////////////
// =Basic model rules=

// Ordering annotation groups
fun Acquire : MemoryEvent { MemoryEvent.acquireRCpc + MemoryEvent.acquireRCsc }
fun Release : MemoryEvent { MemoryEvent.releaseRCpc + MemoryEvent.releaseRCsc }
fun RCpc : MemoryEvent { MemoryEvent.acquireRCpc + MemoryEvent.releaseRCpc }
fun RCsc : MemoryEvent { MemoryEvent.acquireRCsc + MemoryEvent.releaseRCsc }

// There is no such thing as store-acquire or load-release, unless it's both
fact { Load & Release in Acquire }
fact { Store & Acquire in Release }

// FENCE PPO
fun FencePRSR : Fence { Fence.(pr & sr) }
fun FencePRSW : Fence { Fence.(pr & sw) }
fun FencePWSR : Fence { Fence.(pw & sr) }
fun FencePWSW : Fence { Fence.(pw & sw) }

fun ppo_fence : MemoryEvent->MemoryEvent {
    (Load  <: ^po :> FencePRSR).(^po :> Load)
  + (Load  <: ^po :> FencePRSW).(^po :> Store)
  + (Store <: ^po :> FencePWSR).(^po :> Load)
  + (Store <: ^po :> FencePWSW).(^po :> Store)
  + (Load  <: ^po :> FenceTSO) .(^po :> MemoryEvent)
  + (Store <: ^po :> FenceTSO) .(^po :> Store)
}

// auxiliary definitions
fun po_loc : Event->Event { ^po & address.~address }
fun same_hart[e: Event] : set Event { e + e.^~po + e.^po }
fun same_addr[e: Event] : set Event { e.address.~address }

// initial stores
fun NonInit : set Event { Hart.start.*po }
fun Init : set Event { Event - NonInit }
fact { Init in StoreNormal }
fact { Init->(MemoryEvent & NonInit) in ^gmo }
fact { all e: NonInit | one e.*~po.~start }  // each event is in exactly one hart
fact { all a: Address | one Init & a.~address } // one init store per address
fact { no Init <: po and no po :> Init }
\end{lstlisting}}
  \caption{The RVWMO memory model formalized in Alloy (4/5: Basic model rules)}
  \label{fig:alloy4}
\end{figure}

\begin{figure}[h!]
  {
  \tt\bfseries\centering\footnotesize
  \begin{lstlisting}
// po
fact { acyclic[po] }

// gmo
fact { total[^gmo, MemoryEvent] } // gmo is a total order over all MemoryEvents

//rf
fact { rf.~rf in iden } // each read returns the value of only one write
fact { rf in Store <: address.~address :> Load }
fun rfi : MemoryEvent->MemoryEvent { rf & (*po + *~po) }

//dep
fact { no StoreNormal <: (addrdep + ctrldep + datadep) }
fact { addrdep + ctrldep + datadep + pair in ^po }
fact { pair in datadep }
fact { datadep in datadep :> Store }
fact { ctrldep.*po in ctrldep }
fact { no pair & (^po :> (LoadReserve + StoreConditional)).^po }
fact { StoreConditional in LoadReserve.pair } // assume all SCs succeed

// rdw
fun rdw : Event->Event {
  (Load <: po_loc :> Load)  // start with all same_address load-load pairs,
  - (~rf.rf)                // subtract pairs that read from the same store,
  - (po_loc.rfi)            // and subtract out "fri-rfi" patterns
}

// filter out redundant instances and/or visualizations
fact { no gmo & gmo.gmo } // keep the visualization uncluttered
fact { all a: Address | some a.~address }

////////////////////////////////////////////////////////////////////////////////
// =Optional: opcode encoding restrictions=

// the list of blessed fences
fact { Fence in
  Fence.pr.sr
  + Fence.pw.sw
  + Fence.pr.pw.sw
  + Fence.pr.sr.sw
  + FenceTSO
  + Fence.pr.pw.sr.sw
}

pred restrict_to_current_encodings {
  no (LoadNormal + StoreNormal) & (Acquire + Release)
}

////////////////////////////////////////////////////////////////////////////////
// =Alloy shortcuts=
pred acyclic[rel: Event->Event] { no iden & ^rel }
pred total[rel: Event->Event, bag: Event] {
  all disj e, e': bag | e->e' in rel + ~rel
  acyclic[rel]
}
\end{lstlisting}}
  \caption{The RVWMO memory model formalized in Alloy (5/5: Auxiliaries)}
  \label{fig:alloy5}
\end{figure}



\clearpage
\section{Formal Axiomatic Specification in Herd}

The tool \textsf{herd} takes a memory model and a litmus test as input and simulates the execution of the test on top of the memory model. Memory models are written in the domain specific language \textsc{Cat}. This section provides two \textsc{Cat} memory model of RVWMO. The first model, Figure~\ref{fig:herd2}, follows the \emph{global memory order}, Section~\ref{sec:memorymodel}, definition of~RVWMO, as much as is possible for a \textsc{Cat} model. The second model, Figure~\ref{fig:herd3}, is an equivalent, more efficient, partial order based RVWMO model.

The simulator~\textsf{herd} is part of the \textsf{diy} tool suite --- see \url{http://diy.inria.fr} for software and documentation. The models and more are available online at~\url{http://diy.inria.fr/cats7/riscv/}.

\begin{figure}[h!]
  {
  \tt\bfseries\centering\footnotesize
  \begin{lstlisting}
(*************)
(* Utilities *)
(*************)

(* All fence relations *)
let fence.r.r = [R];fencerel(Fence.r.r);[R]
let fence.r.w = [R];fencerel(Fence.r.w);[W]
let fence.r.rw = [R];fencerel(Fence.r.rw);[M]
let fence.w.r = [W];fencerel(Fence.w.r);[R]
let fence.w.w = [W];fencerel(Fence.w.w);[W]
let fence.w.rw = [W];fencerel(Fence.w.rw);[M]
let fence.rw.r = [M];fencerel(Fence.rw.r);[R]
let fence.rw.w = [M];fencerel(Fence.rw.w);[W]
let fence.rw.rw = [M];fencerel(Fence.rw.rw);[M]
let fence = 
  fence.r.r | fence.r.w | fence.r.rw |
  fence.w.r | fence.w.w | fence.w.rw |
  fence.rw.r | fence.rw.w | fence.rw.rw

(* Same address, no W to the same address in-between *)
let po-loc-no-w = po-loc \ (po-loc?;[W];po-loc)
(* Read same write *)
let rsw = rf^-1;rf
(* Acquire, or stronger  *)
let AQ = Acq|AcqRel
(* Release or stronger *)
and RL = RelAcqRel
(* All RCsc *)
let RCsc = Acq|Rel|AcqRel
(* Amo events are both R and W, rmw relate paired lr/sc *)
let AMO = R & W
let StCond = range(rmw)
(* Add "artificial" data dependencies, ie paired lr/sc *)
let data = data | rmw

(*************)
(* ppo rules *)
(*************)

(* Overlapping-Address Orderings *)
let r1 = [M];po-loc;[W]
and r2 = ([R];po-loc-no-w;[R]) \ rsw
and r3 = [AMO|StCond];rfi;[R]
(* Explicit Synchronization *)
and r4 = fence
and r5 = [AQ];po;[M]
and r6 = [M];po;[RL]
and r7 = [RCsc];po;[RCsc]
(* Syntactic Dependencies *)
and r8 = [M];addr;[M]
and r9 = [M];data;[W]
and r10 = [M];ctrl;[W]
(* Pipeline Dependencies *)
and r11 = [R];(addr|data);[W];rfi;[R]
and r12 = [R];addr;[M];po;[W]

let ppo = r1 | r2 | r3 | r4 | r5 | r6 | r7 | r8 | r9 | r10 | r11 | r12
\end{lstlisting}
  }
  \caption{{\tt riscv-defs.cat}, a herd definition of preserved program order (1/3)}
  \label{fig:herd1}
\end{figure}

\begin{figure}[ht!]
  {
  \tt\bfseries\centering\footnotesize
  \begin{lstlisting}
Total

(* Notice that herd has defined its own rf relation *)

(* Define ppo *)
include "riscv-defs.cat"

(********************************)
(* Generate global memory order *)
(********************************)

let gmo0 = (* precursor: ie build gmo as an total order that include gmo0 *)
  loc & (W\FW) * FW | # Final write after any write to the same location
  ppo |               # ppo compatible
  rfe                 # includes herd external rf (optimisation)

(* Walk over all linear extensions of gmo0 *)
with  gmo from linearisations(M\IW,gmo0)

(* Add initial writes upfront -- convenient for computing rfGMO *)
let gmo = gmo | loc & IW * (M\IW)

(**********)
(* Axioms *)
(**********)

(* Compute rf according to the load value axiom, aka rfGMO *)
let WR = loc & ([W];(gmo|po);[R])
let rfGMO = WR \ (loc&([W];gmo);WR)

(* Check equality of herd rf and of rfGMO *)
empty (rf\rfGMO)|(rfGMO\rf) as RfCons

(* Atomicity axiom *)
let infloc = (gmo & loc)^-1
let inflocext = infloc & ext
let winside  = (infloc;rmw;inflocext) & (infloc;rf;rmw;inflocext) & [W]
empty winside as Atomic
\end{lstlisting}
  }
  \caption{{\tt riscv.cat}, a herd version of the RVWMO memory model (2/3)}
  \label{fig:herd2}
\end{figure}

\begin{figure}[h!]
  {
  \tt\bfseries\centering\footnotesize
  \begin{lstlisting}
Partial

(***************)
(* Definitions *)
(***************)

(* Define ppo *)
include "riscv-defs.cat"

(* Compute coherence relation *)
include "cos-opt.cat"

(**********)
(* Axioms *)
(**********)

(* Sc per location *)
acyclic co|rf|fr|po-loc as Coherence

(* Main model axiom *)
acyclic co|rfe|fr|ppo as Model

(* Atomicity axiom *)
empty rmw & (fre;coe) as Atomic
\end{lstlisting}
  }
  \caption{{\tt riscv.cat}, an alternative herd presentation of the RVWMO memory model (3/3)}
  \label{fig:herd3}
\end{figure}

%% Operational Memory Model
\clearpage
\subsection{An Operational Memory Model}
%The formal model is defined in
%executable higher-order logic, written in the \texttt{lem} specification language.

This is an alternative exposition of the RVWMO.
\fixme{give some incentives for having another model: microarchitecture-like, more intuitive to some, executable, incremental...}
The operational model is expressed as a state machine, with states that are an abstract representation of hardware machine states.

An interactive version of the model, together with a library of litmus tests,
is provided online: \url{http://www.cl.cam.ac.uk/~pes20/rmem}

In this subsection, {\em load} means any instruction that can perform a memory load operation (including AMOs), and {\em store} means any instruction that can perform a memory store operation (including AMOs).
% {\em load-acquire} means any load that has {\tt .aq} set (including load-acquire-RCsc), and {\em store-release} means any store that has {\tt .rl} set (including store-release-RCsc).

\paragraph{Model states}
A model state consists of a shared memory and a tuple of thread model states.
% \begin{center}
%   \begin{tikzpicture} [node distance=1cm, font=\sffamily, >=latex]
%     \node[draw,text width=5cm,align=center] (mem) {Shared Memory};
%
%     \node[draw,text width=2cm,align=center] (thread 1) [above=of mem.north west, anchor=south west] {Thread 0};
%     \node[draw,text width=2cm,align=center] (thread n) [above=of mem.north east, anchor=south east] {Thread n};
%
%     \node[font=\bf] (thread dots) at ($(thread 1.east)!0.5!(thread n.west)$) [anchor=center] {\dots};
%     \node[font=\bf] (arrow dots) [node distance=7mm, below=of thread dots, anchor=center] {\dots};
%
%     \draw[->] ($(thread 1.south west)!0.33!(thread 1.south east)$) -- ++(-90:1cm);
%     \draw[<-] ($(thread 1.south west)!0.67!(thread 1.south east)$) -- ++(-90:1cm);
%
%     \draw[->] ($(thread n.south west)!0.33!(thread n.south east)$) -- ++(-90:1cm);
%     \draw[<-] ($(thread n.south west)!0.67!(thread n.south east)$) -- ++(-90:1cm);
%   \end{tikzpicture}
% \end{center}
The shared memory state records the most recent memory store operation to each location.
To handle atomic memory accesses ({\em  lr}, {\em sc} and AMOs), the memory is extended with a map (the atomics map) from load requests to sets of store slices, that associates a load request of an atomic load with the store slices it reads from (excluding stores that have been forwarded to the load and have not reached memory yet).

Each thread model state consists principally of a tree of instruction instances, some of which have been finished, and some of which have not.
Non-finished instruction instances can be subject to restart, e.g.~if they depend on an out-of-order or speculative load that turns out to be unsound.
The load part of AMOs can be marked as finished before the entire instruction is finished; when such instruction is restarted it is rolled back to the state it was in when the load part was marked as finished.
Conditional branch and indirect jump instructions may have multiple successors in the instructions tree.
When such instruction is finished, any un-taken alternative paths are discarded.

Each instruction instance state includes an execution state of the instruction's ISA pseudocode, which one can think of as a representation of the pseudocode control state, pseudocode call stack, and local variable values. \fixme{change ``pseudocode'' to something more RISC-V appropriate}
%
An instruction instance state also includes information, detailed below, about the instruction instance's memory and register footprints, its register and memory reads and writes, whether it is finished, etc.

\paragraph{Model transitions}
For any state, the model defines the set of allowed transitions, each of which is a single atomic step to a new abstract machine state.
Each transition arises from a single instruction instance; it will change the state of that instance, and it may depend on or change the rest of its thread state and the shared memory state.
% Instructions cannot be treated
% as atomic units: complete execution of a single instruction instance may
% involve many transitions, which can be interleaved with those of other
% instances in the same or other threads, and some of this is programmer-visible.
The transitions are introduced below and defined in Section~\ref{sec:omm:thread_trans}, with a precondition and a construction of the post-transition model state for each.

\noindent Transitions for all instructions:
\begin{itemize}
\item \nameref{omm:thread:fetch}: This transition represents a fetch and decode of a new instruction instance, as a program-order successor of a previously fetched instruction instance, or at the initial fetch address for a thread.
\item[$\circ$] \nameref{omm:thread:reg_read}: This is a read of a register value from the most recent program-order predecessor instruction instance that writes to that register.
\item[$\circ$] \nameref{omm:thread:reg_write}
\item[$\circ$] \nameref{omm:thread:sail_interp}: This covers pseudocode internal computation, function calls, etc.
\item[$\circ$] \nameref{omm:thread:finish}: At this point the instruction pseudocode is done, the instruction cannot be restarted or discarded, and all memory effects have taken place. For a conditional branches and indirect jump any non-taken program order successor branches are discarded.
\end{itemize}

\noindent Load instructions:
\begin{itemize}
\item[$\circ$] \nameref{omm:thread:initiate_mem_read}: At this point the memory footprint of the load is provisionally known and its individual memory load operations can start being satisfied.
\item \nameref{omm:thread:sat_by_forwarding}: This partially or entirely satisfies a single memory load operation by forwarding from its program order previous writes.
\item \nameref{omm:thread:sat_from_mem}: This entirely satisfies the outstanding slices of a single memory load operation, from memory.
\item[$\circ$] \nameref{omm:thread:complete_load}: At this point all the memory load operations of the load have been entirely satisfied and the instruction pseudocode can continue executing.
A load instruction can be subject to being restarted until the \nameref{omm:thread:finish} transition or \nameref{omm:thread:finish_load_part}.
But, under some conditions, the model might treat a load instruction as non-restartable if the \nameref{restart_condition} does not indicate the load can be restarted.
\end{itemize}

\noindent Store instructions:
\begin{itemize}
\item[$\circ$] \nameref{omm:thread:announce_mem_write_footprint}: At this point the memory footprint of the store is provisionally known.
\item[$\circ$] \nameref{omm:thread:initiate_mem_write}: At this point the memory store operations have their values and program order subsequent memory load operations can be satisfied by forwarding from them.
\item[$\circ$] \nameref{omm:thread:commit_store}: At this point the store is guaranteed to happen (it cannot be restarted or discarded), and the memory store operations can start being propagated to memory.
\item \nameref{omm:thread:prop_mem_write}: This propagates a single memory store operation to memory.
\item[$\circ$] \nameref{omm:thread:complete_store}: At this point all memory store operations have been propagated to memory, and the instruction pseudocode can continue executing.
\end{itemize}

\noindent {\em  sc.w/d} instructions:
\begin{itemize}
\item \nameref{omm:thread:excl_success}: This commits to the success of the {\em sc.w/d}.
\item \nameref{omm:thread:excl_fail}: This commits to the failure of the {\em sc.w/d}.
\end{itemize}

\noindent AMO instructions:
\begin{itemize}
\item[$\circ$] \nameref{omm:thread:finish_load_part}: At this point the load part of an AMO instruction is done, and the load part cannot be restarted or discarded. If the AMO instruction is restarted after the transition is taken, the instruction rolls back to its state right after this transition.
\end{itemize}

\noindent Fence instructions:
\begin{itemize}
\item[$\circ$] \nameref{omm:thread:commit_barrier}
\end{itemize}

\begin{commentary}
The transitions labelled~$\circ$ can always be taken eagerly, as soon as they are enabled, without excluding other behaviour; the $\bullet$ cannot.
\end{commentary}

\subsubsection{Intra-instruction Pseudocode Execution}
The intra-instruction semantics for each instruction instance is expressed as a state machine, essentially running the instruction pseudocode.
Given a pseudocode execution state, it computes the next state, as one of the following:

\begin{center}
\begin{tabular}{ll}
{\sc Load\_mem}($load\_kind$, $address$, $size$, $load\_continuation$)
    & Load request\\
{\sc Atomic\_res}($res\_continuation$)
    & {\em sc.w/d} result\\
{\sc Store\_ea}($store\_kind$, $address$, $size$, $next\_state$)
    & Store effective address\\
{\sc Store\_memv}($memory\_value$, $store\_continuation$)
    & Store value\\
{\sc Fence}($fence\_kind$, $next\_state$)
    & Fence\\
{\sc Read\_reg}($reg\_name$, $read\_continuation$)
    & Register memory load operation\\
{\sc Write\_reg}($reg\_name$, $register\_value$, $next\_state$)
    & Register write\\
{\sc Internal}($next\_state$)
    & Pseudocode internal step\\
{\sc Done}
    & End of pseudocode\\
\end{tabular}
\end{center}
Here memory values are lists of bytes, addresses are 64-bit \fixme{what about 32-bit?} numbers, load and store kinds identify whether they are regular, atomic, acquire, acquire-RCsc, release and/or release-RCsc operations \fixme{<- these should use the proper names}, register names identify a register and slice thereof (start and end bit indices), and the continuations describe how the instruction instance will continue for each value that might be provided by the surrounding memory model.
Stores are split into two steps, {\sc Store\_ea} and {\sc Store\_memv}.
We ensure these are paired in the pseudocode, but there may be other steps between them: it is observable that the {\sc Store\_ea} can occur before the value to be written is determined, because the potential memory footprint of the instruction becomes provisionally known then.

The pseudocode of each instruction performs at most one store, load, or fence, except for AMOs that preform exactly one load and one store.
Those memory accesses are then split apart into the architecturally atomic units by the thread semantics (see \nameref{omm:thread:initiate_mem_read} and \nameref{omm:thread:announce_mem_write_footprint} below).

Informally, each bit of a register read should be satisfied from a register write by the most recent (in program order) instruction instance that can write that bit, or from the thread's initial register state if there is no such.
That instance may not have executed its register write yet, in which case the register read should block.
Hence, it is essential to know the register write footprint of each instruction instance, which we calculate when the instruction instance is created (see the action of \nameref{omm:thread:fetch} below).
We ensure in the pseudocode that each instruction does at most one register write to each register bit, and also that it does not try to read a register value it just wrote.

Data-flow dependencies in the model emerge from the fact that register read has to wait for the appropriate register write to be executed (as described above).

\subsubsection{Instruction Instance States}\label{sec:omm:inst_state}
Each instruction instance $i$ has a state comprising:
\begin{itemize}
\item $program\_loc$, the memory address from which the instruction was fetched;
\item $instruction\_kind$, identifying whether this is a load, store, AMO, or fence instruction, each with the associated kind; or a conditional branch; or a `simple' instruction;
\item $regs\_in$, the set of input $reg\_name$s, as statically determined from the opcode;
\item $regs\_out$, the output $reg\_name$s, as statically determined from the opcode;
\item $pseudocode\_state$ (or sometimes just `state' for short), one of
  \begin{itemize}
  \item {\sc Plain} $next\_state$, ready to make a pseudocode transition;
  \item {\sc Pending\_mem\_loads} $load\_cont$, performing the memory load operation(s) of a load instruction; or
  \item {\sc Pending\_mem\_stores} $store\_cont$, performing the memory store operation(s) of a store instruction;
  \item {\sc Pending\_exception} $exception$, performing an exception;
  \end{itemize}
\item $reg\_reads$, the accumulated register reads, including their sources and values, of this instance's execution so far;
\item $reg\_writes$, the accumulated register writes, including dependency information to identify the register reads and memory load operations (by this instruction) that might have affected each;
\item $mem\_loads$, a set of memory load operations.
Each operation includes a memory footprint (an address and size) and, if the operation has already been satisfied, the set of store slices (each consisting of a memory store operation and a set of its byte indices) that satisfied it.
\item $mem\_stores$, a set of memory store operations.
Each operation includes a memory footprint and, when available, the memory value to be stored.
In addition, each operation has a flag that indicates whether it has been propagated (passed to the memory) or not.
\item $successful\_atomic$, for {\em sc.w/d}, indicates whether the instruction is committed to succeed or fail, or no commitment has been made yet; for AMOs this will always be set to true (committed to succeed).
\item information recording whether the instance is committed, finished, etc.
\end{itemize}

Memory load operations include their load kind and their memory footprint (their address and size), the as-yet-unsatisfied slices (the byte indices that have not been satisfied), and, for the satisfied slices, information about the memory store operation(s) that they were satisfied from.
%
Memory store operations include their store kind, their memory footprint, and their value.
When we refer to a memory store or load operation without mentioning the kind we mean the operation can be of any kind.
%
A load instruction which has initiated (so its load operation list $mem\_loads$ is not empty) and for which all its load operations are satisfied (i.e.~there are no unsatisfied slices) is said to be
\emph{entirely satisfied}.
%
A {\em lr.w/d} instruction instance is called {\em successful} if it is paired with a {\em sc.w/d} and that {\em sc.w/d} was committed to succeed.
If a successful {\em lr.w/d} has a memory load operation that is mapped, in the atomics map, to a memory store operation slice $ws$, we say the {\em lr.w/d} has an outstanding lock on $ws$.


\subsubsection{Thread States}
The model state of a single hardware thread includes:
\begin{itemize}
\item $thread\_id$, a unique identifier of the thread;
\item $register\_data$, the name, bit width, and start bit index for each register;
\item $initial\_register\_state$, the initial register value for each register;
\item $initial\_fetch\_address$, the initial fetch address for this thread;
\item $instruction\_tree$, a tree of the instruction instances that have been fetched (and not discarded), in program order.
\end{itemize}
\fixme{the above has more details than needed; only $thread\_id$ and $instruction\_tree$ are needed}


\subsubsection{Model Transitions}\label{sec:omm:thread_trans}
\fixme{say something about how we describe transitions?}

\paragraph{Fetch instruction}\label{omm:thread:fetch}
A possible program-order successor of instruction instance $i$ can be fetched from address $loc$ if:
\begin{enumerate}
\item it has not already been fetched, i.e., none of the immediate successors of $i$ in the thread's $instruction\_tree$ are from $loc$; and
\item $loc$ is a possible next fetch address for $i$:
  \begin{enumerate}
  \item for a conditional branch, either the successor address or the branch target address;
  \item for indirect jump instruction ({\em jalr}), when the target address is not yet determined, any address; and
  \item for any other instruction, the value written to the program counter register ({\em pc}); \fixme{<- this is tide to our Sail model}
  \end{enumerate}
\end{enumerate}

\begin{commentary}
The possible next fetch addresses are available immediately after fetching $i$ and the model does not need to wait for the pseudocode to write to {\em pc} (as is the case with GPRs), this allows out of order execution, and speculation past conditional branches and jumps.
For most instructions these addresses are easily attained from a static analysis of the instruction pseudocode.
The only exception to that is the indirect jump instruction ({\em jalr}).
%
The exhaustive search in the {\tt rmem} tool handles this instruction by running the exhaustive search multiple times with a growing set of possible next fetch addresses for each indirect jump.
The initial search uses empty sets, hence there is no fetch after indirect jump instruction until the pseudocode of the instruction writes to {\em pc}, and then we use that value for fetching the next instruction.
Before starting the next iteration of exhaustive search, we collect for each indirect jump (grouped by code location) the set of values it wrote to {\em pc} in all the executions in the previous search iteration, and use that as possible next fetch addresses of the instruction.
This process terminates when it reaches a fixed-point (no new fetch addresses are detected).
\end{commentary}

Action: construct a freshly initialized instruction instance $i'$ for the instruction in the program memory at $loc$, with state ``{\sc Plain} $next\_state$'', including the static information available from the pseudocode such as its $instruction\_kind$, $regs\_in$, and $regs\_out$, and add $i'$ to the thread's $instruction\_tree$ as a successor of $i$. If the instruction fails to decode, set the state of $i'$ to ``{\sc Pending\_exception} $exception$'' with $exception$ that describes the decoding error.

\begin{commentary}
This involves only the thread, not the storage subsystem, as we assume a fixed program rather than modelling fetches with memory reads; we do not model self-modifying code.
\end{commentary}


\paragraph{Initiate memory load operations}\label{omm:thread:initiate_mem_read}
An instruction instance $i$ with next pseudocode state {\sc Load\_mem}($load\_kind$, $address$, $size$, $load\_cont$) can initiate the corresponding memory reads if:
\begin{enumerate}
\item all program order previous {\em fence} instructions with {\em .sr} set are finished;
\item all program order previous {\em fence.i} instructions are finished; \fixme{it was decided that fence.i does not have any memory model effects; remove this}
\item if $i$ is a load-acquire-RCsc, all program order previous store-releases-RCsc are finished; and
\item all non-finished program order previous load-acquire instructions are entirely satisfied.
\end{enumerate}
Action:
\begin{enumerate}
\item Construct the appropriate memory load operations $mlos$:
  \begin{itemize}
  \item if $address$ is aligned to $size$ then $mlos$ is a single memory load operation of $size$ bytes from $address$;
  \item otherwise, $mlos$ is a set of $size$ memory load operations, each of one byte, from the addresses $address\ldots address+size-1$.
  \end{itemize}
\item set $i.mem\_loads$ to $mlos$; and
\item update the state of $i$ to ``{\sc Pending\_mem\_loads} $load\_cont$''.
\end{enumerate}


\paragraph{Satisfy memory load operation by forwarding from stores}\label{omm:thread:sat_by_forwarding}
For a load instruction instance $i$ in state ``{\sc Pending\_mem\_loads} $load\_cont$'', and a memory load operation, $mlo$ in $i.mem\_loads$ that has unsatisfied slices, the memory load operation can be partially or entirely satisfied by forwarding from unpropagated memory store operations by store instruction instances that are program order before $i$.

Let $msoss$ be the maximal set of unpropagated memory store operation slices from store instruction instances that are program order before $i$ (if $i$ is a load-acquire, exclude {\em sc.w/d} and AMO memory store operations \fixme{?}), that overlap with the unsatisfied slices of $mlo$, and which are not superseded by intervening stores that are either propagated or read from by this thread.
The last condition requires, for each memory store operation slice $msos$ in $msoss$ from instruction $i'$:
\begin{itemize}
\item that there is no store instruction program order between $i$ and $i'$ with a memory store operation overlapping $msos$; and
\item that there is no load instruction program order between $i$ and $i'$ that was satisfied from an overlapping memory store operation slice from a different thread.
\end{itemize}
Action:
\begin{enumerate}
\item update $mlo$ to indicate that it was satisfied by $msoss$; and
\item restart any speculative instructions which have violated coherence as a result of this, i.e., for every non-finished instruction $i'$ that is a program order successor of $i$, and every memory load operation $mlo'$ of $i'$ that was satisfied from $msoss'$, if there exists a memory store operation slice $msos'$ in $msoss'$, and an overlapping memory store operation slice from a different memory store operation in $msoss$, and $msos'$ is not from an instruction that is a program order successor of $i$, restart $i'$ and its data-flow dependents (including program order successors of restarted load-acquire instructions).
\end{enumerate}

\begin{commentary}
Forwarding memory store operations to a memory load operation might satisfy only some slices of the request, leaving other slices unsatisfied.

A consequence of the above is that store-release-RCsc memory store operations cannot be forwarded to load-acquires-RCsc:
a load-acquire-RCsc instruction cannot be in state ``{\sc Pending\_mem\_loads} $load\_cont$'' before all the program order previous store-release-RCsc instructions are finished, and $msoss$ does not include memory store operations from finished stores (as those must be propagated memory store operations).
\end{commentary}


\paragraph{Satisfy memory load operation from memory}\label{omm:thread:sat_from_mem}
For a load instruction instance $i$ in state ``{\sc Pending\_mem\_loads} $load\_cont$'', and a memory load operation $mlo$ in $i.mem\_loads$, that has unsatisfied slices, the memory load operation can be satisfied from memory under the condition that if $i$ is an AMO or a successful {\em lr.w/d} then no other AMO or successful {\em lr.w/d} from a different thread has an outstanding lock on the memory store operations $mlo$ is trying to read from.
Action: let $msoss$ be the memory store operation slices from memory covering the unsatisfied slices of $mlo$, and apply the action of \nameref{omm:thread:sat_by_forwarding}.
In addition, if $i$ is an AMO or a successful {\em lr.w/d}, union $msoss$ with the set of memory store operation slices $mlo$ is mapped to in the atomics map.

\begin{commentary}
Note that \nameref{omm:thread:sat_by_forwarding} might leave some slices of the memory load operation unsatisfied.
\nameref{omm:thread:sat_from_mem}, on the other hand, will always satisfy all the unsatisfied slices of the memory load operation.
\end{commentary}


\paragraph{Complete load instruction}\label{omm:thread:complete_load}
A load instruction instance $i$ in state ``{\sc Pending\_mem\_loads} $load\_cont$'' can be completed (not to be confused with finished) if all the memory load operations $i.mem\_loads$ are entirely satisfied (i.e.~there are no unsatisfied slices).
Action: update the state of $i$ to ``{\sc Plain} $load\_cont(memory\_value)$'', where $memory\_value$ is assembled from all the memory store operation slices that satisfied $i.mem\_loads$.


\paragraph{Finish load part of AMO instruction}\label{omm:thread:finish_load_part}
An AMO instruction instance $i$ that has completed the load part, i.e., \nameref{omm:thread:complete_load} has been taken, can be marked as such if the conditions of \nameref{omm:thread:finish} are satisfied, except for the fully determined data condition for register {\em rs2}.
Action: mark the load part of $i$ as finished. If later on $i$ has to be restarted, reset its state to the current state.


% \paragraph{Guarantee the success of {\em sc.w/d}}\label{omm:thread:excl_success}
% A {\em sc.w/d} instruction instance $i$ with next pseudocode state {\sc Atomic\_res}($res\_cont$) can be guaranteed to succeed if:
% \begin{enumerate}
% \item $i$ has not been made to fail (as recorded in $i.successful\_atomic$);
% \item $i$ is paired with a {\em lr.w/d} $i'$; and
% \item if $i'$ has already been satisfied (not necessarily entirely), let $msoss$ be the set of propagated write slices $i'$ has read from, then, no slice in $msoss$ has been overwritten (in memory) by a write from a different thread, and no other AMO or successful {\em lr.w/d} from a different thread has an outstanding lock on a write slice from $msoss$.
% \end{enumerate}
% Action:
% \begin{enumerate}
% \item record in $i.successful\_atomic$ that $i$ will be successful;
% \item if $i'$ has already been satisfied, union $msoss$ with the set of write slices the memory load operation of $i'$ is mapped to in the atomics map, where $msoss$ is as above; and
% \item update the state of $i$ to ``{\sc Plain} $res\_cont(true)$''.
% \end{enumerate}
%
%
% \paragraph{Make a {\em sc.w/d} fail}\label{omm:thread:excl_fail}
% A {\em sc.w/d} instruction instance $i$ with next pseudocode state {\sc Atomic\_res}($res\_cont$) can be made to fail if the {\em sc.w/d} has not been guaranteed to succeed (as recorded in $i.successful\_atomic$).
% Action:
% \begin{enumerate}
% \item record in $i.successful\_atomic$ that the {\em sc.w/d} was made to fail; and
% \item update the state of $i$ to ``{\sc Plain} $res\_cont(false)$''.
% \end{enumerate}
%
% \begin{commentary}
% Note that the promise-success transition is enabled before the {\em sc.w/d} commits, and we do not require it to have a fully-determined address or to be non-restartable.
% As a result, a {\em sc.w/d} that has already promised its success might be restarted.
% Since other instructions may rely on its promise, the restart will not affect the value of $i.successful\_atomic$.
% Instead, when the {\em sc.w/d} is restarted it will take the same promise/failure transition as before its restart --- based on the value of $i.successful\_atomic$.
% \end{commentary}

\paragraph{Initiate memory store operation footprints of store instruction}\label{omm:thread:announce_mem_write_footprint}
An instruction instance $i$ with next pseudocode state {\sc Store\_ea}($store\_kind$, $address$, $size$, $next\_state$) can announce its pending memory store operation footprint.
Action:
\begin{enumerate}
\item construct the appropriate memory store operations $msos$:
  \begin{itemize}
  \item if $address$ is aligned to $size$ then $msos$ is a single memory store operation of $size$ bytes to $address$;
  \item otherwise $msos$ is a set of $size$ memory store operations, each of one byte size, to the addresses $address\ldots address+size-1$.
  \end{itemize}
\item set $i.mem\_stores$ to $msos$; and
\item update the state of $i$ to ``{\sc Plain} $next\_state$''.
\end{enumerate}

\begin{commentary}
Note that after taking the transition above the memory store operations do not yet have their values, but other, non-overlapping program order following memory store operations, can already propagate.
\end{commentary}


\paragraph{Instantiate memory store operation values of store instruction}\label{omm:thread:initiate_mem_write}
An instruction instance $i$ with next pseudocode state {\sc Store\_memv}($memory\_value$, $store\_cont$) can initiate the values of the memory store operations $i.mem\_stores$.
Action:
\begin{enumerate}
\item split $memory\_value$ between the memory store operations $i.mem\_stores$; and
\item update the state of $i$ to ``{\sc Pending\_mem\_stores} $store\_cont$''.
\end{enumerate}


\paragraph{Commit store instruction}\label{omm:thread:commit_store}
For an uncommitted store instruction $i$ in state ``{\sc Pending\_mem\_stores} $store\_cont$'', $i$ can be committed if:
\begin{enumerate}
\item $i$ has fully determined data (i.e.~the register reads cannot change, see Section~\ref{sec:aux});
\item all program order previous conditional branch instructions are finished;
\item all program order previous {\em fence} instructions with {\em .sw} set are finished;
\item all program order previous {\em fence.i} instructions are finished; \fixme{remove?}
\item all program order previous load-acquire instructions are finished;
\item  if $i$ is a store-release, all program order previous memory access instructions are finished;
\item\label{omm:commit_store:prev_addrs} all program order previous memory access instructions have a fully determined memory footprint;
\item\label{omm:commit_store:prev_stores} all program order previous store instructions, except for {\em sc.w/d} that failed, have initiated and so have non-empty $mem\_stores$; and
\item\label{omm:commit_store:prev_loads} all program order previous load instructions have initiated and so have non-empty $mem\_loads$.
\end{enumerate}
Action: record $i$ as committed.

\begin{commentary}
Notice that if condition \ref{omm:commit_store:prev_addrs} is satisfied the conditions \ref{omm:commit_store:prev_stores} and \ref{omm:commit_store:prev_loads} are also satisfied, or will be satisfied after taking just eager transitions.
Hence requiring them does not strengthen the model.
It guarantees that previous memory access instructions have taken enough transitions to make the memory operations visible for the condition checks of the \nameref{omm:thread:prop_mem_write} transition.
\end{commentary}


\paragraph{Propagate memory store operation}\label{omm:thread:prop_mem_write}
For an instruction $i$ in state ``{\sc Pending\_mem\_stores} $store\_cont$'', and an unpropagated memory store operation $mso$ in $i.mem\_stores$, the memory store operation can be propagated if:
\begin{enumerate}
\item all memory memory store operations of program order previous store instructions that overlap with $mso$ have already propagated;
\item all memory load operations of program order previous load instructions that overlap with $mso$ have already been satisfied, and (the load instructions) are non-restartable (see \nameref{restart_condition});
\item all memory load operations satisfied by forwarding $mso$ are entirely satisfied; and
\item no AMO or successful {\em lr.w/d} from a different thread has an outstanding lock on a memory store operation slice that overlaps with $mso$.
\end{enumerate}
Action:
\begin{enumerate}
\item update the memory with $mso$;
\item record $mso$ as propagated;
\item restart any speculative instructions which have violated coherence as a result of this, i.e., for every non-finished instruction $i'$ program order after $i$ and every memory load operation $mlo'$ of $i'$ that was satisfied from $msoss'$, if there exists a memory store operation slice $msos'$ in $msoss'$ that overlaps with $mso$ and is not from $mso$, and $msos'$ is not from a program order successor of $i$, restart $i'$ and its data-flow dependents; and
\item for every AMO and successful {\em lr.w/d} that has read from $mso$ (by forwarding), add the slices of $mso$ this load reads from to the set of memory store operation slices the memory load operation of the load is mapped to in the exclusives map.
\end{enumerate}

\paragraph{Complete store instruction}\label{omm:thread:complete_store}
A store instruction $i$ in state ``{\sc Pending\_mem\_stores} $store\_cont$'', for which all the memory store operations in $i.mem\_stores$ have been propagated, can be completed (not to be confused with finished).
Action: update the state of $i$ to ``{\sc Plain} $store\_cont(true)$''.


\paragraph{Commit fence}\label{omm:thread:commit_barrier}
A fence instruction $i$ in state ``{\sc Plain} $next\_state$'' where $next\_state$ is {\sc Fence}($fence\_kind$, $next\_state'$) can be committed if:
\begin{enumerate}
\item all program order previous conditional branch instructions are finished;
\fixme{this looks stronger than intended, but actually it has no observable effect for most fences; the exception is ``fence w,r'', e.g., MP+fence.w.w+ctrlfence.w.r is allowed by Daniel's model but forbidden as a consequence of this condition. Fixing this means changing the invariant that finished instructions are never discarded, to finished load/store instructions are never discarded. Is RISC-V going to include ``fence w,r''?}
\item if $i$ has {\em .pr} set, all program order previous load instructions are finished;
\item if $i$ has {\em .pw} set, all program order previous store instructions are finished; and
\item if $i$ is a {\em fence.i} instruction, all program order previous memory access instructions have fully determined memory footprints. \fixme{remove?}
\end{enumerate}
Action: update the state of $i$ to ``{\sc Plain} $next\_state'$''.


\paragraph{Register read}\label{omm:thread:reg_read}
An instruction instance $i$ with next pseudocode state {\sc Read\_reg}($reg\_name$, $read\_cont$) can do a $reg\_name$ register read of if every instruction instance that it needs to read from has already performed the expected $reg\_name$ register write.

Let $read\_sources$ include, for each bit of $reg\_name$, the write to that bit by the most recent (in program order) instruction instance that can write to that bit, if any. If there is no such instruction, the source is the initial register value from $initial\_register\_state$.
Let  $register\_value$ be the value assembled from $read\_sources$.

Action:
\begin{enumerate}
\item add $reg\_name$ to $i.reg\_reads$ with $read\_sources$ and $register\_value$; and
\item update the state of $i$ to ``{\sc Plain} $read\_cont(register\_value)$''.
\end{enumerate}


\paragraph{Register write}\label{omm:thread:reg_write}
An instruction instance $i$ with next pseudocode state {\sc Write\_reg}($reg\_name$, $register\_value$, $next\_state'$) can do a $reg\_name$ register write.
Action:
\begin{enumerate}
\item add $reg\_name$ to $i.reg\_writes$ with $write\_deps$ and $register\_value$; and
\item update the state of $i$ to ``{\sc Plain} $next\_state'$''.
\end{enumerate}
where $write\_deps$ is the set of all $read\_sources$ from $i.reg\_reads$ and a flag that is set to true if $i$ is a load instruction that has already been entirely satisfied.


\paragraph{Pseudocode internal step}\label{omm:thread:sail_interp}
An instruction instance $i$ with next pseudocode state {\sc Internal}($next\_state'$) can do that pseudocode-internal step.
Action: Update the state of $i$ to ``{\sc Plain} $next\_state'$''.


\paragraph{Finish instruction}\label{omm:thread:finish}
A non-finished instruction $i$ with next pseudocode state {\sc Done} can be finished if:
\begin{enumerate}
\item if $i$ is a load instruction:
  \begin{enumerate}
  \item all program order previous load-acquire instructions are finished; and
  \item it is guaranteed that the values read by the memory load operations of $i$ will not cause coherence violations, i.e., for any program order previous instruction instance $i'$, let $cfp$ be the combined footprint of propagated memory store operations from store instructions program order between $i$ and $i'$ and fixed memory store operations that were forwarded to $i$ from store instructions program order between $i$ and $i'$ including $i'$, and let $cfp'$ be the complement of $cfp$ in the memory footprint of $i$.
  If $cfp'$ is not empty:
    \begin{enumerate}
    \item $i'$ has a fully determined memory footprint;
    \item $i'$ has no unpropagated memory store operations that overlap with $cfp'$; and
    \item if $i'$ is a load with a memory footprint that overlaps with $cfp'$, then all the memory load operations of $i'$ that overlap with $cfp'$ are satisfied and $i'$ can not be restarted (see \nameref{restart_condition}).
    \end{enumerate}
  Here, a memory store operation is called fixed if the store instruction has a fully determined data.
  \end{enumerate}
\item $i$ has a fully determined data; and
\item all program order previous conditional branch and indirect jump instructions are finished.
\end{enumerate}
Action:
\begin{enumerate}
\item if $i$ is a conditional branch or indirect jump instruction, discard any untaken path of execution, i.e., remove any (non-finished) instructions that are not reachable by the branch/jump taken in $instruction\_tree$; and
\item record the instruction as finished, i.e., set $finished$ to $true$.
\end{enumerate}



%%> \subsubsection{Auxiliary Definitions}\label{sec:aux}
%%>
%%> \paragraph{Fully determined}
%%> % This is the
%%> Informally, an instruction is said to have \emph{fully determined data}
%%> if the load instructions feeding its input registers are finished.
%%> Similarly, it is said to have \emph{fully determined memory footprint}
%%> if the load instructions feeding its memory location register are finished.
%%> %
%%> Formally, we first define the notion of \emph{fully determined register write}:
%%> a register write $mso$, of instruction $i$, with the associated
%%> \lem{write\_deps} from \lem{i.reg\_writes} is said to be \emph{fully determined}
%%> if one of the following conditions hold:
%%> \begin{enumerate}
%%>   \item $i$ is finished\ifexcl{\ (just the load part for AMOs)}; or
%%>   \item the load flag in \lem{write\_deps} is \lem{false} and every register
%%>   write in \lem{write\_deps} is fully determined.
%%> \end{enumerate}
%%> %
%%> Now, an instruction $i$ is said to have a \emph{fully determined data} if all the
%%> register writes of \lem{read\_sources} in \lem{i.reg\_reads} are fully determined;
%%> %
%%> and, $i$ is said to have a \emph{fully determined memory footprint}
%%> if all the register writes of \lem{read\_sources} in \lem{i.reg\_reads} that are
%%> associated with registers that feed into $i$'s memory access footprint are fully
%%> determined.
%%>
%%>
%%> \paragraph{Restart condition}\label{restart_condition}
%%> To determine if instruction $i$ might be restarted we use the
%%> following condition:
%%> $i$ is a non-finished instruction and at least one of the following holds,
%%> \begin{enumerate}
%%>     \item there exists a store instruction \lem{s} and an unpropagated write $mso$ of \lem{s} such that applying the
%%>     action of the \nameref{omm:thread:prop_mem_write} transition to $mso$ will
%%>     result in the restart of $i$;
%%>
%%>     \item there exists a non-finished load instruction \lem{l} and a memory load operation \lem{rr} of \lem{l} such that applying the
%%>     action of the \nameref{omm:thread:sat_from_mem} transition to \lem{rr}
%%>     will result in the restart of $i$ (even if \lem{rr} is already satisfied); or
%%>
%%>     \item there exists a non-finished instruction $i'$ that might be restarted and
%%>     $i$ is in its data-flow dependents\ifrelacq{, or $i'$ is a load-acquire}.
%%> \end{enumerate}
%%>
%%>
%%> \comment[Shaked]{
%%> \subsubsection{to do to the model}
%%>
%%> push ``potential mem write'' -> ``pending mem write'' everywhere
%%>
%%> shaked to fix fully-determined-address (and turn it into footprint)
%%>
%%> }

